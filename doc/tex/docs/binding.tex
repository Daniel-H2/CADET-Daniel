%!TEX root = ../all.tex
% =============================================================================
%  CADET - The Chromatography Analysis and Design Toolkit
%  
%  Copyright © 2008-2019: The CADET Authors
%            Please see the AUTHORS and CONTRIBUTORS file.
%  
%  All rights reserved. This program and the accompanying materials
%  are made available under the terms of the GNU Public License v3.0 (or, at
%  your option, any later version) which accompanies this distribution, and
%  is available at http://www.gnu.org/licenses/gpl.html
% =============================================================================

\section{Binding models}

The following binding models \index{Binding!Models}\index{Model!Binding model} are presented in dynamic binding mode.
By replacing all occurrences of $\mathrm{d}q / \mathrm{d}t$ with $0$, quasi-stationary (rapid-equlibrium) binding mode is achieved. 
In quasi-stationary binding it is assumed that ad- and desorption take place on a much faster time scale than the other transport processes such that bead liquid phase $c_{p,i}$ (or bulk liquid phase $c_i$ for certain unit operation models) are always in equilibrium with the solid phase $q_i$.

\paragraph{Equilibrium constants}
\phantomsection\label{par:MBEquilibriumConstants}

For the quasi-stationary binding mode, adsorption and desorption rate are no longer separate entities. \index{Binding!Equilibrium constant}
Instead, the quotient $k_{\text{eq}} = k_a / k_d$ of adsorption and desorption coefficient is the relevant parameter as shown for the linear binding model (see Section~\ref{sec:MBLinear}):
\begin{align*}
  \frac{\mathrm{d} q_i}{\mathrm{d} t} &= k_{a,i} c_{p,i} - k_{d,i} q_i \qquad \Rightarrow 0 = k_{a,i} c_{p,i} - k_{d,i} q_i \qquad \Leftrightarrow q_i = \frac{k_{a,i}}{k_{d,i}} c_{p,i} = k_{\text{eq},i} c_{p,i}.
\end{align*}
The equilibrium constant $k_{\text{eq},i}$ is used in CADET by setting $k_{d,i} = 1$ and $k_{a,i} = k_{\text{eq},i}$.

\paragraph{Correlation of ad- and desorption rates}

Note that adsorption rate $k_{a,i}$ and desorption rate $k_{d,i}$ are linearly correlated in both binding modes due to the form of the equilibrium constant $k_{\text{eq}}$:
\begin{align*}
  k_{a,i} = k_{\text{eq}} k_{d,i}.
\end{align*}
This correlation can potentially degrade performance of some optimization algorithms.
While in quasi-stationary binding mode this is prevented by using the technique above, a dynamic binding model has to be reparameterized in order to decouple parameters:
\begin{align*}
  \frac{\mathrm{d} q_i}{\mathrm{d} t} &= k_{a,i} c_{p,i} - k_{d,i} q_i = k_{d,i} \left[ k_{\text{eq},i} c_{p,i} - q_i \right] = k_{a,i} \left[ c_{p,i} - \frac{1}{k_{\text{eq},i}} q_i \right].
\end{align*}
This can be achieved by a (nonlinear) parameter transform
\begin{align*}
  F\left( k_{\text{eq},i}, k_{d,i} \right) = \begin{pmatrix} k_{\text{eq},i} k_{d,i} \\ k_{d,i} \end{pmatrix} \text{ with Jacobian } J_F\left( k_{\text{eq},i}, k_{d,i} \right) = \begin{pmatrix} k_{d,i} & k_{\text{eq},i} \\ 0 & 1 \end{pmatrix}.
\end{align*}

\paragraph{Dependence on external function}
\phantomsection\label{par:MBExternalFunctions}

A binding model may depend on an external function or profile $T\colon \left[ 0, T_{\text{end}}\right] \times [0, L] \to \mathds{R}$, where $L$ denotes the physical length of the unit operation, or $T\colon \left[0, T_{\text{end}}\right] \to \mathds{R}$ if the unit operation model has no axial length. \index{Binding!External function}
By using an external profile it is possible to account for effects that are not directly modeled in CADET (e.g., temperature).
The dependence of each parameter is modeled by a polynomial of third degree.
For example, the adsorption rate $k_a$ is really given by
\begin{align*}
  k_a(T) &= k_{a,3} T^3 + k_{a,2} T^2 + k_{a,1} T + k_{a,0}.
\end{align*}
While $k_{a,0}$ is set by the original parameter \texttt{XXX\_KA} of the file format (\texttt{XXX} being a placeholder for the binding model), the parameters $k_{a,3}$, $k_{a,2}$, and $k_{a,1}$ are given by \texttt{XXX\_KA\_TTT}, \texttt{XXX\_KA\_TT}, and \texttt{XXX\_KA\_T}, respectively.
The identifier of the externally dependent binding model is constructed from the original identifier by prepending \texttt{EXT\_} (e.g., \texttt{MULTI\_COMPONENT\_LANGMUIR} is changed into \texttt{EXT\_MULTI\_COMPONENT\_LANGMUIR}).
This pattern applies to all parameters and supporting binding models (see Table~\ref{tab:MBFeatureMatrix}).
Note that the parameter units have to be adapted to the unit of the external profile by dividing with an appropriate power.

Each parameter of the externally dependent binding model can depend on a different external source.
The 0-based indices of the external source for each parameter is given in the dataset \texttt{EXTFUN}.
By assigning only one index to \texttt{EXTFUN}, all parameters use the same source.
The ordering of the parameters in \texttt{EXTFUN} is given by the ordering in the file format specification in Section~\ref{sec:FFAdsorption}.

\paragraph{Binding model feature matrix}
\phantomsection\label{par:MBFeatureMatrix}

A short comparison of the most prominent binding model features is given in Table~\ref{tab:MBFeatureMatrix}.
The implemented binding models can be divided into two main classes: Single-state and multi-state binding.
While single-state models only have one bound state per component (or less), multi-state models provide multiple (possibly different) bound states for each component, which may correspond to different binding orientations or binding site types.
The models also differ in whether a mobile phase modifier (e.g., salt) is supported to modulate the binding behavior.

\begin{table}[!ht]
\centering
\begin{tabu}to \linewidth[m]{X>{\centering\arraybackslash}p{1.8cm}>{\centering\arraybackslash}p{2cm}>{\centering\arraybackslash}p{1.4cm}>{\centering\arraybackslash}p{1.8cm}} \toprule
\rowfont[c]\normalfont Binding model & Competitive & Mobile phase modifier & External function & Multi-state \\ \midrule
Linear & \xmark & \xmark & \cmark & \xmark \\
Multi component Langmuir & \cmark & \xmark & \cmark & \xmark \\
Multi component Anti-Langmuir & \cmark & \xmark & \cmark & \xmark \\
Steric mass action & \cmark & \cmark & \cmark & \xmark \\
Self association & \cmark & \cmark & \cmark & \xmark \\
Mobile phase modulator Langmuir & \cmark & \cmark & \cmark & \xmark \\
Kumar-Langmuir & \cmark & \cmark & \cmark & \xmark \\
Saska & \xmark & \xmark & \cmark & \xmark \\
Multi component Bi-Langmuir & \cmark & \xmark & \cmark & \cmark \\
Multi component spreading & \cmark & \xmark & \cmark & \cmark \\
Multi-state steric mass action & \cmark & \cmark & \cmark & \cmark \\
Simplified multi-state steric mass action & \cmark & \cmark & \xmark & \cmark \\
Bi steric mass action & \cmark & \cmark & \cmark & \cmark \\
\bottomrule
\end{tabu}
\caption{\label{tab:MBFeatureMatrix}Supported features of the different binding models}
\end{table}

\paragraph{Reference concentrations}
\phantomsection\label{par:MBReferenceConcentrations}

Some binding models use reference concentrations $c_{\text{ref}}$ and $q_{\text{ref}}$ of the mobile phase modulator (e.g., salt) in the particle liquid and solid phase, respectively. \index{Binding!Reference concentrations}
The reference values are mainly used for normalizing adsorption and desorption rates, but also for other parameters that appear with those concentrations.
They amount to a simple parameter transformation that is exemplified at one equation of the steric mass action binding model
\begin{align*}
  \frac{\mathrm{d} q_i}{\mathrm{d} t} = k_{a,i} c_{p,i}\bar{q}_0^{\nu_i} - k_{d,i} q_i c_{p,0}^{\nu_i},
\end{align*}
where $c_{p,0}$ denotes the mobile phase salt concentration and
\begin{align*}
  \bar{q}_0 = \Lambda - \sum_{j=1}^{N_{\text{comp}} - 1} \left( \nu_j + \sigma_j \right) q_j
\end{align*}
is the number of available binding sites which is related to the number of bound salt ions.
Using the parameter transformation
\begin{align*}
  k_{a,i} &= \tilde{k}_{a,i} q_{\text{ref}}^{-\nu_i}, \\
  k_{d,i} &= \tilde{k}_{d,i} c_{\text{ref}}^{-\nu_i},
\end{align*}
we obtain the modified model equation
\begin{align*}
  \frac{\mathrm{d} q_i}{\mathrm{d} t} = \tilde{k}_{a,i} c_{p,i} \left(\frac{\bar{q}_0}{q_{\text{ref}}}\right)^{\nu_i} - \tilde{k}_{d,i} q_i \left(\frac{c_{p,0}}{c_{\text{ref}}}\right)^{\nu_i}.
\end{align*}
This transformation serves as a (partial) nondimensionalization of the adsorption and desorption rates and, by properly choosing the reference concentrations $c_{\text{ref}}$ and $q_{\text{ref}}$, may improve the optimizer performance.

Recommended choices for $c_{\text{ref}}$ are the average or maximum inlet concentration of the mobile phase modifier $c_0$, and for $q_{\text{ref}}$ the ionic capacity $\Lambda$.
Note that setting the reference concentrations to $\num{1.0}$ each results in the original binding model.

\subsection{Linear}\label{sec:MBLinear}

A linear \index{Binding!Linear} binding model, which is often employed for low concentrations or in analytic settings \cite{Guiochon2006}.
\begin{align*}
  \frac{\mathrm{d} q_i}{\mathrm{d} t} = k_{a,i} c_{p,i} - k_{d,i} q_i && i = 0, \dots, N_{\text{comp}} - 1.
\end{align*}
See Table~\ref{tab:FFAdsorptionLinear}.

\subsection{Multi Component Langmuir}\label{sec:MBLangmuir}

The Langmuir binding model includes a saturation term and takes into account the capacity of the resin \cite{Langmuir1916,Guiochon2006}. \index{Binding!Multi component Langmuir}
All components compete for the same binding sites.
\begin{align*}
  \frac{\mathrm{d} q_i}{\mathrm{d} t} = k_{a,i}\: c_{p,i}\: q_{\text{max},i} \left( 1 - \sum_{j=0}^{N_{\text{comp}} - 1} \frac{q_j}{q_{\text{max},j}} \right) - k_{d,i} q_i && i = 0, \dots, N_{\text{comp}} - 1.
\end{align*}
See Table~\ref{tab:FFAdsorptionMultiCompLangmuir}.

\subsection{Multi Component Anti-Langmuir}\label{sec:MBAntiLangmuir}

The Anti-Langmuir (or generalized Langmuir) binding model extends the Langmuir model (see Section~\ref{sec:MBLangmuir}). \index{Binding!Multi component Anti-Langmuir}
The factor $p_j \in \{ -1, 1 \}$ determines the shape of the isotherm.
For $p_j = 1$ (standard Langmuir) the chromatograms have sharp fronts and a dispersed tail (isotherm is concave).
In case of the Anti-Langmuir ($p_j = -1$) it is the other way around (isotherm is convex).
\begin{align*}
    \frac{\mathrm{d} q_i}{\mathrm{d} t} = k_{a,i} c_{p,i} q_{\text{max},i} \left( 1 - \sum_{j=0}^{N_{\text{comp}} - 1} p_j \frac{q_j}{q_{\text{max},j}} \right) - k_{d,i} q_i && i = 0, \dots, N_{\text{comp}} - 1.
\end{align*}
See Table~\ref{tab:FFAdsorptionMultiCompAntiLangmuir}.

\subsection{Steric Mass Action}\label{sec:MBStericMassAction}

The steric mass action model takes charges of the molecules into account \cite{Brooks1992} and is, thus, often used in ion-exchange chromatography. \index{Binding!Steric mass action}
Each component has a characteristic charge $\nu$ that determines the number of available binding sites $\Lambda$ (ionic capacity) used up by a molecule.
Due to the molecule's shape, some additional binding sites (steric shielding factor $\sigma$) may be shielded from other molecules and are not available for binding.
\begin{align*}
  \frac{\mathrm{d} q_i}{\mathrm{d} t} = k_{a,i} c_{p,i}\left( \frac{\bar{q}_0 }{q_{\text{ref}}} \right)^{\nu_i} - k_{d,i}\: q_i\: \left(\frac{c_{p,0}}{c_{\text{ref}}}\right)^{\nu_i} && i = 1, \dots, N_{\text{comp}} - 1,
\end{align*}
where $c_{p,0}$ and $q_0$ denote the salt concentrations in the liquid and solid phase of the beads, respectively.
The number of free binding sites
\begin{align*}
  \bar{q}_0 = \Lambda - \sum_{j=1}^{N_{\text{comp}} - 1} \left( \nu_j + \sigma_j \right) q_j = q_0 - \sum_{j=1}^{N_{\text{comp}} - 1} \sigma_j q_j
\end{align*}
is calculated from the number of bound counter ions $q_0$ by taking steric shielding into account.
In turn, the number of bound counter ions $q_0$ (electro-neutrality condition) is given by
\begin{align*}
  q_0 = \Lambda - \sum_{j=1}^{N_{\text{comp}} - 1} \nu_j q_j,
\end{align*}
which also compensates for the missing equation for $\frac{\mathrm{d} q_0}{\mathrm{d}t}$.
See Table~\ref{tab:FFAdsorptionStericMassAction}.

The concept of reference concentrations ($c_{\text{ref}}$ and $q_{\text{ref}}$) is explained in the respective paragraph in Section~\ref{par:MBReferenceConcentrations}.

\subsection{Self Association}

This binding model is similar to the steric mass action model (see Section~\ref{sec:MBStericMassAction}) but is also capable of describing dimerization \cite{Mollerup2008,Westerberg2012}. \index{Binding!Self association}
The dimerization, which is the immobilization of protein at some already bound protein, is also termed ``self-association''.
It is modeled by adding a quadratic (in $c_{p,i}$) term to the adsorption part of the equation.
\begin{align*}
  \frac{\mathrm{d} q_i}{\mathrm{d} t} &= c_{p,i}\left( \frac{\bar{q}_0}{q_{\text{ref}}} \right)^{\nu_i} \left[ k_{a,i,1} + k_{a,i,2} c_{p,i} \right] - k_{d,i}\: q_i\: \left(\frac{c_{p,0}}{c_{\text{ref}}}\right)^{\nu_i} && i = 1, \dots, N_{\text{comp}} - 1, \\
  q_0 &= \Lambda - \sum_{j=1}^{N_{\text{comp}} - 1} \nu_j q_j,
\end{align*}
where the number of available binding sites is given by
\begin{align*}
  \bar{q}_0 = \Lambda - \sum_{j=1}^{N_{\text{comp}} - 1} \left( \nu_j + \sigma_j \right) q_j = q_0 - \sum_{j=1}^{N_{\text{comp}} - 1} \sigma_j q_j.
\end{align*}
See Table~\ref{tab:FFAdsorptionSelfAssociation}.

The concept of reference concentrations ($c_{\text{ref}}$ and $q_{\text{ref}}$) is explained in the respective paragraph in Section~\ref{par:MBReferenceConcentrations}.

\subsection{Mobile Phase Modulator Langmuir}

This model is a modified Langmuir model (see Section~\ref{sec:MBLangmuir}) which can be used to describe hydrophobic interaction chromatography \cite{Melander1989,Karlsson2004}. \index{Binding!Mobile phase modulator}
A modulator component (termed ``salt'', $c_{p,0}$ and $q_0$) influences ad- and desorption processes:
\begin{align*}
  \frac{\mathrm{d} q_i}{\mathrm{d} t} = k_{a,i} e^{\gamma_i c_{p,0}} c_{p,i}\: q_{\text{max},i} \left( 1 - \sum_{j=1}^{N_{\text{comp}} - 1} \frac{q_j}{q_{\text{max},j}} \right) - k_{d,i} \: c_{p,0}^{\beta_i} \: q_i && i = 1, \dots, N_{\text{comp}} - 1.
\end{align*}
where $c_{p,0}$ and $q_0$ denote the salt concentrations in the liquid and solid phase of the beads respectively. 
Salt is considered to be inert, therefore either
\begin{align*}
  \frac{\mathrm{d} q_0}{\mathrm{d} t} = 0
\end{align*}
is used if salt has one bound state, or salt can be used without a bound state.
The parameter $\gamma$ describes the hydrophobicity and $\beta$ the ion-exchange characteristics.
See Table~\ref{tab:FFAdsorptionMobilePhaseModulator}.

\subsection{Kumar-Langmuir}

This extension of the Langmuir isotherm (see Section~\ref{sec:MBLangmuir}) developed in \cite{Kumar2015} was used to model charge variants of monoclonal antibodies in ion-exchange chromatography. \index{Binding!Kumar-Langmuir}
A non-binding salt component $c_{p,0}$ is added to modulate the ad- and desorption process.
\begin{align*}
  \frac{\mathrm{d} q_i}{\mathrm{d} t} &= k_{a,i} \exp\left( \frac{k_{\text{act},i}}{T} \right) c_{p,i} q_{\text{max},i} \left( 1 - \sum_{j=1}^{N_{\text{comp}} - 1} \frac{q_j}{q_{\text{max},j}} \right) - k_{d,i} \left( c_{p,0} \right)^{\nu_i} q_i && i = 1, \dots, N_{\text{comp}} - 1
\end{align*}
In this model, the true adsorption rate $k_{a,i,\text{true}}$ is governed by the Arrhenius law in order to take temperature into account
\begin{align*}
  k_{a,i,\text{true}} = k_{a,i} \exp\left( \frac{k_{\text{act},i}}{T} \right).
\end{align*}
Here, $k_{a,i}$ is the frequency or pre-exponential factor, $k_{\text{act},i} = E / R$ is the activation temperature ($E$ denotes the activation energy and $R$ the Boltzmann gas constant), and $T$ is the temperature.
The characteristic charge $\nu$ of the protein is taken into account by the power law.
See Table~\ref{tab:FFAdsorptionKumarLangmuir}.

\subsection{Saska}

In this binding model an additional quadratic term is added to the linear model \cite{Saska1992}. \index{Binding!Saska}
The quadratic term allows to take interactions of liquid phase components into account.
\begin{align*}
  \frac{\mathrm{d} q_i}{\mathrm{d} t} = H_i c_{p,i} + \sum_{j=0}^{N_{\text{comp}} - 1} k_{ij} c_{p,i} c_{p,j} - q_i && i = 0, \dots, N_{\text{comp}} - 1
\end{align*}
See Table~\ref{tab:FFAdsorptionSaska}.

\subsection{Multi Component Bi-Langmuir}\label{sec:MBBiLangmuir}

The multi component Bi-Langmuir model \cite{Guiochon2006} adds $M - 1$ \emph{additional} types of binding sites $q_{i,j}$ ($0 \leq j \leq M - 1$) to the Langmuir model (see Section~\ref{sec:MBLangmuir}) without allowing an exchange between the different sites $q_{i,j}$ and $q_{i,k}$ ($k \neq j$). \index{Binding!Multi component Bi-Langmuir}
Therefore, there are no competitivity effects between the different types of binding sites and they have independent capacities.
\begin{align*}
  \frac{\mathrm{d} q_{i,j}}{\mathrm{d} t} &=  k_{a,i}^{(j)}\: c_{p,i}\: q_{\text{max},i}^{(j)} \left( 1 - \sum_{k=0}^{N_{\text{comp}} - 1} \frac{q_{k,j}}{q_{\text{max},k}^{(j)}}\right) - k_{d,i}^{(j)} q_{i,j} & i = 0, \dots, N_{\text{comp}} - 1, \: j = 0, \dots, M - 1.% (0 \leq i \leq N_{\text{comp}} - 1, \: 0 \leq j \leq M - 1).
\end{align*}
Note that all binding components must have exactly the same number of binding site types $M \geq 1$.
See the Langmuir isotherm in Section~\ref{sec:MBLangmuir} and Table~\ref{tab:FFAdsorptionBiLangmuir}.

Originally, the Bi-Langmuir model is limited to two different binding site types.
Here, the model has been extended to arbitrary many binding site types.

\subsection{Multi Component Spreading}

The multi component spreading model adds a second bound state $q_{i,2}$ to the Langmuir model (see Section~\ref{sec:MBLangmuir}) and allows the exchange between the two bound states $q_{i,1}$ and $q_{i,2}$. \index{Binding!Multi component spreading}
In the spreading model a second state of the bound molecule (e.g., a different orientation on the surface or a different folding state) is added.
The exchange of molecules between the two states is allowed and, since the molecules can potentially bind in both states at the same binding site, competitivity effects are present.
This is different to the Bi-Langmuir model in which another type of binding sites is added and no exchange between the different bound states is considered (see Section~\ref{sec:MBBiLangmuir}). For all components $i = 0, \dots, N_{\text{comp}} - 1$ the equations are given by
\begin{align*}
  \frac{\mathrm{d} q_{i,1}}{\mathrm{d} t} &= \left( k_a^A\: c_{p,i} - k_{12} q_{i,1} \right) q_{\text{max},i}^A \left( 1 - \sum_{j=0}^{N_{\text{comp}} - 1} \frac{q_j^A}{q_{\text{max},j}^A} - \sum_{j=0}^{N_{\text{comp}} - 1} \frac{q_j^B}{q_{\text{max},j}^B} \right) - k_d^A q_{i,1} + k_{21} q_{i,2}, \\
  \frac{\mathrm{d} q_{i,2}}{\mathrm{d} t} &= \left( k_a^B\: c_{p,i} + k_{12} q_{i,1} \right) q_{\text{max},i}^A \left( 1 - \sum_{j=0}^{N_{\text{comp}} - 1} \frac{q_j^A}{q_{\text{max},j}^A} - \sum_{j=0}^{N_{\text{comp}} - 1} \frac{q_j^B}{q_{\text{max},j}^B} \right) - \left( k_d^B + k_{21} \right) q_{i,2}.
\end{align*}
See Table~\ref{tab:FFAdsorptionMultiCompSpreading}.

\subsection{Multi-State Steric Mass Action}\label{sec:MBMultiStateStericMassAction}

The multi-state steric mass action model adds $M_i-1$ \emph{additional} bound states $q_{i,j}$ ($j = 0, \dots, M_i - 1$) for each component $i$ to the steric mass action model (see Section~\ref{sec:MBStericMassAction}) and allows the exchange between the different bound states $q_{i,0}, \dots, q_{i,M-1}$ of each component. \index{Binding!Multi-state steric mass action}
In the multi-state SMA model a variable number of states of the bound molecule (e.g., different orientations on the surface, binding strength of tentacle adsorbers) is added which are more and more strongly bound, i.e.,
\begin{align*}
  \nu_{i,j} \leq \nu_{i,j+1} \qquad i = 1, \dots, N_{\text{comp}} - 1, \quad j = 0,\dots, M_i - 1.
\end{align*}
The exchange between the different states of each component is allowed and, since the molecules can potentially bind in all states at the same binding site, competitive effects are present.
\begin{align*}
  \frac{\mathrm{d} q_{i,j}}{\mathrm{d} t} =& \phantom{+} k_{a,i}^{(j)} c_{p,i} \left(\frac{\bar{q}_0}{q_{\text{ref}}}\right)^{\nu_{i,j}} - k_{d,i}^{(j)}\: q_{i,j}\: \left(\frac{c_{p,0}}{c_{\text{ref}}}\right)^{\nu_{i,j}} \\
  &- \sum_{\ell = 0}^{j-1} \underbrace{k^{(i)}_{j\ell}\: q_{i,j}\: \left(\frac{c_{p,0}}{c_{\text{ref}}}\right)^{\left(\nu_{i,j} - \nu_{i,\ell}\right)}}_{\text{to weak state}} - \sum_{\ell = j+1}^{M_i - 1} \underbrace{k^{(i)}_{j\ell}\: q_{i,j}\: \left(\frac{\bar{q}_0}{q_{\text{ref}}}\right)^{\left(\nu_{i,\ell} - \nu_{i,j}\right)}}_{\text{to strong state}} \\
  &+ \sum_{\ell = 0}^{j-1} \underbrace{k^{(i)}_{\ell j}\: q_{i,\ell}\: \left(\frac{\bar{q}_0}{q_{\text{ref}}}\right)^{\left(\nu_{i,j} - \nu_{i,\ell}\right)}}_{\text{from weak state}} + \sum_{\ell = j+1}^{M_i - 1} \underbrace{k^{(i)}_{\ell j}\: q_{i,\ell}\: \left(\frac{c_{p,0}}{c_{\text{ref}}}\right)^{\left(\nu_{i,\ell} - \nu_{i,j}\right)}}_{\text{from strong state}} & \begin{aligned}
    i &= 1, \dots, N_{\text{comp}} - 1, \\ j &= 0, \dots, M_i - 1,
  \end{aligned}
\end{align*}
where $c_{p,0}$ and $q_0$ denote the salt concentrations in the liquid and solid phase of the beads respectively. The number of available salt ions $\bar{q}_0$ is given by
\begin{align*}
  \bar{q}_0 = \Lambda - \sum_{j=1}^{N_{\text{comp}} - 1} \sum_{\ell=0}^{M_j - 1} \left( \nu_{j,\ell} + \sigma_{j,\ell} \right) q_{j,\ell}.
\end{align*}
A neutrality condition compensating for the missing equation for $\frac{\mathrm{d} q_0}{\mathrm{d}t}$ is required:
\begin{align*}
  q_0 = \Lambda - \sum_{j=1}^{N_{\text{comp}} - 1} \sum_{\ell=0}^{M_j - 1} \nu_{j,\ell} q_{j,\ell}.
\end{align*}
See Table~\ref{tab:FFAdsorptionMultiStateStericMassAction}.

The concept of reference concentrations ($c_{\text{ref}}$ and $q_{\text{ref}}$) is explained in the respective paragraph in Section~\ref{par:MBReferenceConcentrations}.

\subsection{Simplified Multi-State Steric Mass Action}

The simplified multi-state steric mass action is the same as the multi-state SMA model described above (see Section~\ref{sec:MBMultiStateStericMassAction}), but with additional assumptions: \index{Binding!Simplified multi-state steric mass action}
\begin{itemize}
  \item Molecules are only exchanged between two adjacent states, that is, no transfer from state $q_{i,1}$ to state $q_{i,3}$ is allowed.
  \item Characteristic charge $\nu_{i,j}$ and shielding factor $\sigma_{i,j}$ only depend on the index of the state $j$.
\end{itemize}
Thus, the exchange parameters $k^{(i)}_{j\ell}$, the characteristic charge $\nu_{i,j}$, and the shielding $\sigma_{i,j}$ can be parameterized with few degrees of freedom.
For all $i = 1,\dots,N_{\text{comp}} - 1$ and $j,\ell = 0,\dots,M_i - 1$ let
\begin{align*}
  k^{(i)}_{j\ell} &= \begin{cases}
    0, & \text{for } \abs{j-\ell} \neq 1 \\
    K^{(i)}_{ws} + j K^{(i)}_{ws,\text{lin}} - K^{(i)}_{ws,\text{quad}} j(j - M_i+2), & \text{for } \ell = j+1 \\
    K^{(i)}_{sw} + \ell K^{(i)}_{sw,\text{lin}} - K^{(i)}_{sw,\text{quad}} \ell(\ell - M_i+2), & \text{for } \ell = j-1,
  \end{cases}\\
  \nu_{i,j} &= \nu_{\text{min},i} + \frac{j}{M_i-1} \left( \nu_{\text{max},i} - \nu_{\text{min},i} \right) - \nu_{\text{quad},i} j (j-M_i+1), \\
  \sigma_{i,j} &= \sigma_{\text{min},i} + \frac{j}{M_i-1} \left( \sigma_{\text{max},i} - \sigma_{\text{min},i} \right) - \sigma_{\text{quad},i} j (j-M_i+1).
\end{align*}
Note that the characteristic charge $\nu_{i,j}$ has to be monotonically non-decreasing in the second index $j$ and all other rates and the steric factor $\sigma_{i,j}$ have to be non-negative.
See Table~\ref{tab:FFAdsorptionSimpleMultiStateStericMassAction}.

\subsection{Bi Steric Mass Action}

Similar to the Bi-Langmuir model (see Section~\ref{sec:MBBiLangmuir}), the Bi-SMA model adds $M - 1$ \emph{additional} types of binding sites $q_{i,j}$ ($0 \leq j \leq M - 1$) to the SMA model (see Section~\ref{sec:MBStericMassAction}) without allowing an exchange between the different sites $q_{i,j}$ and $q_{i,k}$ ($k \neq j$). \index{Binding!Bi steric mass action}
Therefore, there are no competitivity effects between the two types of binding sites and they have independent capacities.
\begin{align*}
  \frac{\mathrm{d} q_{i,j}}{\mathrm{d} t} &= k_{a,i,j} c_{p,i} \left(\frac{\bar{q}_{0,j}}{q_{\text{ref},j}} \right)^{\nu_{i,j}} - k_{d,i,j}\: q_{i,j}\: \left(\frac{c_{p,0}}{c_{\text{ref},j}}\right)^{\nu_{i,j}} & i = 1, \dots, N_{\text{comp}} - 1, \quad j = 0, \dots, M - 1,
\end{align*}
where $c_{p,0}$ and $q_{0,j}$ ($0 \leq j \leq M - 1$) denote the salt concentrations in the liquid and solid phases of the beads respectively. The number of available salt ions $\bar{q}_{0,j}$ for each binding site type $0 \leq j \leq M - 1$ is given by
\begin{align*}
  \bar{q}_{0,j} &= \Lambda_j - \sum_{k=1}^{N_{\text{comp}} - 1} \left( \nu_{k,j} + \sigma_{k,j} \right) q_{k,j}.
\end{align*}
Electro-neutrality conditions compensating for the missing equations for $\frac{\mathrm{d} q_{0,j}}{\mathrm{d}t}$ are required:
\begin{align*}
  q_{0,j} &= \Lambda_j - \sum_{k=1}^{N_{\text{comp}} - 1} \nu_{k,j} q_{k,j} & j = 0, \dots, M - 1.
\end{align*}
Note that all binding components must have exactly the same number of binding site types $M \geq 1$.
See Table~\ref{tab:FFAdsorptionBiStericMassAction}.

The reference concentrations $c_{\text{ref},j}$ and $q_{\text{ref},j}$ can be specified for each binding site type $0 \leq j \leq M - 1$.
The concept of reference concentrations is explained in the respective paragraph in Section~\ref{par:MBReferenceConcentrations}.

Originally, the Bi-SMA model is limited to two different binding site types.
Here, the model has been extended to arbitrary many binding site types.
