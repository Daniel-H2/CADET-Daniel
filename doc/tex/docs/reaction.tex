%!TEX root = ../all.tex
% =============================================================================
%  CADET - The Chromatography Analysis and Design Toolkit
%  
%  Copyright © 2008-2020: The CADET Authors
%            Please see the AUTHORS and CONTRIBUTORS file.
%  
%  All rights reserved. This program and the accompanying materials
%  are made available under the terms of the GNU Public License v3.0 (or, at
%  your option, any later version) which accompanies this distribution, and
%  is available at http://www.gnu.org/licenses/gpl.html
% =============================================================================

\section{Reaction models}

Reaction models \index{Reaction!Models}\index{Model!Reaction model} describe the (net) fluxes $f_{\text{react}}$ of a reaction mechanism.
The most common mechanism is the mass action law.

\paragraph{Correlation of forward- and backward rate constants}

Note that forward rate constant $k_{\text{fwd},i}$ and backward rate constant $k_{\text{bwd},i}$ of reaction $i$ are linearly correlated due to the form of the equilibrium constant $k_{\text{eq},i}$:
\begin{align*}
  k_{\text{fwd},i} = k_{\text{eq},i} k_{\text{bwd},i}.
\end{align*}
This correlation can potentially degrade performance of some optimization algorithms.
The parameters can be decoupled by reparameterization:
\begin{align*}
  r_{\text{net},i} &= k_{\text{fwd},i} f_{\text{fwd},i} - k_{\text{bwd},i} f_{\text{bwd},i} = k_{\text{bwd},i} \left[ k_{\text{eq},i} f_{\text{fwd},i} - f_{\text{bwd},i} \right] = k_{\text{fwd},i} \left[ f_{\text{fwd},i} - \frac{1}{k_{\text{eq},i}} f_{\text{bwd},i} \right].
\end{align*}
This can be achieved by a (nonlinear) parameter transform
\begin{align*}
  F\left( k_{\text{eq},i}, k_{\text{bwd},i} \right) = \begin{pmatrix} k_{\text{eq},i} k_{\text{bwd},i} \\ k_{\text{bwd},i} \end{pmatrix} \text{ with Jacobian } J_F\left( k_{\text{eq},i}, k_{\text{bwd},i} \right) = \begin{pmatrix} k_{\text{bwd},i} & k_{\text{eq},i} \\ 0 & 1 \end{pmatrix}.
\end{align*}

\paragraph{Dependence on external function}
\phantomsection\label{par:MRExternalFunctions}

A reaction model may depend on an external function or profile $T\colon \left[ 0, T_{\text{end}}\right] \times [0, L] \to \mathds{R}$, where $L$ denotes the physical length of the unit operation, or $T\colon \left[0, T_{\text{end}}\right] \to \mathds{R}$ if the unit operation model has no axial length. \index{Reaction!External function}
By using an external profile, it is possible to account for effects that are not directly modeled in \CADET{} (e.g., temperature).
The dependence of each parameter is modeled by a polynomial of third degree.
For example, the forward rate constant $k_{\text{fwd}}$ is really given by
\begin{align*}
  k_{\text{fwd}}(T) &= k_{\text{fwd},3} T^3 + k_{\text{fwd},2} T^2 + k_{\text{fwd},1} T + k_{\text{fwd},0}.
\end{align*}
While $k_{\text{fwd},0}$ is set by the original parameter \texttt{XXX\_KFWD} of the file format (\texttt{XXX} being a placeholder for the reaction model), the parameters $k_{\text{fwd},3}$, $k_{\text{fwd},2}$, and $k_{\text{fwd},1}$ are given by \texttt{XXX\_KFWD\_TTT}, \texttt{XXX\_KFWD\_TT}, and \texttt{XXX\_KFWD\_T}, respectively.
The identifier of the externally dependent reaction model is constructed from the original identifier by prepending \texttt{EXT\_} (e.g., \texttt{MASS\_ACTION\_LAW} is changed into \texttt{EXT\_MASS\_ACTION\_LAW}).
This pattern applies to all parameters and supporting reaction models.
Note that the parameter units have to be adapted to the unit of the external profile by dividing with an appropriate power.

Each parameter of the externally dependent reaction model can depend on a different external source.
The 0-based indices of the external source for each parameter is given in the dataset \texttt{EXTFUN}.
By assigning only one index to \texttt{EXTFUN}, all parameters use the same source.
The ordering of the parameters in \texttt{EXTFUN} is given by the ordering in the file format specification in Section~\ref{sec:FFReaction}.

\subsection{Mass action law}\label{sec:MRMassActionLaw}

The mass action law \index{Reaction!Mass action law} reaction model is suitable for most reactions.
Note that the concentrations are directly used for calculating the fluxes.
Hence, the model only holds for dilute solutions under the assumption of a well-stirred reaction vessel.
These assumptions can be weakened by passing to the generalized mass action law, which uses chemical activities instead of concentrations.

The mass action law states that the speed of a reaction is proportional to the product of the concentrations of their reactants.
The net flux for component $i$ is given by
\begin{align*}
  f_{\text{react},i}^l\left(c^l\right) = \sum_{j=0}^{N_{\text{react}}-1} s_{i,j}^l \varphi^l_j\left(c^l\right), \qquad \varphi^l_j(c^l) = k^l_{\text{fwd},j} \prod_{\ell=0}^{N_{\text{comp}}-1} \left(c^l_{\ell}\right)^{e^l_{\text{fwd},\ell,j}} - k^l_{\text{bwd},j} \prod_{\ell=0}^{N_{\text{comp}}-1} \left(c^l_{\ell}\right)^{e^l_{\text{bwd},\ell,j}},
\end{align*}
where $S^l = (s^l_{i,j}) \in \mathds{R}^{N_{\text{comp}} \times N_{\text{react}}}$ is the stoichiometric matrix, $\varphi^l_j(c)$ is the net flux of reaction $j$, and $k^l_{\text{fwd},j}$ and $k^l_{\text{bwd},j}$ are the rate constants.
The matrices $E^l_{\text{fwd}} = (e^l_{\text{fwd},\ell,j}) \in \mathds{R}^{N_{\text{comp}} \times N_{\text{react}}}$ and $E^l_{\text{bwd}} = (e^l_{\text{bwd},\ell,j}) \in \mathds{R}^{N_{\text{comp}} \times N_{\text{react}}}$ are usually derived by the order of the reaction, that is,
\begin{align}
	e^l_{\text{fwd},\ell,j} = -\min(0, s^l_{\ell,j}), \qquad e^l_{\text{bwd},\ell,j} = \max(0, -s^l_{\ell,j}). \label{eq:MRMassActionLawExpMatDefault}
\end{align}
However, these defaults can be changed by providing those matrices.

In situations where both liquid and solid phase are present (e.g., in a bead), the respective other phase may act as a modifier in the net flux equation.
For example, consider reactions in the liquid phase of a particle given by
\begin{align*}
  f_{\text{react},i}^p\left(c^p, c^s\right) &= \sum_{j=0}^{N_{\text{react}}-1} s_{i,j}^p \varphi^p_j\left(c^p, c^s\right),
\end{align*}
where
\begin{equation*} \begin{split}
  \varphi^p_j(c^p, c^s) &= k^p_{\text{fwd},j} \left[\prod_{\ell=0}^{N_{\text{comp}}-1} \left(c^p_{\ell}\right)^{e^p_{\text{fwd},\ell,j}}\right] \left[\prod_{m=0}^{\sum_{i=0}^{N_{\text{comp}}-1} N_{\text{bnd},i}-1} \left(c^s_{m}\right)^{e^{ps}_{\text{fwd},m,j}}\right] \\
  &\phantom{=}\quad - k^p_{\text{bwd},j} \left[\prod_{\ell=0}^{N_{\text{comp}}-1} \left(c^p_{\ell}\right)^{e^p_{\text{bwd},\ell,j}}\right] \left[\prod_{m=0}^{\sum_{i=0}^{N_{\text{comp}}-1} N_{\text{bnd},i}-1} \left(c^s_{m}\right)^{e^{ps}_{\text{bwd},m,j}}\right].
\end{split} \end{equation*}
The forward and backward rates of the liquid phase particle reactions can be modified by a power of every bound state in the solid phase of the particle.
The exponents of these powers are given by the matrices $E^{ps}_{\text{fwd}} = (e^{ps}_{\text{fwd},m,j})$ and $E^{ps}_{\text{bwd}} = (e^{ps}_{\text{bwd},m,j})$, which are both of size $(\sum_i N_{\text{bnd},i}) \times N_{\text{react}}$.
Whereas the exponent matrices $E^{p}_{\text{fwd}}, E^{p}_{\text{bwd}} \in \mathds{R}^{N_{\text{comp}} \times N_{\text{react}}}$ are initialized based on the stoichiometric matrix $S^{p} \in \mathds{R}^{N_{\text{comp}} \times N_{\text{react}}}$, see Eq.~\eqref{eq:MRMassActionLawExpMatDefault}, the exponent matrices $E^{ps}_{\text{fwd}}, E^{ps}_{\text{bwd}}$ of the modifier terms default to $0$.

Vice versa, the rates of solid phase reactions can be modified by liquid phase concentrations.
The corresponding exponent matrices $E^{sp}_{\text{fwd}} = (e^{sp}_{\text{fwd},\ell,j})$ and $E^{sp}_{\text{bwd}} = (e^{sp}_{\text{bwd},\ell,j})$ are both of size $N_{\text{comp}} \times N_{\text{react}}$.
\begin{align*}
  f_{\text{react},i}^s\left(c^s, c^p\right) &= \sum_{j=0}^{N_{\text{react}}-1} s_{i,j}^s \varphi^s_j\left(c^s, c^p\right),
\end{align*}
where
\begin{equation*} \begin{split}
  \varphi^s_j(c^s, c^p) &= k^s_{\text{fwd},j} \left[\prod_{m=0}^{\sum_{i=0}^{N_{\text{comp}}-1} N_{\text{bnd},i}-1} \left(c^s_{m}\right)^{e^{s}_{\text{fwd},m,j}}\right] \left[\prod_{\ell=0}^{N_{\text{comp}}-1} \left(c^p_{\ell}\right)^{e^{sp}_{\text{fwd},\ell,j}}\right] \\
  &\phantom{=}\quad - k^p_{\text{bwd},j} \left[\prod_{m=0}^{\sum_{i=0}^{N_{\text{comp}}-1} N_{\text{bnd},i}-1} \left(c^s_{m}\right)^{e^{s}_{\text{bwd},m,j}}\right] \left[\prod_{\ell=0}^{N_{\text{comp}}-1} \left(c^p_{\ell}\right)^{e^{sp}_{\text{bwd},\ell,j}}\right].
\end{split} \end{equation*}
Whereas the exponent matrices $E^{s}_{\text{fwd}}, E^{s}_{\text{bwd}} \in \mathds{R}^{(\sum_i N_{\text{bnd},i}) \times N_{\text{react}}}$ are initialized based on the stoichiometric matrix $S^{s} \in \mathds{R}^{(\sum_i N_{\text{bnd},i}) \times N_{\text{react}}}$, see Eq.~\eqref{eq:MRMassActionLawExpMatDefault}, the exponent matrices $E^{sp}_{\text{fwd}}, E^{sp}_{\text{bwd}}$ of the modifier terms default to $0$.

The layout of the matrices in the file format is presented in Table~\ref{tab:FFReactionMassActionLaw}.
