%!TEX root = ../all.tex
% =============================================================================
%  CADET - The Chromatography Analysis and Design Toolkit
%  
%  Copyright © 2008-2019: The CADET Authors
%            Please see the AUTHORS and CONTRIBUTORS file.
%  
%  All rights reserved. This program and the accompanying materials
%  are made available under the terms of the GNU Public License v3.0 (or, at
%  your option, any later version) which accompanies this distribution, and
%  is available at http://www.gnu.org/licenses/gpl.html
% =============================================================================

\chapter{CADET File Format Specifications}

The CADET framework is designed to work on a file format structured into groups and datasets. This
concept may be implemented by different file formats.
At the moment, CADET natively supports HDF5 and XML as file formats.
The choice is not limited to those two formats but can be extended as needed.
In this section the general layout and structure of the file format is described.

\paragraph{File format versions}
\phantomsection\label{par:FFVersions}

The file format may change and evolve over time as new features are added to the simulator.
This manual describes the most recent file format version that is also set as default value in \texttt{/meta/FILE\_FORMAT} (see Tab.~\ref{tab:FFMeta}).
The simulator assumes that the input file uses the most recent format version and does not update old files to the current standard.

\section{Global structure}

The global structure (see Fig.~\ref{fig:FFRoot}) is divided into three parts: \texttt{input}, \texttt{output}, and \texttt{meta}.
Every valid CADET file needs an \texttt{input} group (see Fig.~\ref{fig:FFInput}) which contains all relevant information for simulating a model.
It does not need an \texttt{output} (see Fig.~\ref{fig:FFOutput}) or \texttt{meta} (see Fig.~\ref{fig:FFRoot}) group, since those are created when results are written.
Whereas the \texttt{output} group is solely used as output and holds the results of the simulation, the \texttt{meta} group is used for input and output.
Details such as file format version and simulator version are read from and written to the \texttt{meta} group.

If not explicitly stated otherwise, all datasets are mandatory. 
By convention all group names are lowercase, whereas all dataset names are uppercase.
Note that this is just a description of the file format and not a detailed explanation of the meaning of the parameters.
For the latter, please refer to the corresponding sections in the previous chapter.

\begin{figure}[!ht]
\centering
\begin{tikzpicture}[%
  every node/.style={draw=black,semithick,font={\footnotesize\ttfamily}},
  level 2/.style={sibling distance=16mm},
  ]
  \node {/ (Root)} [edge from parent fork down]
    child[sibling distance=30mm] { node {\hyperref[fig:FFInput]{input} } }
    child[sibling distance=30mm] { node {\hyperref[fig:FFOutput]{output} } }
    child[sibling distance=30mm] { node {\hyperref[tab:FFMeta]{meta} } };
\end{tikzpicture}
\caption{\label{fig:FFRoot}Structure of the groups in the root group of the file format}
\end{figure}

\begin{figure}[!ht]
\centering
\begin{tikzpicture}[%
  every node/.style={draw=black,semithick,font={\footnotesize\ttfamily}},
  level 2/.style={sibling distance=16mm},
  ]
  \node {\hyperref[sec:FFInput]{input}} [edge from parent fork down]
    child[sibling distance=45mm] { node {\hyperref[tab:FFModelSystem]{model} } [edge from parent fork down]
              child[sibling distance=17mm] { node { \hyperref[tab:FFModelSystemConnections]{connections} } [edge from parent fork down]
                  child { node { \hyperref[tab:FFModelConnectionSwitch]{switch\_000} } }
              }
              child { node { external } [edge from parent fork down]
                  child { node { \hyperref[tab:FFModelExternalSourceLinInterp]{source\_000} } }
              }
              child { node { \hyperref[tab:FFModelSolver]{solver} } }
              child { node { \hyperref[sec:FFModelUnitOp]{unit\_000} } }
          }
    child[sibling distance=40mm] { node { \hyperref[tab:FFSolver]{solver} } [edge from parent fork down]
              child[sibling distance=15mm] { node { \hyperref[tab:FFSolverSections]{sections} } }
              child[sibling distance=25mm] { node { \hyperref[tab:FFSolverTime]{time\_integrator} } }
          }
    child[sibling distance=28mm] { node { \hyperref[tab:FFReturn]{return} } [edge from parent fork down]
              child[sibling distance=25mm] { node { \hyperref[tab:FFReturnUnit]{unit\_000} } }
          }
    child[sibling distance=23mm] { node { \hyperref[tab:FFSensitivity]{sensitivity} } [edge from parent fork down]
              child[sibling distance=20mm] { node { \hyperref[tab:FFSensitivityParam]{param\_000} } }
          };
\end{tikzpicture}
\caption{\label{fig:FFInput}High-level structure of the groups in the input part of the file format}
\end{figure}

\begin{figure}[!ht]
\centering
\begin{tikzpicture}[%
  every node/.style={draw=black,semithick,font={\footnotesize\ttfamily}},
  ]
  \node { \hyperref[sec:FFModelUnitOp]{unit\_000} } [edge from parent fork down]
        child[sibling distance=23mm] { node { \hyperref[sec:FFAdsorption]{adsorption} } [edge from parent fork down]
                child { node { \hyperref[tab:FFModelUnitOpAdsorptionConsSolver]{consistency\_solver} } }
        }
        child[sibling distance=23mm] { node { \hyperref[tab:FFModelUnitOpDiscretizationGRM]{discretization} } [edge from parent fork down]
                child { node { \hyperref[tab:FFModelUnitOpDiscretizationWeno]{weno} } }
        };
\end{tikzpicture}
\caption[Structure of the groups in a column unit operation]{\label{fig:FFModelUnitOpColumn}Structure of the groups in a column unit operation (\texttt{/input/model} group)}
\end{figure}

\begin{figure}[!ht]
\centering
\begin{tikzpicture}[%
  every node/.style={draw=black,semithick,font={\footnotesize\ttfamily}},
  ]
  \node {\hyperref[sec:FFOutput]{output}} [edge from parent fork down]
    child[sibling distance=45mm] { node { \hyperref[tab:FFOutputSolution]{solution} } [edge from parent fork down]
              child[sibling distance=20mm] { node { \hyperref[tab:FFOutputSensitivityUnit]{unit\_000} } }
              child[sibling distance=20mm] { node { \hyperref[tab:FFOutputSensitivityUnit]{unit\_001} } }
    }   
    child[sibling distance=45mm] { node { sensitivity } [edge from parent fork down]
              child[sibling distance=20mm] { node { param\_000 } }
              child[sibling distance=20mm] { node { param\_001 } [edge from parent fork down]
                      child[sibling distance=20mm] { node { \hyperref[tab:FFOutputSensitivityParamUnit]{unit\_000} } }
                      child[sibling distance=20mm] { node { \hyperref[tab:FFOutputSensitivityParamUnit]{unit\_001} } }
              }
          };
\end{tikzpicture}
\caption{\label{fig:FFOutput}Structure of the groups in the output part of the file format}
\end{figure}

\FloatBarrier

\section{Notation and identifiers}

Reference volumes are denoted by subscripts:
\begin{itemize}
  \item[\si{\cubic\metre\of{IV}}] Interstitial volume
  \item[\si{\cubic\metre\of{MP}}] Bead mobile phase volume
  \item[\si{\cubic\metre\of{SP}}] Bead solid phase volume
\end{itemize}

Common notation and identifiers that are used in the subsequent description are listed in Table~\ref{tab:FFNotationIdentifiers}.
\begin{table}[!ht]
\centering
\footnotesize
\begin{tabu}to \linewidth[m]{ll} \toprule
\normalfont Identifier & Meaning \\ \midrule
\texttt{NCOMP} & Number of components of a unit operation \\[0.5ex]
\texttt{NTOTALCOMP} & Total number of components in the system (sum of all unit operation components) \\[0.5ex]
\texttt{NPARTYPE} & Number of particle types of a unit operation \\[0.5ex]
\texttt{NBND}\textsubscript{i} & Number of bound states of component $i$ of a unit operation \\[0.5ex]
\texttt{NTOTALBND} & Total number of bound states of a unit operation (sum of all bound states of all components) \\[0.5ex]
\texttt{NSTATES} & Maximum of the number of bound states for each component of a unit operation \\[0.5ex]
\texttt{NDOF} & Total number of degrees of freedom of the current unit operation model or system of unit operations \\[0.5ex]
\texttt{NSEC} & Number of time integration sections \\[0.5ex]
\texttt{PARAM\_VALUE} & Value of a generic unspecified parameter \\
\bottomrule
\end{tabu}
\caption{\label{tab:FFNotationIdentifiers}Common notation and identifiers used in the file format description}
\end{table}

\FloatBarrier
\section{Ordering of multi dimensional data}

Some model parameters, especially in certain binding models, require multi dimensional data.
Since CADET only reads one dimensional arrays, the layout of the data has to be specified (i.e., the way how the data is linearized in memory).
The term ``\emph{xyz}-major'' means that the index corresponding to \emph{xyz} changes the slowest.

For instance, suppose a model with $2$ components and $3$ bound states has a ``component-major'' dataset.
Then, the requested matrix is stored in memory such that all bound states are listed for each component (i.e., the component index changes the slowest and the bound state index the fastest):
\begin{verbatim}
  comp0bnd0, comp0bnd1, comp0bnd2, comp1bnd0, comp1bnd1, comp1bnd2.
\end{verbatim}
This linear array can be represented as a $2 \times 3$ matrix in ``row-major'' storage format, or a $3 \times 2$ matrix in ``column-major'' ordering.

\section{Section dependent model parameters}

Some model parameters (see Table~\ref{tab:FFSectionDependentParams}) can be assigned different values for each section. For example, the velocity a column is operated with could differ in the load, wash, and elution phases.
Section dependency is recognized by specifying the appropriate number of values for the parameters (see \emph{Length} column in the following tables).
If a parameter depends on both the component and the section, the ordering is section-major.

For instance, the \emph{Length} field of the parameter \texttt{VELOCITY} reads ``1 / \texttt{NSEC}'' which means that it is not recognized as section dependent if only $1$ value (scalar) is passed.
However, if \texttt{NSEC} many values (vector) are present, it will be treated as section dependent.

Note that all components of component dependent datasets have to be section dependent (e.g., you cannot have a section dependency on component $2$ only while the other components are not section dependent).

\begin{table}[!ht]
\centering
\footnotesize
\begin{tabu}to \linewidth[m]{lcc} \toprule
\rowfont[c]\normalfont Dataset & Component dependent & Section dependent \\ \midrule
\texttt{COL\_DISPERSION} & \checkmark & \checkmark \\[0.5ex]
\texttt{FILM\_DIFFUSION} & \checkmark  & \checkmark \\[0.5ex]
\texttt{PAR\_DIFFUSION} & \checkmark  & \checkmark \\[0.5ex]
\texttt{PAR\_SURFDIFFUSION} & \checkmark  & \checkmark \\[0.5ex]
\texttt{VELOCITY} & & \checkmark \\
\bottomrule
\end{tabu}
\caption[Section dependent datasets in the 1D unit operation models]{\label{tab:FFSectionDependentParams}Section dependent datasets in the 1D unit operation models (\texttt{/input/model/unit\_XXX} group)}
\end{table}

\FloatBarrier
\section{Input group}\label{sec:FFInput}
\subsection{System of unit operations}\label{sec:FFModelSystem}

\newenvironment{groupscope}[2]{%
  \paragraph{\fbox{Group \texttt{#1}}}\phantomsection\label{#2}
  \begin{description}
}{%
  \end{description}
}
\newenvironment{condsubgroup}[3]{%
  \paragraph{\fbox{Group \texttt{#1} -- \texttt{#2}}}\phantomsection\label{#3}
  \begin{description}
}{%
  \end{description}
}

\newenvironment{subgroup}[3]{%
  \paragraph{\fbox{Group \texttt{#1} -- #2}}\phantomsection\label{#3}
  \begin{description}
}{%
  \end{description}
}

\pgfkeys{
  /dataset/.is family, /dataset,
  unit/.store in=\datasetUnit,
  type/.store in=\datasetType,
  range/.store in=\datasetRange,
  length/.store in=\datasetLength,
  inout/.store in=\datasetInout,
}
\newenvironment{dataset}[2][]{%
  \pgfkeys{/dataset,#1}\item[\texttt{#2}]
}{%
  \ifthenelse{\NOT\( \isundefined{\datasetUnit} \AND \isundefined{\datasetType} \AND \isundefined{\datasetRange} \AND \isundefined{\datasetLength} \AND \isundefined{\datasetInout}\)}{%
    % At least one option is set
    % Unit & Type & Range & Length
    % Type & Range & Length
    % Unit & Type
    % In/out & Type
    \hfill\vspace{0.25\baselineskip}\linebreak\noindent%\newline
    \ifthenelse{\isundefined{\datasetUnit}}{}{\makebox[0.2\linewidth][l]{\textbf{Unit:} \datasetUnit}}
    \ifthenelse{\isundefined{\datasetInout}}{}{\makebox[0.15\linewidth][l]{\textbf{In/out:} \datasetInout}}
    \ifthenelse{\isundefined{\datasetType}}{}{\makebox[0.15\linewidth][l]{\textbf{Type:} \datasetType}}
    \ifthenelse{\isundefined{\datasetRange}}{}{\makebox[0.35\linewidth][l]{\textbf{Range:} \datasetRange}}
    \ifthenelse{\isundefined{\datasetLength}}{}{\makebox[0.3\linewidth][l]{\textbf{Length:} \datasetLength}}
  }{%
    % No option is set
  }
}

\begin{groupscope}{/input/model}{tab:FFModelSystem}
  \begin{dataset}[type=int,range={$\geq 1$},length=1]{NUNITS}
    Number of unit operations in the system
  \end{dataset}
  \begin{dataset}[type=int,range={$\geq 1$},length=1]{INIT\_STATE\_Y}
    Initial full state vector (optional, unit operation specific initial data is ignored)
  \end{dataset}
  \begin{dataset}[type=double,length={\texttt{NDOF}}]{INIT\_STATE\_YDOT}
    Initial full time derivative state vector (optional, unit operation specific initial data is ignored)
  \end{dataset}
  \begin{dataset}[type=double,length={\texttt{NDOF}}]{INIT\_STATE\_SENSY\_XXX}
    Number of unit operations in the system
  \end{dataset}
  \begin{dataset}[type=double,length={\texttt{NDOF}}]{INIT\_STATE\_SENSYDOT\_XXX}
    Initial full state vector of the \texttt{XXX}th sensitivity system (optional, unit operation specific initial data is ignored)
  \end{dataset}
  \begin{dataset}[type=double,length={\texttt{NDOF}}]{NUNITS}
    Initial full time derivative state vector of the \texttt{XXX}th sensitivity system (optional, unit operation specific initial data is ignored)
  \end{dataset}
\end{groupscope}

\begin{groupscope}{/input/model/connections}{tab:FFModelSystemConnections}
  \begin{dataset}[type=int,range={$\geq 1$},length=1]{NSWITCHES}
    Number of valve switches
  \end{dataset}
\end{groupscope}

\begin{groupscope}{/input/model/connections/switch\_XXX}{tab:FFModelConnectionSwitch}
  \begin{dataset}[type=int,range={$\geq 0$},length=1]{SECTION}
    Index of the section that activates this connection set
  \end{dataset}
  \begin{dataset}[type=double,range={$\geq -1$},length={$7 \cdot \texttt{NCONNECTIONS}$}]{CONNECTIONS}
    Matrix with list of connections in row-major storage.
    Columns are \emph{UnitOpID from}, \emph{UnitOpID to}, \emph{Port from}, \emph{Port to}, \emph{Component from}, \emph{Component to}, \emph{volumetric flow rate}.
    If both port indices are $-1$, all ports are connected.
    If both component indices are $-1$, all components are connected.
  \end{dataset}
\end{groupscope}

\begin{condsubgroup}{/input/model/external/source\_XXX}{EXTFUN\_TYPE = LINEAR\_INTERP\_DATA}{tab:FFModelExternalSourceLinInterp}
  \begin{dataset}[type=double,unit=\si{\per\second},range={$\geq 0$},length=1]{VELOCITY}
    Velocity of the external profile in positive column axial direction
  \end{dataset}
  \begin{dataset}[type=double,unit=\si{\ExternalUnit},range={$\mathds{R}$},length=Arbitrary]{DATA}
    Function values $T$ at the data points
  \end{dataset}
  \begin{dataset}[type=double,unit=\si{\second},range={$\geq 0.0$},length={Same as \texttt{DATA}}]{TIME}
    Time of the data points
  \end{dataset}
\end{condsubgroup}

\begin{condsubgroup}{/input/model/external/source\_XXX}{EXTFUN\_TYPE = PIECEWISE\_CUBIC\_POLY}{tab:FFModelExternalSourcePieceCubicPoly}
  \begin{dataset}[type=double,unit=\si{\per\second},range={$\geq 0$},length=1]{VELOCITY}
    Velocity of the external profile in positive column axial direction
  \end{dataset}
  \begin{dataset}[type=double,unit=\si{\ExternalUnit},range={$\mathds{R}$},length=Arbitrary]{CONST\_COEFF}
    Constant coefficients of piecewise cubic polynomial
  \end{dataset}
  \begin{dataset}[type=double,unit=\si{\ExternalUnit\per\second},range={$\mathds{R}$},length={Same as \texttt{CONST\_COEFF}}]{LIN\_COEFF}
    Linear coefficients of piecewise cubic polynomial
  \end{dataset}
  \begin{dataset}[type=double,unit=\si{\ExternalUnit\per\square\second},range={$\mathds{R}$},length={Same as \texttt{CONST\_COEFF}}]{CONST\_COEFF}
    Quadratic coefficients of piecewise cubic polynomial
  \end{dataset}
  \begin{dataset}[type=double,unit=\si{\ExternalUnit\per\cubic\second},range={$\mathds{R}$},length={Same as \texttt{CONST\_COEFF}}]{QUAD\_COEFF}
    Cubic coefficients of piecewise cubic polynomial
  \end{dataset}
  \begin{dataset}[type=double,unit=\si{\second},range={$\geq 0.0$},length={\texttt{CONST\_COEFF}+1}]{SECTION\_TIMES}
    Simulation times at which a new piece begins (breaks of the piecewise polynomial)
  \end{dataset}
\end{condsubgroup}

\begin{groupscope}{/input/model/solver}{tab:FFModelSolver}
  \begin{dataset}[type=int,range={$\{0, 1\}$},length=1]{GS\_TYPE}
    Type of Gram-Schmidt orthogonalization, see IDAS guide Section~4.5.7.3, p.~41f.
    A value of $0$ enables classical Gram-Schmidt, a value of 1 uses modified Gram-Schmidt.
  \end{dataset}
  \begin{dataset}[type=int,range={$\{0, \dots, \texttt{NDOF}\}$},length=1]{MAX\_KRYLOV}
    Defines the size of the Krylov subspace in the iterative linear GMRES solver (0: \texttt{MAX\_KRYLOV} = \texttt{NDOF})
  \end{dataset}
  \begin{dataset}[type=int,range={$\geq 0$},length=1]{MAX\_RESTARTS}
    Maximum number of restarts in the GMRES algorithm. If lack of memory is not an issue, better use a larger Krylov space than restarts.
  \end{dataset}
  \begin{dataset}[type=double,range={$\geq 0$},length=1]{SCHUR\_SAFETY}
    Schur safety factor; Influences the tradeoff between linear iterations and nonlinear error control; see IDAS guide Section~2.1 and 5.
  \end{dataset}
\end{groupscope}

\subsection{Unit operation models}\label{sec:FFModelUnitOp}

\subsubsection{Inlet}

\begin{condsubgroup}{/input/model/unit\_XXX}{UNIT\_TYPE = INLET}{tab:FFModelUnitOpInlet}
  \begin{dataset}[type=string,range={\texttt{INLET}},length=1]{UNIT\_TYPE}
    Specifies the type of unit operation model
  \end{dataset}
  \begin{dataset}[type=int,range={$\geq 1$},length=1]{NCOMP}
    Number of chemical components in the chromatographic media
  \end{dataset}
  \begin{dataset}[type=string,range={\texttt{PIECEWISE\_CUBIC\_POLY}},length=1]{INLET\_TYPE}
    Specifies the type of inlet profile
  \end{dataset}
\end{condsubgroup}

\begin{groupscope}{/input/model/unit\_XXX/sec\_XXX}{tab:FFModelInletPiecewiseCubicPoly}
  \begin{dataset}[type=double,unit=\si{\mol\per\cubic\metre\of{IV}},range={$\mathds{R}$},length={\texttt{NCOMP}}]{CONST\_COEFF}
    Constant coefficients for inlet concentrations
  \end{dataset}
  \begin{dataset}[type=double,unit=\si{\mol\per\cubic\metre\of{IV}\per\second},range={$\mathds{R}$},length={\texttt{NCOMP}}]{LIN\_COEFF}
    Linear coefficients for inlet concentrations
  \end{dataset}
  \begin{dataset}[type=double,unit=\si{\mol\per\cubic\metre\of{IV}\per\square\second},range={$\mathds{R}$},length={\texttt{NCOMP}}]{QUAD\_COEFF}
    Quadratic coefficients for inlet concentrations
  \end{dataset}
  \begin{dataset}[type=double,unit=\si{\mol\per\cubic\metre\of{IV}\per\cubic\second},range={$\mathds{R}$},length={\texttt{NCOMP}}]{CUBE\_COEFF}
    Cubic coefficients for inlet concentrations
  \end{dataset}
\end{groupscope}

\subsubsection{Outlet}

\begin{condsubgroup}{/input/model/unit\_XXX}{UNIT\_TYPE = OUTLET}{tab:FFModelUnitOpOutlet}
  \begin{dataset}[type=string,range={\texttt{OUTLET}},length=1]{UNIT\_TYPE}
    Specifies the type of unit operation model
  \end{dataset}
  \begin{dataset}[type=int,range={$\geq 1$},length=1]{NCOMP}
    Number of chemical components in the chromatographic media
  \end{dataset}
\end{condsubgroup}

\subsubsection{General rate model}

\begin{condsubgroup}{/input/model/unit\_XXX}{UNIT\_TYPE = GENERAL\_RATE\_MODEL}{tab:FFModelUnitOpGRM}
  \begin{dataset}[type=string,range={\texttt{GENERAL\_RATE\_MODEL}},length=1]{UNIT\_TYPE}
    Specifies the type of unit operation model
  \end{dataset}
  \begin{dataset}[type=int,range={$\geq 1$},length=1]{NCOMP}
    Number of chemical components in the chromatographic media
  \end{dataset}
  \begin{dataset}[type=string,range={See Section~\ref{sec:FFAdsorption}},length={\texttt{NPARTYPE}}]{ADSORPTION\_MODEL}
    Specifies the type of adsorption models of each particle type
  \end{dataset}
  \begin{dataset}[unit=\si{\mol\per\cubic\metre\of{IV}},type=double,range={$\geq 0$},length={\texttt{NCOMP}}]{INIT\_C}
    Initial concentrations for each component in the bulk mobile phase
  \end{dataset}
  \begin{dataset}[unit=\si{\mol\per\cubic\metre\of{MP}},type=double,range={$\geq 0$},length={$\texttt{NPARTYPE} \cdot \texttt{NCOMP}$}]{INIT\_CP}
    Initial concentrations for each component in the bead liquid phase of each particle type in type-major ordering (optional, \texttt{INIT\_C} is used if left out)
  \end{dataset}
  \begin{dataset}[unit=\si{\mol\per\cubic\metre\of{SP}},type=double,range={$\geq 0$},length={\texttt{NTOTALBND}}]{INIT\_Q}
    Initial concentrations for each bound state of each component in the bead solid phase of each particle type in type-component-major ordering
  \end{dataset}
  \begin{dataset}[unit=various,type=double,range={$\mathds{R}$},length={\texttt{NDOF} / $2\texttt{NDOF}$}]{INIT\_STATE}
    Full state vector for initialization (optional, \texttt{INIT\_C}, \texttt{INIT\_CP}, and \texttt{INIT\_Q} will be ignored; if length is $2\texttt{NDOF}$, then the second half is used for time derivatives)
  \end{dataset}
  \begin{dataset}[unit=\si{\square\metre\of{IV}\per\second},type=double,range={$\geq 0$},length={see \texttt{{COL\_DISPERSION\_MULTIPLEX}}}]{COL\_DISPERSION}
    Axial dispersion coefficient
  \end{dataset}
  \begin{dataset}[unit=--,type=int,range={$\{0, \dots, 3 \}$},length={1}]{COL\_DISPERSION\_MULTIPLEX}
    Multiplexing mode of \texttt{COL\_DISPERSION}.
    Determines whether \texttt{COL\_DISPERSION} is treated as component- and/or section-independent.

    This field is optional.
    When left out, multiplexing behavior is inferred from the length of \texttt{COL\_DISPERSION}.

    Valid modes are:
    \begin{description}
      \item[0] Component-independent, section-independent; length of \texttt{COL\_DISPERSION} is $1$
      \item[1] Component-dependent, section-independent; length of \texttt{COL\_DISPERSION} is $\texttt{NCOMP}$
      \item[2] Component-independent, section-dependent; length of \texttt{COL\_DISPERSION} is $\texttt{NSEC}$
      \item[3] Component-dependent, section-dependent; length of \texttt{COL\_DISPERSION} is $\texttt{NCOMP} \cdot \texttt{NSEC}$; ordering is section-major
    \end{description}\vspace{-\baselineskip}
  \end{dataset}
  \begin{dataset}[unit=\si{\metre},type=double,range={$> 0$},length={1}]{COL\_LENGTH}
    Column length
  \end{dataset}
  \begin{dataset}[unit=--,type=double,range={$[0,1]$},length={1}]{COL\_POROSITY}
    Column porosity
  \end{dataset}
  \begin{dataset}[unit=\si{\metre\per\second},type=double,range={$\geq 0$},length={$\texttt{NPARTYPE} \cdot \texttt{NCOMP}$ / $\texttt{NPARTYPE} \cdot \texttt{NCOMP} \cdot \texttt{NSEC}$}]{FILM\_DIFFUSION}
    Film diffusion coefficients for each component of each particle type in section-type-major ordering
  \end{dataset}
  \begin{dataset}[unit=--,type=double,range={$[0,1]$},length={\texttt{NPARTYPE}}]{PAR\_POROSITY}
    Particle porosity of each particle type
  \end{dataset}
  \begin{dataset}[unit=\si{\metre},type=double,range={$>0$},length={\texttt{NPARTYPE}}]{PAR\_RADIUS}
    Particle radius of each particle type
  \end{dataset}
  \begin{dataset}[unit=\si{\metre},type=double,range={$[0, \texttt{PAR\_RADIUS})$},length={\texttt{NPARTYPE}}]{PAR\_CORERADIUS}
    Particle core radius of each particle type (optional, defaults to \SI{0}{\metre})
  \end{dataset}
  \begin{dataset}[unit=--,type=double,range={$(0, 1]$},length={$\texttt{NPARTYPE} \cdot \texttt{NCOMP}$}]{PORE\_ACCESSIBILITY}
    Pore accessibility factor of each component in each particle type in type-major ordering (optional, defaults to $1$)
  \end{dataset}
  \begin{dataset}[unit=\si{\square\metre\of{MP}\per\second},type=double,range={$\geq 0$},length={$\texttt{NPARTYPE} \cdot \texttt{NCOMP}$ / $\texttt{NSEC} \cdot \texttt{NPARTYPE} \cdot \texttt{NCOMP}$}]{PAR\_DIFFUSION}
    Effective particle diffusion coefficients of each component in each particle type in section-type-major ordering
  \end{dataset}
  \begin{dataset}[unit=\si{\square\metre\of{SP}\per\second},type=double,range={$\geq 0$},length={$\texttt{NTOTALBND}$ / $\texttt{NSEC} \cdot \texttt{NTOTALBND}$}]{PAR\_SURFDIFFUSION}
    Particle surface diffusion coefficients of each bound state of each component in each particle type in section-type-component-major ordering
  \end{dataset}
  \begin{dataset}[unit=\si{\metre\per\second},type=double,range={$\mathds{R}$},length={1 / \texttt{NSEC}}]{VELOCITY}
    Interstitial velocity of the mobile phase (optional if \texttt{CROSS\_SECTION\_AREA} is present, see Section~\ref{par:MUOPGRMflow})
  \end{dataset}
  \begin{dataset}[unit=\si{\square\metre},type=double,range={$>0$},length={1}]{CROSS\_SECTION\_AREA}
    Cross section area of the column (optional if \texttt{VELOCITY} is present, see Section~\ref{par:MUOPGRMflow})
  \end{dataset}
  \begin{dataset}[unit=--,type=double,range={$[0,1]$},length={\texttt{NPARTYPE} / $\texttt{NCOL} \cdot \texttt{NPARTYPE}$}]{PAR\_TYPE\_VOLFRAC}
    Volume fractions of the particle types.
    The volume fractions can be set for all axial cells together or for each individual axial cell.
    In case of a spatially inhomogeneous setting, the data is expected in cell-major ordering and the \texttt{SENS\_SECTION} field is used for indexing the axial cell when specifying parameter sensitivities.
  \end{dataset}
\end{condsubgroup}

\begin{condsubgroup}{/input/model/unit\_XXX/discretization}{UNIT\_TYPE = GENERAL\_RATE\_MODEL}{tab:FFModelUnitOpDiscretizationGRM}
  \begin{dataset}[type=int,range={$\geq 1$},length=1]{NCOL}
    Number of axial column discretization cells
  \end{dataset}
  \begin{dataset}[type=int,range={$\geq 1$},length={\texttt{NPARTYPE}}]{NPAR}
    Number of particle (radial) discretization cells for each particle type
  \end{dataset}
  \begin{dataset}[type=int,range={$\geq 0$},length={$\texttt{NPARTYPE} \cdot \texttt{NCOMP}$}]{NBOUND}
    Number of bound states for each component in each particle type in type-major ordering
  \end{dataset}
  \begin{dataset}[type=string,length={\texttt{NPARTYPE}}]{PAR\_DISC\_TYPE}
    Specifies the discretization scheme inside the particles for each particle type.
    Valid values are \texttt{EQUIDISTANT\_PAR}, \texttt{EQUIVOLUME\_PAR}, and \texttt{USER\_DEFINED\_PAR}.
  \end{dataset}
  \begin{dataset}[unit=--,type=double,range={$[0,1]$},length={$\sum_i (\texttt{NPAR}_i + 1)$}]{PAR\_DISC\_VECTOR}
    Node coordinates for the cell boundaries (ignored if $\texttt{PAR\_DISC\_TYPE} \neq \texttt{USER\_DEFINED\_PAR}$).
    The coordinates are relative and have to include the endpoints $0$ and $1$.
    They are later linearly mapped to the true radial range $[r_{c,j}, r_{p,j}]$.
    The coordinates for each particle type are appended to one long vector in type-major ordering.
  \end{dataset}
  \begin{dataset}[type=int,range={$\{0, 1\}$},length=1]{USE\_ANALYTIC\_JACOBIAN}
    Determines whether analytically computed Jacobian matrix (faster) is used (value is $1$) instead of Jacobians generated by algorithmic differentiation (slower, value is $0$)
  \end{dataset}
  \begin{dataset}[type=string,range={\texttt{WENO}},length={1}]{RECONSTRUCTION}
    Type of reconstruction method for fluxes
  \end{dataset}
  \begin{dataset}[type=int,range={$\{0, 1\}$},length=1]{GS\_TYPE}
    Type of Gram-Schmidt orthogonalization, see IDAS guide Section~4.5.7.3, p.~41f.
    A value of $0$ enables classical Gram-Schmidt, a value of 1 uses modified Gram-Schmidt.
  \end{dataset}
  \begin{dataset}[type=int,range={$\{0, \dots, \texttt{NCOL} \cdot \texttt{NCOMP} \cdot \texttt{NPARTYPE} \}$},length=1]{MAX\_KRYLOV}
    Defines the size of the Krylov subspace in the iterative linear GMRES solver (0: $\texttt{MAX\_KRYLOV} = \texttt{NCOL} \cdot \texttt{NCOMP} \cdot \texttt{NPARTYPE}$)
  \end{dataset}
  \begin{dataset}[type=int,range={$\geq 0$},length=1]{MAX\_RESTARTS}
    Maximum number of restarts in the GMRES algorithm. If lack of memory is not an issue, better use a larger Krylov space than restarts.
  \end{dataset}
  \begin{dataset}[type=double,range={$\geq 0$},length=1]{SCHUR\_SAFETY}
    Schur safety factor; Influences the tradeoff between linear iterations and nonlinear error control; see IDAS guide Section~2.1 and 5.
  \end{dataset}
\end{condsubgroup}

\subsubsection{Lumped rate model with pores}

\begin{condsubgroup}{/input/model/unit\_XXX}{UNIT\_TYPE = LUMPED\_RATE\_MODEL\_WITH\_PORES}{tab:FFModelUnitOpLRMP}
  \begin{dataset}[type=string,range={\texttt{LUMPED\_RATE\_MODEL\_WITH\_PORES}},length=1]{UNIT\_TYPE}
    Specifies the type of unit operation model
  \end{dataset}
  \begin{dataset}[type=int,range={$\geq 1$},length=1]{NCOMP}
    Number of chemical components in the chromatographic media
  \end{dataset}
  \begin{dataset}[type=string,range={See Section~\ref{sec:FFAdsorption}},length={\texttt{NPARTYPE}}]{ADSORPTION\_MODEL}
    Specifies the type of adsorption models of each particle type
  \end{dataset}
  \begin{dataset}[unit=\si{\mol\per\cubic\metre\of{IV}},type=double,range={$\geq 0$},length={\texttt{NCOMP}}]{INIT\_C}
    Initial concentrations for each component in the bulk mobile phase
  \end{dataset}
  \begin{dataset}[unit=\si{\mol\per\cubic\metre\of{MP}},type=double,range={$\geq 0$},length={$\texttt{NPARTYPE} \cdot \texttt{NCOMP}$}]{INIT\_CP}
    Initial concentrations for each component in the bead liquid phase of each particle type in type-major ordering (optional, \texttt{INIT\_C} is used if left out)
  \end{dataset}
  \begin{dataset}[unit=\si{\mol\per\cubic\metre\of{SP}},type=double,range={$\geq 0$},length={\texttt{NTOTALBND}}]{INIT\_Q}
    Initial concentrations for each bound state of each component in the bead solid phase of each particle type in type-component-major ordering
  \end{dataset}
  \begin{dataset}[unit=various,type=double,range={$\mathds{R}$},length={\texttt{NDOF} / $2\texttt{NDOF}$}]{INIT\_STATE}
    Full state vector for initialization (optional, \texttt{INIT\_C}, \texttt{INIT\_CP}, and \texttt{INIT\_Q} will be ignored; if length is $2\texttt{NDOF}$, then the second half is used for time derivatives)
  \end{dataset}
  \begin{dataset}[unit=\si{\square\metre\of{IV}\per\second},type=double,range={$\geq 0$},length={see \texttt{{COL\_DISPERSION\_MULTIPLEX}}}]{COL\_DISPERSION}
    Axial dispersion coefficient
  \end{dataset}
  \begin{dataset}[unit=--,type=int,range={$\{0, \dots, 3 \}$},length={1}]{COL\_DISPERSION\_MULTIPLEX}
    Multiplexing mode of \texttt{COL\_DISPERSION}.
    Determines whether \texttt{COL\_DISPERSION} is treated as component- and/or section-independent.

    This field is optional.
    When left out, multiplexing behavior is inferred from the length of \texttt{COL\_DISPERSION}.

    Valid modes are:
    \begin{description}
      \item[0] Component-independent, section-independent; length of \texttt{COL\_DISPERSION} is $1$
      \item[1] Component-dependent, section-independent; length of \texttt{COL\_DISPERSION} is $\texttt{NCOMP}$
      \item[2] Component-independent, section-dependent; length of \texttt{COL\_DISPERSION} is $\texttt{NSEC}$
      \item[3] Component-dependent, section-dependent; length of \texttt{COL\_DISPERSION} is $\texttt{NCOMP} \cdot \texttt{NSEC}$; ordering is section-major
    \end{description}\vspace{-\baselineskip}
  \end{dataset}
  \begin{dataset}[unit=\si{\metre},type=double,range={$> 0$},length={1}]{COL\_LENGTH}
    Column length
  \end{dataset}
  \begin{dataset}[unit=--,type=double,range={$[0,1]$},length={1}]{COL\_POROSITY}
    Column porosity
  \end{dataset}
  \begin{dataset}[unit=\si{\metre\per\second},type=double,range={$\geq 0$},length={$\texttt{NPARTYPE} \cdot \texttt{NCOMP}$ / $\texttt{NPARTYPE} \cdot \texttt{NCOMP} \cdot \texttt{NSEC}$}]{FILM\_DIFFUSION}
    Film diffusion coefficients for each component of each particle type in section-type-major ordering
  \end{dataset}
  \begin{dataset}[unit=--,type=double,range={$[0,1]$},length={\texttt{NPARTYPE}}]{PAR\_POROSITY}
    Particle porosity of each particle type
  \end{dataset}
  \begin{dataset}[unit=\si{\metre},type=double,range={$>0$},length={\texttt{NPARTYPE}}]{PAR\_RADIUS}
    Particle radius of each particle type
  \end{dataset}
  \begin{dataset}[unit=--,type=double,range={$(0, 1]$},length={$\texttt{NPARTYPE} \cdot \texttt{NCOMP}$}]{PORE\_ACCESSIBILITY}
    Pore accessibility factor of each component in each particle type in type-major ordering (optional, defaults to $1$)
  \end{dataset}
  \begin{dataset}[unit=\si{\metre\per\second},type=double,range={$\mathds{R}$},length={1 / \texttt{NSEC}}]{VELOCITY}
    Interstitial velocity of the mobile phase (optional if \texttt{CROSS\_SECTION\_AREA} is present, see Section~\ref{par:MUOPGRMflow})
  \end{dataset}
  \begin{dataset}[unit=\si{\square\metre},type=double,range={$>0$},length={1}]{CROSS\_SECTION\_AREA}
    Cross section area of the column (optional if \texttt{VELOCITY} is present, see Section~\ref{par:MUOPGRMflow})
  \end{dataset}
  \begin{dataset}[unit=--,type=double,range={$[0,1]$},length={\texttt{NPARTYPE} / $\texttt{NCOL} \cdot \texttt{NPARTYPE}$}]{PAR\_TYPE\_VOLFRAC}
    Volume fractions of the particle types.
    The volume fractions can be set for all axial cells together or for each individual axial cell.
    In case of a spatially inhomogeneous setting, the data is expected in cell-major ordering and the \texttt{SENS\_SECTION} field is used for indexing the axial cell when specifying parameter sensitivities.
  \end{dataset}
\end{condsubgroup}

\begin{condsubgroup}{/input/model/unit\_XXX/discretization}{UNIT\_TYPE = LUMPED\_RATE\_MODEL\_WITH\_PORES}{tab:FFModelUnitOpDiscretizationLRMP}
  \begin{dataset}[type=int,range={$\geq 1$},length=1]{NCOL}
    Number of axial column discretization cells
  \end{dataset}
  \begin{dataset}[type=int,range={$\geq 0$},length={$\texttt{NPARTYPE} \cdot \texttt{NCOMP}$}]{NBOUND}
    Number of bound states for each component in each particle type in type-major ordering
  \end{dataset}
  \begin{dataset}[type=int,range={$\{0, 1\}$},length=1]{USE\_ANALYTIC\_JACOBIAN}
    Determines whether analytically computed Jacobian matrix (faster) is used (value is $1$) instead of Jacobians generated by algorithmic differentiation (slower, value is $0$)
  \end{dataset}
  \begin{dataset}[type=string,range={\texttt{WENO}},length={1}]{RECONSTRUCTION}
    Type of reconstruction method for fluxes
  \end{dataset}
  \begin{dataset}[type=int,range={$\{0, 1\}$},length=1]{GS\_TYPE}
    Type of Gram-Schmidt orthogonalization, see IDAS guide Section~4.5.7.3, p.~41f.
    A value of $0$ enables classical Gram-Schmidt, a value of 1 uses modified Gram-Schmidt.
  \end{dataset}
  \begin{dataset}[type=int,range={$\{0, \dots, \texttt{NCOL} \cdot \texttt{NCOMP} \cdot \texttt{NPARTYPE} \}$},length=1]{MAX\_KRYLOV}
    Defines the size of the Krylov subspace in the iterative linear GMRES solver (0: $\texttt{MAX\_KRYLOV} = \texttt{NCOL} \cdot \texttt{NCOMP} \cdot \texttt{NPARTYPE}$)
  \end{dataset}
  \begin{dataset}[type=int,range={$\geq 0$},length=1]{MAX\_RESTARTS}
    Maximum number of restarts in the GMRES algorithm. If lack of memory is not an issue, better use a larger Krylov space than restarts.
  \end{dataset}
  \begin{dataset}[type=double,range={$\geq 0$},length=1]{SCHUR\_SAFETY}
    Schur safety factor; Influences the tradeoff between linear iterations and nonlinear error control; see IDAS guide Section~2.1 and 5.
  \end{dataset}
\end{condsubgroup}

\subsubsection{Lumped rate model without pores}

\begin{condsubgroup}{/input/model/unit\_XXX}{UNIT\_TYPE = LUMPED\_RATE\_MODEL\_WITHOUT\_PORES}{tab:FFModelUnitOpLRM}
  \begin{dataset}[type=string,range={\texttt{LUMPED\_RATE\_MODEL\_WITHOUT\_PORES}},length=1]{UNIT\_TYPE}
    Specifies the type of unit operation model
  \end{dataset}
  \begin{dataset}[type=int,range={$\geq 1$},length=1]{NCOMP}
    Number of chemical components in the chromatographic media
  \end{dataset}
  \begin{dataset}[type=string,range={See Section~\ref{sec:FFAdsorption}},length=1]{ADSORPTION\_MODEL}
    Specifies the type of adsorption model
  \end{dataset}
  \begin{dataset}[unit=\si{\mol\per\cubic\metre\of{IV}},type=double,range={$\geq 0$},length={\texttt{NCOMP}}]{INIT\_C}
    Initial concentrations for each component in the bulk mobile phase
  \end{dataset}
  \begin{dataset}[unit=\si{\mol\per\cubic\metre\of{SP}},type=double,range={$\geq 0$},length={\texttt{NTOTALBND}}]{INIT\_Q}
    Initial concentrations for each bound state of each component in the bead solid phase in component-major ordering
  \end{dataset}
  \begin{dataset}[unit=various,type=double,range={$\mathds{R}$},length={\texttt{NDOF} / $2\texttt{NDOF}$}]{INIT\_STATE}
    Full state vector for initialization (optional, \texttt{INIT\_C} and \texttt{INIT\_Q} will be ignored; if length is $2\texttt{NDOF}$, then the second half is used for time derivatives)
  \end{dataset}
  \begin{dataset}[unit=\si{\square\metre\of{IV}\per\second},type=double,range={$\geq 0$},length={see \texttt{{COL\_DISPERSION\_MULTIPLEX}}}]{COL\_DISPERSION}
    Axial dispersion coefficient
  \end{dataset}
  \begin{dataset}[unit=--,type=int,range={$\{0, \dots, 3 \}$},length={1}]{COL\_DISPERSION\_MULTIPLEX}
    Multiplexing mode of \texttt{COL\_DISPERSION}.
    Determines whether \texttt{COL\_DISPERSION} is treated as component- and/or section-independent.

    This field is optional.
    When left out, multiplexing behavior is inferred from the length of \texttt{COL\_DISPERSION}.

    Valid modes are:
    \begin{description}
      \item[0] Component-independent, section-independent; length of \texttt{COL\_DISPERSION} is $1$
      \item[1] Component-dependent, section-independent; length of \texttt{COL\_DISPERSION} is $\texttt{NCOMP}$
      \item[2] Component-independent, section-dependent; length of \texttt{COL\_DISPERSION} is $\texttt{NSEC}$
      \item[3] Component-dependent, section-dependent; length of \texttt{COL\_DISPERSION} is $\texttt{NCOMP} \cdot \texttt{NSEC}$; ordering is section-major
    \end{description}\vspace{-\baselineskip}
  \end{dataset}
  \begin{dataset}[unit=\si{\metre},type=double,range={$> 0$},length={1}]{COL\_LENGTH}
    Column length
  \end{dataset}
  \begin{dataset}[unit=--,type=double,range={$[0,1]$},length={1}]{TOTAL\_POROSITY}
    Total porosity
  \end{dataset}
  \begin{dataset}[unit=\si{\metre\per\second},type=double,range={$\mathds{R}$},length={1 / \texttt{NSEC}}]{VELOCITY}
    Interstitial velocity of the mobile phase (optional if \texttt{CROSS\_SECTION\_AREA} is present, see Section~\ref{par:MUOPGRMflow})
  \end{dataset}
  \begin{dataset}[unit=\si{\square\metre},type=double,range={$>0$},length={1}]{CROSS\_SECTION\_AREA}
    Cross section area of the column (optional if \texttt{VELOCITY} is present, see Section~\ref{par:MUOPGRMflow})
  \end{dataset}
\end{condsubgroup}

\begin{condsubgroup}{/input/model/unit\_XXX/discretization}{UNIT\_TYPE = LUMPED\_RATE\_MODEL\_WITHOUT\_PORES}{tab:FFModelUnitOpDiscretizationLRM}
  \begin{dataset}[type=int,range={$\geq 1$},length=1]{NCOL}
    Number of axial column discretization cells
  \end{dataset}
  \begin{dataset}[type=int,range={$\geq 0$},length={\texttt{NCOMP}}]{NBOUND}
    Number of bound states for each component
  \end{dataset}
  \begin{dataset}[type=int,range={$\{0, 1\}$},length=1]{USE\_ANALYTIC\_JACOBIAN}
    Determines whether analytically computed Jacobian matrix (faster) is used (value is $1$) instead of Jacobians generated by algorithmic differentiation (slower, value is $0$)
  \end{dataset}
  \begin{dataset}[type=string,range={\texttt{WENO}},length={1}]{RECONSTRUCTION}
    Type of reconstruction method for fluxes
  \end{dataset}
\end{condsubgroup}

\subsubsection{Continuous stirred tank reactor model}

\begin{condsubgroup}{/input/model/unit\_XXX}{UNIT\_TYPE = CSTR}{tab:FFModelUnitOpCSTR}
  \begin{dataset}[type=string,range={\texttt{CSTR}},length=1]{UNIT\_TYPE}
    Specifies the type of unit operation model
  \end{dataset}
  \begin{dataset}[type=int,range={$\geq 1$},length=1]{NCOMP}
    Number of chemical components in the chromatographic media
  \end{dataset}
  \begin{dataset}[type=int,range={$\geq 0$},length={$\texttt{NPARTYPE} \cdot \texttt{NCOMP}$}]{NBOUND}
    Number of bound states for each component in each particle type in type-major ordering (optional, defaults to all $0$)
  \end{dataset}
  \begin{dataset}[type=int,range={$\{0, 1\}$},length=1]{USE\_ANALYTIC\_JACOBIAN}
    Determines whether analytically computed Jacobian matrix (faster) is used (value is $1$) instead of Jacobians generated by algorithmic differentiation (slower, value is $0$)
  \end{dataset}
  \begin{dataset}[type=string,range={See Section~\ref{sec:FFAdsorption}},length={\texttt{NPARTYPE}}]{ADSORPTION\_MODEL}
    Specifies the type of adsorption models of each particle type (optional, defaults to \texttt{NONE})
  \end{dataset}
  \begin{dataset}[unit=\si{\mol\per\cubic\metre\of{IV}},type=double,range={$\geq 0$},length={\texttt{NCOMP}}]{INIT\_C}
    Initial concentrations for each component in the mobile phase
  \end{dataset}
  \begin{dataset}[unit=\si{\cubic\metre},type=double,range={$\geq 0$},length=1]{INIT\_VOLUME}
    Initial tank volume
  \end{dataset}
  \begin{dataset}[unit=\si{\mol\per\cubic\metre\of{SP}},type=double,range={$\geq 0$},length={\texttt{NTOTALBND}}]{INIT\_Q}
    Initial concentrations for each bound state of each component in each particle type's solid phase in type-component-major ordering (optional, defaults to all $0$)
  \end{dataset}
  \begin{dataset}[unit=various,type=double,range={$\mathds{R}$},length={\texttt{NDOF} / $2\texttt{NDOF}$}]{INIT\_STATE}
    Full state vector for initialization (optional, \texttt{INIT\_C}, \texttt{INIT\_Q}, and \texttt{INIT\_VOLUME} will be ignored; if length is $2\texttt{NDOF}$, then the second half is used for time derivatives)
  \end{dataset}
  \begin{dataset}[unit=--,type=double,range={$[0,1]$},length={1}]{POROSITY}
    Porosity $\varepsilon$ (defaults to $1$)
  \end{dataset}
  \begin{dataset}[unit=\si{\cubic\metre\per\second},type=double,range={$\geq 0$},length={1 / \texttt{NSEC}}]{FLOWRATE\_FILTER}
    Flow rate of pure liquid without components (optional, defaults to \SI{0}{\cubic\metre\per\second})
  \end{dataset}
  \begin{dataset}[unit=--,type=double,range={$[0,1]$},length={\texttt{NPARTYPE}}]{PAR\_TYPE\_VOLFRAC}
    Volume fractions of the particle types
  \end{dataset}
\end{condsubgroup}

\subsection{Flux reconstruction methods}

\begin{subgroup}{/input/model/unit\_XXX}{WENO parameters}{tab:FFModelUnitOpDiscretizationWeno}
  \begin{dataset}[type=int,range={$\{ 0,1,2,3 \}$},length=1]{BOUNDARY\_MODEL}
    Boundary model type:
    \begin{description}
      \item[0] Lower WENO order (stable)
      \item[1] Zero weights (unstable for small $D_{\mathrm{ax}}$)
      \item[2] Zero weights for $p \neq 0$ (stable?)
      \item[3] Large ghost points
    \end{description}\vspace{-\baselineskip}
  \end{dataset}
  \begin{dataset}[type=double,range={$\geq 0$},length=1]{WENO\_EPS}
    WENO $\varepsilon$
  \end{dataset}
  \begin{dataset}[type=int,range={$\{ 1,2,3 \}$},length=1]{WENO\_ORDER}
    WENO order, also called WENO $k$:
    \begin{description}
      \item[1] Standard upwind scheme (order 1)
      \item[1] WENO 2 (order 3)
      \item[2] WENO 3 (order 5)
    \end{description}\vspace{-\baselineskip}
  \end{dataset}
\end{subgroup}

\subsection{Adsorption models}\label{sec:FFAdsorption}

\paragraph{Externally dependent binding models}

Some binding models have a variant that can use external sources\index{Binding!External function} as specified in Section~\ref{sec:FFModelSystem} (also see Section~\ref{par:MBFeatureMatrix} and Table~\ref{tab:MBFeatureMatrix} on which binding models support this feature).
For the sake of brevity, only the standard variant of those binding models is specified below.
In order to obtain the format for the externally dependent variant, first replace the binding model name \texttt{XXX} by \texttt{EXT\_XXX}.
Each parameter $p$ (except for reference concentrations \texttt{XXX\_REFC0} and \texttt{XXX\_REFQ}) depends on a (possibly distinct) external source in a polynomial way:
\begin{align*}
  p(T) &= p_{\texttt{TTT}} T^3 + p_{\texttt{TT}} T^2 + p_{\texttt{T}} T + p.
\end{align*}
Thus, a parameter \texttt{XXX\_YYY} of the standard binding model variant is replaced by the four parameters \texttt{EXT\_XXX\_YYY}, \texttt{EXT\_XXX\_YYY\_T}, \texttt{EXT\_XXX\_YYY\_TT}, and \texttt{EXT\_XXX\_YYY\_TTT}.
Since each parameter can depend on a different external source, the dataset \texttt{EXTFUN} (not listed in the standard variants below) should contain a vector of 0-based integer indices of the external source of each parameter.
The ordering of the parameters in \texttt{EXTFUN} is given by the ordering in the standard variant.
However, if only one index is passed in \texttt{EXTFUN}, this external source is used for all parameters.

Note that parameter sensitivities with respect to column radius, column length, particle core radius, and particle radius may be wrong when using externally dependent binding models.
This is caused by not taking into account the derivative of the external profile with respect to column position.

\paragraph{Non-binding components}

For binding models that do not support multiple bound states, many parameters can vary per component and their length is taken as \texttt{NCOMP}.
However, these models still support non-binding components.\index{Binding!Non-binding components}
In this case, the entries in their parameters that correspond to non-binding components are simply ignored.

\paragraph{Multiple particle types}\index{Model!Multiple particle types}\index{Model!Particle types}

The group that contains the parameters of a binding model in unit operation with index \texttt{XXX} reads \texttt{/input/model/unit\_XXX/adsorption}.
This is valid for models with a single particle type.
If a model has multiple particle types, it may have a different binding model in each type.
The parameters are then placed in the group \texttt{/input/model/unit\_XXX/adsorption\_YYY} instead, where \texttt{YYY} denotes the index of the particle type.

Note that, in any case, \texttt{/input/model/unit\_XXX/adsorption\_000} contains the parameters of the first (and possibly sole) particle type.
This group also takes precedence over a possibly existing \texttt{/input/model/unit\_XXX/adsorption} group.

\begin{subgroup}{/input/model/unit\_XXX/adsorption/consistency\_solver}{Nonlinear consistency solver parameters}{tab:FFModelUnitOpAdsorptionConsSolver}
  \begin{dataset}[type=string,length=1]{SOLVER\_NAME}
    Name of the solver.
    Available solvers are \texttt{LEVMAR}, \texttt{ATRN\_RES}, \texttt{ATRN\_ERR}, and \texttt{COMPOSITE}.
  \end{dataset}
  \begin{dataset}[type=double,range={$\geq 0$},length=1]{INIT\_DAMPING}
    Initial damping factor (default is $0.01$)
  \end{dataset}
  \begin{dataset}[type=double,range={$\geq 0$},length=1]{MIN\_DAMPING}
    Minimal damping factor (default is $0.0001$; ignored by \texttt{LEVMAR})
  \end{dataset}
  \begin{dataset}[type=string,length={$> 1$}]{SUBSOLVERS}
    Vector with names of solvers for the composite solver (only required for composite solver).
    See \texttt{SOLVER\_NAME} for available solvers.
  \end{dataset}
\end{subgroup}

\begin{condsubgroup}{/input/model/unit\_XXX/adsorption}{ADSORPTION\_MODEL = LINEAR}{tab:FFAdsorptionLinear}
  \begin{dataset}[type=int,range={$\{ 0,1 \}$},length=1]{IS\_KINETIC}
    Selects kinetic or quasi-stationary adsorption mode: 1 = kinetic, 0 = quasi-stationary
  \end{dataset}
  \begin{dataset}[unit=\si{\cubic\metre\of{MP}\per\cubic\metre\of{SP}\per\second},type=double,range={$\geq 0$},length={\texttt{NCOMP}}]{LIN\_KA}
    Adsorption rate constants for each component
  \end{dataset}
  \begin{dataset}[unit=\si{\per\second},type=double,range={$\geq 0$},length={\texttt{NCOMP}}]{LIN\_KD}
    Desorption rate constants for each component
  \end{dataset}
\end{condsubgroup}

\begin{condsubgroup}{/input/model/unit\_XXX/adsorption}{ADSORPTION\_MODEL = MULTI\_COMPONENT\_LANGMUIR}{tab:FFAdsorptionMultiCompLangmuir}
  \begin{dataset}[type=int,range={$\{ 0,1 \}$},length=1]{IS\_KINETIC}
    Selects kinetic or quasi-stationary adsorption mode: 1 = kinetic, 0 = quasi-stationary
  \end{dataset}
  \begin{dataset}[type=double,range={$\geq 0$},length={\texttt{NCOMP}}]{MCL\_KA}
    Adsorption rate constants
  \end{dataset}
  \begin{dataset}[unit=\si{\per\second}, type=double,range={$\geq 0$},length={\texttt{NCOMP}}]{MCL\_KD}
    Desorption rate constants
  \end{dataset}
  \begin{dataset}[unit=\si{\mol\per\cubic\metre\of{SP}}, type=double,range={$> 0$},length={\texttt{NCOMP}}]{MCL\_QMAX}
    Maximum adsorption capacities
  \end{dataset}
\end{condsubgroup}

\begin{condsubgroup}{/input/model/unit\_XXX/adsorption}{ADSORPTION\_MODEL = MULTI\_COMPONENT\_ANTILANGMUIR}{tab:FFAdsorptionMultiCompAntiLangmuir}
  \begin{dataset}[type=int,range={$\{ 0,1 \}$},length=1]{IS\_KINETIC}
    Selects kinetic or quasi-stationary adsorption mode: 1 = kinetic, 0 = quasi-stationary
  \end{dataset}
  \begin{dataset}[unit=\si{\cubic\metre\of{MP}\per\mol\per\second}, type=double,range={$\geq 0$},length={\texttt{NCOMP}}]{MCAL\_KA}
    Adsorption rate constants
  \end{dataset}
  \begin{dataset}[unit=\si{\per\second}, type=double,range={$\geq 0$},length={\texttt{NCOMP}}]{MCAL\_KD}
    Desorption rate constants
  \end{dataset}
  \begin{dataset}[unit=\si{\mol\per\cubic\metre\of{SP}}, type=double,range={$> 0$},length={\texttt{NCOMP}}]{MCAL\_QMAX}
    Maximum adsorption capacities
  \end{dataset}
  \begin{dataset}[unit=\si{\mol\per\cubic\metre\of{SP}}, type=double,range={$\{ -1, 1\}$},length={\texttt{NCOMP}}]{MCAL\_ANTILANGMUIR}
    Anti-Langmuir coefficients (optional)
  \end{dataset}
\end{condsubgroup}

\begin{condsubgroup}{/input/model/unit\_XXX/adsorption}{ADSORPTION\_MODEL = MOBILE\_PHASE\_MODULATOR}{tab:FFAdsorptionMobilePhaseModulator}
\begin{dataset}[type=int,range={$\{ 0,1 \}$},length=1]{IS\_KINETIC}
    Selects kinetic or quasi-stationary adsorption mode: 1 = kinetic, 0 = quasi-stationary
  \end{dataset}
  \begin{dataset}[unit=\si{\cubic\metre\of{MP}\per\mol\per\second}, type=double,range={$\geq 0$},length={\texttt{NCOMP}}]{MPM\_KA}
    Adsorption rate constants
  \end{dataset}
  \begin{dataset}[unit=\si{\raiseto{3\beta}\metre\of{MP}\per\raiseto{\beta}\mol\per\second}, type=double,range={$\geq 0$},length={\texttt{NCOMP}}]{MPM\_KD}
    Desorption rate constants
  \end{dataset}
  \begin{dataset}[unit=\si{\mol\per\cubic\metre\of{SP}}, type=double,range={$\geq 0$},length={\texttt{NCOMP}}]{MPM\_QMAX}
    Maximum adsorption capacities
  \end{dataset}
  \begin{dataset}[unit=\si{\mol\per\cubic\metre\of{SP}}, type=double,range={$\geq 0$},length={\texttt{NCOMP}}]{MPM\_BETA}
    Parameters describing the ion-exchange characteristics (IEX)
  \end{dataset}
  \begin{dataset}[unit=\si{\cubic\metre\of{MP}\per\mol}, type=double,range={$\geq 0$},length={\texttt{NCOMP}}]{MPM\_GAMMA}
    Parameters describing the hydrophobicity (HIC)
  \end{dataset}
\end{condsubgroup}
  
\begin{condsubgroup}{/input/model/unit\_XXX/adsorption}{ADSORPTION\_MODEL = STERIC\_MASS\_ACTION}{tab:FFAdsorptionStericMassAction}
  \begin{dataset}[type=int,range={$\{ 0,1 \}$},length=1]{IS\_KINETIC}
    Selects kinetic or quasi-stationary adsorption mode: 1 = kinetic, 0 = quasi-stationary
  \end{dataset}
  \begin{dataset}[unit=\si{\raiseto{3}\metre\of{MP}\per\raiseto{3}\metre\of{SP}\per\second}, type=double,range={$\geq 0$},length={\texttt{NCOMP}}]{SMA\_KA}
    Adsorption rate constants
  \end{dataset}
  \begin{dataset}[unit=\si{\per\second}, type=double,range={$\geq 0$},length={\texttt{NCOMP}}]{SMA\_KD}
    Desorption rate constants
  \end{dataset}
  \begin{dataset}[type=double,range={$\geq 0$},length={\texttt{NCOMP}}]{SMA\_NU}
    Characteristic charges of the protein; The number of sites $\nu$ that the protein interacts with on the resin surface
  \end{dataset}
  \begin{dataset}[type=double,range={$\geq 0$},length={\texttt{NCOMP}}]{SMA\_SIGMA}
    Steric factors of the protein; The number of sites $\sigma$ on the surface that are shielded by the protein and prevented from exchange with the salt counterions in solution
  \end{dataset}
  \begin{dataset}[unit=\si{\mol\per\cubic\metre\of{SP}}, type=double,range={$\geq 0$},length={1}]{SMA\_LAMBDA}
    Stationary phase capacity (monovalent salt counterions); The total number of binding sites available on the resin surface 
  \end{dataset}
  \begin{dataset}[unit=\si{\mol\per\raiseto{3}\metre\of{MP}}, type=double,range={$> 0$},length={1}]{SMA\_REFC0}
    Reference liquid phase concentration (optional, defaults to $1.0$)
  \end{dataset}
  \begin{dataset}[unit=\si{\mol\per\raiseto{3}\metre\of{SP}}, type=double,range={$> 0$},length={1}]{SMA\_REFQ}
    Reference solid phase concentration (optional, defaults to $1.0$) 
  \end{dataset}
\end{condsubgroup}

\begin{condsubgroup}{/input/model/unit\_XXX/adsorption}{ADSORPTION\_MODEL = SELF\_ASSOCIATION}{tab:FFAdsorptionSelfAssociation}
\begin{dataset}[type=int,range={$\{ 0,1 \}$},length=1]{IS\_KINETIC}
    Selects kinetic or quasi-stationary adsorption mode: 1 = kinetic, 0 = quasi-stationary
  \end{dataset}
  \begin{dataset}[unit=\si{\raiseto{3}\metre\of{MP}\per\raiseto{3}\metre\of{SP}\per\second}, type=double,range={$\geq 0$},length={\texttt{NCOMP}}]{SAI\_KA1}
    Adsorption rate constants
  \end{dataset}
  \begin{dataset}[unit=\si{\raiseto{6}\metre\of{MP}\per\raiseto{6}\metre\of{SP}\per\second}, type=double,range={$\geq 0$},length={\texttt{NCOMP}}]{SAI\_KA2}
    Adsorption rate constants
  \end{dataset}
  \begin{dataset}[unit=\si{\per\second}, type=double,range={$\geq 0$},length={\texttt{NCOMP}}]{SAI\_KD}
    Desorption rate constants
  \end{dataset}
  \begin{dataset}[type=double,range={$\geq 0$},length={\texttt{NCOMP}}]{SAI\_NU}
    Characteristic charges $\nu$ of the protein
  \end{dataset}
  \begin{dataset}[type=double,range={$\geq 0$},length={\texttt{NCOMP}}]{SAI\_SIGMA}
    Steric factors $\sigma$ of the protein
  \end{dataset}
  \begin{dataset}[unit=\si{\mol\per\cubic\metre\of{SP}}, type=double,range={$\geq 0$},length={1}]{SAI\_LAMBDA}
    Stationary phase capacity (monovalent salt counterions); The total number of binding sites available on the resin surface
  \end{dataset}
  \begin{dataset}[unit=\si{\mol\per\raiseto{3}\metre\of{MP}}, type=double,range={$> 0$},length={1}]{SAI\_REFC0}
    Reference liquid phase concentration (optional, defaults to $1.0$)
  \end{dataset}
  \begin{dataset}[unit=\si{\mol\per\raiseto{3}\metre\of{SP}}, type=double,range={$> 0$},length={1}]{SAI\_REFQ}
    Reference solid phase concentration (optional, defaults to $1.0$) 
  \end{dataset}
\end{condsubgroup}

\begin{condsubgroup}{/input/model/unit\_XXX/adsorption}{ADSORPTION\_MODEL = SASKA}{tab:FFAdsorptionSaska}
\begin{dataset}[type=int,range={$\{ 0,1 \}$},length=1]{IS\_KINETIC}
    Selects kinetic or quasi-stationary adsorption mode: 1 = kinetic, 0 = quasi-stationary
  \end{dataset}
  \begin{dataset}[unit=\si{\cubic\metre\of{MP}\per\cubic\metre\of{SP}\per\second}, type=double,range={$\mathds{R}$},length={\texttt{NCOMP}}]{SASKA\_H}
    Henry coefficient
  \end{dataset}
  \begin{dataset}[unit=\si{\raiseto{6}\metre\of{MP}\per\cubic\metre\of{SP}\per\mol\per\second}, type=double,range={$\mathds{R}$},length={$\texttt{NCOMP}^2$}]{SASKA\_K}
    Quadratic factors
  \end{dataset}
\end{condsubgroup}

\begin{condsubgroup}{/input/model/unit\_XXX/adsorption}{ADSORPTION\_MODEL = MULTI\_COMPONENT\_BILANGMUIR}{tab:FFAdsorptionBiLangmuir}
\begin{dataset}[type=int,range={$\{ 0,1 \}$},length=1]{IS\_KINETIC}
    Selects kinetic or quasi-stationary adsorption mode: 1 = kinetic, 0 = quasi-stationary
  \end{dataset}
  \begin{dataset}[unit=\si{\cubic\metre\of{MP}\per\mol\per\second}, type=double,range={$\geq 0$},length={$\texttt{NSTATES} \cdot \texttt{NCOMP}$}]{MCBL\_KA}
    Adsorption rate constants in state-major ordering
  \end{dataset}
  \begin{dataset}[unit=\si{\per\second}, type=double,range={$\geq 0$},length={$\texttt{NSTATES} \cdot \texttt{NCOMP}$}]{MCBL\_KD}
    Desorption rate constants in state-major ordering
  \end{dataset}
  \begin{dataset}[unit=\si{\mol\per\cubic\metre\of{SP}}, type=double,range={$> 0.0$},length={$\texttt{NSTATES} \cdot \texttt{NCOMP}$}]{MCBL\_QMAX}
    Maximum adsorption capacities in state-major ordering
  \end{dataset}
\end{condsubgroup}

\begin{condsubgroup}{/input/model/unit\_XXX/adsorption}{ADSORPTION\_MODEL = KUMAR\_MULTI\_COMPONENT\_LANGMUIR}{tab:FFAdsorptionKumarLangmuir}
\begin{dataset}[type=int,range={$\{ 0,1 \}$},length=1]{IS\_KINETIC}
    Selects kinetic or quasi-stationary adsorption mode: 1 = kinetic, 0 = quasi-stationary
  \end{dataset}
  \begin{dataset}[unit=\si{\raiseto{3}\metre\of{MP}\per\mol\per\second}, type=double,range={$\geq 0$},length={\texttt{NCOMP}}]{KMCL\_KA}
    Adsorption pre-exponential factors
  \end{dataset}
  \begin{dataset}[unit=\si{\raiseto{3\nu_i}\metre\of{MP}\per\raiseto{\nu_i}\mol\per\second}, type=double,range={$\geq 0$},length={\texttt{NCOMP}}]{KMCL\_KD}
    Desorption rate
  \end{dataset}
\begin{dataset}[unit=\si{\kelvin}, type = double, range={$\geq 0$}, length={\texttt{NCOMP}}]{KMCL\_KACT} 
  Activation temperatures
  \end{dataset}
\begin{dataset}[unit=\si{\mol\per\cubic\metre\of{SP}} , type = double, range={$> 0$}, length={\texttt{NCOMP}}]{KMCL\_QMAX} 
  Maximum adsorption capacities
  \end{dataset}
\begin{dataset}[type = double, range={$> 0$}, length={\texttt{NCOMP}}]{KMCL\_NU} 
  Salt exponents / characteristic charges 
  \end{dataset}
\begin{dataset}[unit=\si{\kelvin}, type = double, range={$\geq 0$}, length={1}]{KMCL\_TEMP} 
  Temperature 
  \end{dataset}
\end{condsubgroup}

\begin{condsubgroup}{/input/model/unit\_XXX/adsorption}{ADSORPTION\_MODEL = MULTI\_COMPONENT\_SPREADING}{tab:FFAdsorptionMultiCompSpreading}
\begin{dataset}[type=int,range={$\{ 0,1 \}$},length=1]{IS\_KINETIC}
    Selects kinetic or quasi-stationary adsorption mode: 1 = kinetic, 0 = quasi-stationary
  \end{dataset}
  \begin{dataset}[unit=\si{\cubic\metre\of{MP}\per\mol\per\second}, type = double, range={$\geq 0$}, length={\texttt{NTOTALBND}}]{MCSPR\_KA} 
    Adsorption rate constants in state-major ordering   
  \end{dataset}  
  \begin{dataset}[unit=\si{\per\second}, type = double, range={$\geq 0$}, length={\texttt{NTOTALBND}}]{MCSPR\_KD} 
    Desorption rate constants in state-major ordering  
  \end{dataset}  
  \begin{dataset}[unit=\si{\mol\per\cubic\metre\of{SP}}, type = double, range={$> 0$}, length={\texttt{NTOTALBND}}]{MCSPR\_QMAX} 
    Maximum adsorption capacities in state-major ordering  
  \end{dataset}  
  \begin{dataset}[unit=\si{\per\second}, type = double, range={$\geq 0$}, length={\texttt{NCOMP}}]{MCSPR\_K12} 
    Exchange rates from the first to the second bound state 
  \end{dataset}  
  \begin{dataset}[unit=\si{\per\second}, type = double, range={$\geq 0$}, length={\texttt{NCOMP}}]{MCSPR\_K21} 
    Exchange rates from the second to the first bound state 
  \end{dataset}  
\end{condsubgroup}

\begin{condsubgroup}{/input/model/unit\_XXX/adsorption}{ADSORPTION\_MODEL = MULTISTATE\_STERIC\_MASS\_ACTION}{tab:FFAdsorptionMultiStateStericMassAction}
\begin{dataset}[type=int,range={$\{ 0,1 \}$},length=1]{IS\_KINETIC}
    Selects kinetic or quasi-stationary adsorption mode: 1 = kinetic, 0 = quasi-stationary
  \end{dataset}
  \begin{dataset}[unit=\si{\raiseto{3}\metre\of{MP}\per\raiseto{3}\metre\of{SP}\per\second}, type = double, range={$\geq 0$}, length={\texttt{NTOTALBND}}]{MSSMA\_KA} 
    Adsorption rate constants of the components to the different bound states in component-major ordering
  \end{dataset} 
  \begin{dataset}[unit=\si{\per\second} , type = double, range={$\geq 0$}, length={\texttt{NTOTALBND}}]{MSSMA\_KD} 
    Desorption rate constants of the components in the different bound states in component-major ordering
  \end{dataset} 
  \begin{dataset}[type = double, range={$\geq 0$}, length={\texttt{NTOTALBND}}]{MSSMA\_NU} 
    Characteristic charges of the components in the different bound states in component-major ordering
  \end{dataset} 
  \begin{dataset}[type = double, range={$\geq 0$}, length={\texttt{NTOTALBND}}]{MSSMA\_SIGMA} 
    Steric factors of the components in the different bound states in component-major ordering
  \end{dataset} 
  \begin{dataset}[unit=\si{\per\second}, type = double, range={$\geq 0$}, length={$\sum_{i=0}^{\texttt{NCOMP} - 1} \texttt{NBND}_i^2$}]{MSSMA\_RATES} 
    Conversion rates between different bound states in component-row-major ordering 
  \end{dataset}
  \begin{dataset}[unit=\si{\mol\per\cubic\metre\of{SP}}, type = double, range={$\geq 0$}, length={1}]{MSSMA\_LAMBDA} 
    Stationary phase capacity (monovalent salt counterions); The total number of binding sites available on the resin surface 
  \end{dataset}
  \begin{dataset}[unit=\si{\mol\per\raiseto{3}\metre\of{MP}}, type = double, range={$> 0$}, length={1}]{MSSMA\_REFC0} 
    Reference liquid phase concentration (optional, defaults to $1.0$) 
  \end{dataset} 
  \begin{dataset}[unit=\si{\mol\per\raiseto{3}\metre\of{SP}}, type = double, range={$> 0$}, length={1}]{MSSMA\_REFQ} 
    Reference solid phase concentration (optional, defaults to $1.0$)
  \end{dataset} 
\end{condsubgroup}

\begin{condsubgroup}{/input/model/unit\_XXX/adsorption}{ADSORPTION\_MODEL = SIMPLE\_MULTISTATE\_STERIC\_MASS\_ACTION}{tab:FFAdsorptionSimpleMultiStateStericMassAction}
\begin{dataset}[type=int,range={$\{ 0,1 \}$},length=1]{IS\_KINETIC}
    Selects kinetic or quasi-stationary adsorption mode: 1 = kinetic, 0 = quasi-stationary
  \end{dataset}
  \begin{dataset}[unit=\si{\mol\per\cubic\metre\of{SP}}, type = double, range={$\geq 0$}, length={1}]{SMSSMA\_LAMBDA} 
    Stationary phase capacity (monovalent salt counterions); The total number of binding sites available on the resin surface
  \end{dataset}
  \begin{dataset}[unit=\si{\raiseto{3}\metre\of{MP}\per\raiseto{3}\metre\of{SP}\per\second} , type = double, range={$\geq 0$}, length={\texttt{NTOTALBND}}]{SMSSMA\_KA} 
    Adsorption rate constants of the components to the different bound states in component-major ordering
  \end{dataset} 
  \begin{dataset}[unit=\si{\per\second}, type = double, range={$\geq 0$}, length={\texttt{NTOTALBND}}]{SMSSMA\_KD} 
    Desorption rate constants of the components to the different bound states in component-major ordering
  \end{dataset} 
  \begin{dataset}[unit=\si{\per\second}, type = double, range={$\geq 0$}, length={\texttt{NCOMP}}]{SMSSMA\_NU\_MIN} 
    Characteristic charges of the components in the first (weakest) bound state
  \end{dataset} 
  \begin{dataset}[type = double, range={$\geq 0$}, length={\texttt{NCOMP}}]{SMSSMA\_NU\_MAX} 
    Characteristic charges of the components in the last (strongest) bound state
  \end{dataset} 
  \begin{dataset}[type = double, range={$\mathds{R}$}, length={\texttt{NCOMP}}]{SMSSMA\_NU\_QUAD} 
    Quadratic modifiers of the characteristic charges of the different components depending on the index of the bound state
  \end{dataset} 
  \begin{dataset}[type = double, range={$\geq 0$}, length={\texttt{NCOMP}}]{SMSSMA\_SIGMA\_MIN} 
    Steric factors of the components in the first (weakest) bound state
  \end{dataset} 
  \begin{dataset}[type = double, range={$\geq 0$}, length={\texttt{NCOMP}}]{SMSSMA\_SIGMA\_MAX} 
    Steric factors of the components in the last (strongest) bound state
  \end{dataset} 
  \begin{dataset}[type = double, range={$\mathds{R}$}, length={\texttt{NCOMP}}]{SMSSMA\_SIGMA\_QUAD} 
    Quadratic modifiers of steric factors of the different components depending on the index of the bound state
  \end{dataset} 
  \begin{dataset}[unit=\si{\per\second}, type = double, range={$\geq 0$}, length={\texttt{NCOMP}}]{SMSSMA\_KWS} 
    Exchange rates from a weakly bound state to the next stronger bound state
  \end{dataset} 
  \begin{dataset}[unit=\si{\per\second}, type = double, range={$\mathds{R}$}, length={\texttt{NCOMP}}]{SMSSMA\_KWS\_LIN} 
    Linear exchange rate coefficients from a weakly bound state to the next stronger bound state
  \end{dataset} 
  \begin{dataset}[unit=\si{\per\second}, type = double, range={$\mathds{R}$}, length={\texttt{NCOMP}}]{SMSSMA\_KWS\_QUAD} 
    Quadratic exchange rate coefficients from a weakly bound state to the next stronger bound state
  \end{dataset} 
  \begin{dataset}[unit=\si{\per\second}, type = double, range={$\geq 0$}, length={\texttt{NCOMP}}]{SMSSMA\_KSW} 
    Exchange rates from a strongly bound state to the next weaker bound state
  \end{dataset} 
  \begin{dataset}[unit=\si{\per\second}, type = double, range={$\mathds{R}$}, length={\texttt{NCOMP}}]{SMSSMA\_KSW\_LIN} 
    Linear exchange rate coefficients from a strongly bound state to the next weaker bound state
  \end{dataset} 
  \begin{dataset}[unit=\si{\per\second}, type = double, range={$\mathds{R}$}, length={\texttt{NCOMP}}]{SMSSMA\_KSW\_QUAD} 
    Quadratic exchange rate coefficients from a strongly bound state to the next weaker bound state
  \end{dataset} 
  \begin{dataset}[unit=\si{\mol\per\raiseto{3}\metre\of{MP}}, type = double, range={$> 0$}, length={1}]{SMSSMA\_REFC0} 
    Reference liquid phase concentration (optional, defaults to $1.0$) 
  \end{dataset} 
  \begin{dataset}[unit=\si{\mol\per\raiseto{3}\metre\of{SP}}, type = double, range={$> 0$}, length={1}]{SMSSMA\_REFQ} 
    Reference solid phase concentration (optional, defaults to $1.0$)
  \end{dataset} 
\end{condsubgroup}

\begin{condsubgroup}{/input/model/unit\_XXX/adsorption}{ADSORPTION\_MODEL = BI\_STERIC\_MASS\_ACTION}{tab:FFAdsorptionBiStericMassAction}
\begin{dataset}[type=int,range={$\{ 0,1 \}$},length=1]{IS\_KINETIC}
    Selects kinetic or quasi-stationary adsorption mode: 1 = kinetic, 0 = quasi-stationary
  \end{dataset}
  \begin{dataset}[unit=\si{\raiseto{3}\metre\of{MP}\per\raiseto{3}\metre\of{SP}\per\second}, type = double, range={$\geq 0.0$}, length={$\texttt{NSTATES} \cdot \texttt{NCOMP}$}]{BISMA\_KA} 
    Adsorption rate constants in state-major ordering
  \end{dataset} 
  \begin{dataset}[unit=\si{\per\second}, type = double, range={$\geq 0.0$}, length={$\texttt{NSTATES} \cdot \texttt{NCOMP}$}]{BISMA\_KD} 
    Desorption rate constants in state-major ordering
  \end{dataset} 
  \begin{dataset}[unit=\si{\per\second}, type = double, range={$\geq 0.0$}, length={$\texttt{NSTATES} \cdot \texttt{NCOMP}$}]{BISMA\_NU} 
    Characteristic charges $\nu_{i,j}$ of the $i$th protein with respect to the $j$th binding site type in state-major ordering
  \end{dataset} 
  \begin{dataset}[unit=\si{\per\second}, type = double, range={$\geq 0.0$}, length={$\texttt{NSTATES} \cdot \texttt{NCOMP}$}]{BISMA\_SIGMA} 
    Steric factors $\sigma_{i,j}$ of the $i$th protein with respect to the $j$th binding site type in state-major ordering
  \end{dataset} 
  \begin{dataset}[unit=\si{\mol\per\cubic\metre\of{SP}}, type = double, range={$\geq 0.0$}, length={\texttt{NSTATES}}]{BISMA\_LAMBDA} 
    Stationary phase capacity (monovalent salt counterions) of the different binding site types $\lambda_j$
  \end{dataset}
  \begin{dataset}[unit=\si{\mol\per\raiseto{3}\metre\of{MP}}, type = double, range={$> 0$}, length={$\{1,\texttt{NSTATES}\}$}]{BISMA\_REFC0} 
    Reference liquid phase concentration for each binding site type or one value for all types (optional, defaults to $1.0$)
  \end{dataset} 
  \begin{dataset}[unit=\si{\mol\per\raiseto{3}\metre\of{SP}}, type = double, range={$> 0$}, length={$\{1,\texttt{NSTATES}\}$}]{BISMA\_REFQ} 
    Reference solid phase concentration for each binding site type or one value for all types (optional, defaults to $1.0$)
  \end{dataset} 
\end{condsubgroup}

\subsection{Return data}

\begin{groupscope}{/input/return}{tab:FFReturn}
  \begin{dataset}[type = int, range={$\{0,1\}$}]{WRITE\_SOLUTION\_TIMES}
    Write times at which a solution was produced (optional, defaults to 1)
  \end{dataset}
  \begin{dataset}[type = int, range={$\{0,1\}$}]{WRITE\_SOLUTION\_LAST}
    Write full solution state vector at last time point (optional, defaults to 0)
  \end{dataset}
  \begin{dataset}[type = int, range={$\{0,1\}$}]{WRITE\_SENS\_LAST}
    Write full sensitivity state vectors at last time point (optional, defaults to 0)
  \end{dataset}
  \begin{dataset}[type = int, range={$\{0,1\}$}]{SPLIT\_COMPONENTS\_DATA}
    Determines whether a joint dataset (matrix) for all components is created or if each component is put in a separate dataset (\texttt{XXX\_COMP\_000}, \texttt{XXX\_COMP\_001}, etc.) (optional, defaults to 1)
  \end{dataset}
  \begin{dataset}[type = int, range={$\{0,1\}$}]{SPLIT\_PORTS\_DATA}
    Determines whether a joint dataset (matrix) for all inlet/outlet ports is created or if each port is put in a separate dataset (\texttt{XXX\_PORT\_000}, \texttt{XXX\_PORT\_001}, etc.) (optional, defaults to 1)
  \end{dataset}
\end{groupscope}

\begin{groupscope}{/input/return/unit\_XXX}{tab:FFReturnUnit}
  \begin{dataset}[type=int, range={$\{0,1\}$}]{WRITE\_SOLUTION\_INLET}
    Write solutions at unit operation inlet $c^l_i(t,0)$
  \end{dataset}
  \begin{dataset}[type=int, range={$\{0,1\}$}]{WRITE\_SOLUTION\_OUTLET}
    Write solutions at unit operation outlet (chromatograms) $c^l_i(t,L)$
  \end{dataset}
  \begin{dataset}[type=int, range={$\{0,1\}$}]{WRITE\_SOLUTION\_BULK}
    Write solutions of the bulk volume $c^l_i$
  \end{dataset}
  \begin{dataset}[type=int, range={$\{0,1\}$}]{WRITE\_SOLUTION\_PARTICLE}
    Write solutions of the particle mobile phase $c^p_{j,i}$
  \end{dataset}
  \begin{dataset}[type=int, range={$\{0,1\}$}]{WRITE\_SOLUTION\_SOLID}
    Write solutions of the solid phase $c^s_{j,i,m_{j,i}}$
  \end{dataset}
  \begin{dataset}[type=int, range={$\{0,1\}$}]{WRITE\_SOLUTION\_FLUX}
    Write solutions of the bead fluxes $j_{f,i}$
  \end{dataset}
  \begin{dataset}[type=int, range={$\{0,1\}$}]{WRITE\_SOLUTION\_VOLUME}
    Write solutions of the volume $V$
  \end{dataset}
  \begin{dataset}[type=int, range={$\{0,1\}$}]{WRITE\_SOLDOT\_INLET}
    Write solution time derivatives at unit operation inlet $\partial c^l_i(t,0) / \partial t$
  \end{dataset}
  \begin{dataset}[type=int, range={$\{0,1\}$}]{WRITE\_SOLDOT\_OUTLET}
    Write solution time derivatives at unit operation outlet (chromatograms) $\partial c^l_i(t,L) / \partial t$
  \end{dataset}
  \begin{dataset}[type=int, range={$\{0,1\}$}]{WRITE\_SOLDOT\_BULK}
    Write solution time derivatives of the bulk volume $\partial c^l_i / \partial t$
  \end{dataset}
  \begin{dataset}[type=int, range={$\{0,1\}$}]{WRITE\_SOLDOT\_PARTICLE}
    Write solution time derivatives of the particle mobile phase $\partial c^p_{j,i} / \partial t$
  \end{dataset}
  \begin{dataset}[type=int, range={$\{0,1\}$}]{WRITE\_SOLDOT\_SOLID}
    Write solution time derivatives of the solid phase $\partial c^s_{j,i,m_{j,i}} / \partial t$
  \end{dataset}
  \begin{dataset}[type=int, range={$\{0,1\}$}]{WRITE\_SOLDOT\_FLUX}
    Write solution time derivatives of the bead fluxes $\partial j_{f,i} / \partial t$
  \end{dataset}
  \begin{dataset}[type=int, range={$\{0,1\}$}]{WRITE\_SOLDOT\_VOLUME}
    Write solution time derivatives of the volume $\partial V / \partial t$
  \end{dataset}
  \begin{dataset}[type=int, range={$\{0,1\}$}]{WRITE\_SENS\_INLET}
    Write sensitivities at unit operation inlet $\partial c^l_i(t,0) / \partial p$
  \end{dataset}
  \begin{dataset}[type=int, range={$\{0,1\}$}]{WRITE\_SENS\_OUTLET}
    Write sensitivities at unit operation outlet (chromatograms) $\partial c^l_i(t,L) / \partial p$
  \end{dataset}
  \begin{dataset}[type=int, range={$\{0,1\}$}]{WRITE\_SENS\_BULK}
    Write sensitivities of the bulk volume $\partial c^l_i / \partial p$
  \end{dataset}
  \begin{dataset}[type=int, range={$\{0,1\}$}]{WRITE\_SENS\_PARTICLE}
    Write sensitivities of the particle mobile phase $\partial c^p_{j,i} / \partial p$
  \end{dataset}
  \begin{dataset}[type=int, range={$\{0,1\}$}]{WRITE\_SENS\_SOLID}
    Write sensitivities of the solid phase $\partial c^s_{j,i,m_{j,i}} / \partial p$
  \end{dataset}
  \begin{dataset}[type=int, range={$\{0,1\}$}]{WRITE\_SENS\_FLUX}
    Write sensitivities of the bead fluxes $\partial j_{f,i} / \partial p$
  \end{dataset}
  \begin{dataset}[type=int, range={$\{0,1\}$}]{WRITE\_SENS\_VOLUME}
    Write sensitivities of the volume $\partial V / \partial p$
  \end{dataset}
  \begin{dataset}[type=int, range={$\{0,1\}$}]{WRITE\_SENSDOT\_INLET}
    Write sensitivity time derivatives at unit operation inlet $\partial^2 c^l_i(t,0) / (\partial p, \partial t)$
  \end{dataset}
  \begin{dataset}[type=int, range={$\{0,1\}$}]{WRITE\_SENSDOT\_OUTLET}
    Write sensitivity time derivatives at unit operation outlet (chromatograms) $\partial^2 c^l_i(t,L) / (\partial p, \partial t)$
  \end{dataset}
  \begin{dataset}[type=int, range={$\{0,1\}$}]{WRITE\_SENSDOT\_BULK}
    Write sensitivity time derivatives of the bulk volume $\partial^2 c^l_i / (\partial p, \partial t)$
  \end{dataset}
  \begin{dataset}[type=int, range={$\{0,1\}$}]{WRITE\_SENSDOT\_PARTICLE}
    Write sensitivity time derivatives of the particle mobile phase $\partial^2 c^p_{j,i} / (\partial p, \partial t)$
  \end{dataset}
  \begin{dataset}[type=int, range={$\{0,1\}$}]{WRITE\_SENSDOT\_SOLID}
    Write sensitivity time derivatives of the solid phase $\partial^2 c^s_{j,i,m_{j,i}} / (\partial p, \partial t)$
  \end{dataset}
  \begin{dataset}[type=int, range={$\{0,1\}$}]{WRITE\_SENSDOT\_FLUX}
    Write sensitivity time derivatives of the bead fluxes $\partial^2 j_{f,i} / (\partial p, \partial t)$
  \end{dataset}
  \begin{dataset}[type=int, range={$\{0,1\}$}]{WRITE\_SENSDOT\_VOLUME}
    Write sensitivity time derivatives of the volume $\partial^2 V / (\partial p, \partial t)$
  \end{dataset}
\end{groupscope}

\subsection{Parameter sensitivities}

\begin{groupscope}{/input/sensitivity}{tab:FFSensitivity} 
  \begin{dataset}[type=int, range={$\geq 0$}, length=1]{NSENS}
    Number of sensitivities to be computed
  \end{dataset}     
  \begin{dataset}[type=string, range={\texttt{ad1}}, length=1]{SENS\_METHOD}
    Method used for computation of sensitivities (algorithmic differentiation)
  \end{dataset}    
\end{groupscope}

\begin{groupscope}{/input/sensitivity/param\_XXX}{tab:FFSensitivityParam}
  \begin{dataset}[type = int, range={$\geq 0$}, length={$\geq 1$}]{SENS\_UNIT} 
    Unit operation index  
  \end{dataset}
  \begin{dataset}[type = string, range={See} ,length={$\geq 1$}]{SENS\_NAME} 
    Name of the parameter
  \end{dataset}
  \begin{dataset}[type = int, range={$\geq -1$}, length={$\geq 1$}]{SENS\_COMP} 
    Component index ($-1$ if parameter is independent of components)  
  \end{dataset}
  \begin{dataset}[type = int, range={$\geq -1$}, length={$\geq 1$}]{SENS\_REACTION} 
    Reaction index ($-1$ if parameter is independent of reactions)  
  \end{dataset}
  \begin{dataset}[type = int, range={$\geq -1$}, length={$\geq 1$}]{SENS\_BOUNDPHASE} 
    Bound phase index ($-1$ if parameter is independent of bound phases)  
  \end{dataset}
  \begin{dataset}[type = int, range={$\geq -1$}, length={$\geq 1$}]{SENS\_SECTION} 
    Section index ($-1$ if parameter is independent of sections)  
  \end{dataset}
  \begin{dataset}[type = double, range={$\geq 0.0$}, length={$\geq 1$}]{SENS\_ABSTOL} 
    Absolute tolerance used in the computation of the sensitivities (optional).
    Rule of thumb: \texttt{ABSTOL} / \texttt{PARAM\_VALUE} 
  \end{dataset}
  \begin{dataset}[type = double, range={$\mathds{R}$} ,length={$\geq 1$}]{SENS\_FACTOR} 
    Linear factor of the combined sensitivity (optional, taken as $1.0$ if left out) 
  \end{dataset}
\end{groupscope}

\subsection{Solver configuration}

\begin{groupscope}{/input/solver}{tab:FFSolver}
  \begin{dataset}[type=int,range={$\geq 1$},length=1]{NTHREADS}
    Number of used threads
  \end{dataset}
  \begin{dataset}[type=double,unit={\si{\second}},range={$\geq 0$},length={Arbitrary}]{USER\_SOLUTION\_TIMES}
    Vector with timepoints at which the solution is evaluated
  \end{dataset}
  \begin{dataset}[type=int,range={$\{ 0, \dots, 7\}$},length=1]{CONSISTENT\_INIT\_MODE}
    Consistent initialization mode (optional, defaults to $1$).
    Valid values are:
    \begin{description}
      \item[0] None
      \item[1] Full
      \item[2] Once, full
      \item[3] Lean
      \item[4] Once, lean
      \item[5] Full once, then lean
      \item[6] None once, then full
      \item[7] None once, then flean
    \end{description}\vspace{-\baselineskip}
  \end{dataset}
  \begin{dataset}[type=int,range={$\{ 0, \dots, 7\}$},length=1]{CONSISTENT\_INIT\_MODE\_SENS}
    Consistent initialization mode for parameter sensitivities (optional, defaults to $1$).
    Valid values are:
    \begin{description}
      \item[0] None
      \item[1] Full
      \item[2] Once, full
      \item[3] Lean
      \item[4] Once, lean
      \item[5] Full once, then lean
      \item[6] None once, then full
      \item[7] None once, then flean
    \end{description}\vspace{-\baselineskip}
  \end{dataset}
\end{groupscope}

\begin{groupscope}{/input/solver/time\_integrator}{tab:FFSolverTime}
  \begin{dataset}[type=double,range={$> 0$},length=1]{ABSTOL}
    Absolute tolerance in the solution of the original system
  \end{dataset}
  \begin{dataset}[type=double,range={$\geq 0$},length=1]{RELTOL}
    Relative tolerance in the solution of the original system
  \end{dataset}
  \begin{dataset}[type=double,range={$> 0$},length=1]{ALGTOL}
    Tolerance in the solution of the nonlinear consistency equations
  \end{dataset}
  \begin{dataset}[type=double,range={$\geq 0$},length=1]{RELTOL\_SENS}
    Relative tolerance in the solution of the sensitivity systems
  \end{dataset}
  \begin{dataset}[type=double,unit={\si{\second}},range={$\geq 0$},length={1 / \texttt{NSEC}}]{INIT\_STEP\_SIZE}
    Initial time integrator step size for each section or one value for all sections ($0.0$: IDAS default value), see IDAS guide 4.5, p.\ 36f.
  \end{dataset}
  \begin{dataset}[type=int,range={$\geq 0$},length=1]{MAX\_STEPS}
    Maximum number of timesteps taken by IDAS (0: IDAS default = 500), see IDAS guide Sec.~4.5
  \end{dataset}
  \begin{dataset}[type=double,unit={\si{\second}},range={$\geq 0$},length=1]{MAX\_STEP\_SIZE}
    Maximum size of timesteps taken by IDAS (optional, defaults to unlimited $0.0$), see IDAS guide Sec.~4.5
  \end{dataset}
  \begin{dataset}[type=int,range={$\{0,1\}$},length=1]{ERRORTEST\_SENS}
    Determines whether (forward) sensitivities take part in local error test (optional, defaults to $1$)
  \end{dataset}
  \begin{dataset}[type=int,range={$\geq 0$},length=1]{MAX\_NEWTON\_ITER}
    Maximum number of Newton iterations in time step (optional, defaults to $3$)
  \end{dataset}
  \begin{dataset}[type=int,range={$\geq 0$},length=1]{MAX\_ERRTEST\_FAIL}
    Maximum number of local error test failures in time step (optional, defaults to $7$)
  \end{dataset}
  \begin{dataset}[type=int,range={$\geq 0$},length=1]{MAX\_CONVTEST\_FAIL}
    Maximum number of Newton convergence test failures (optional, defaults to $10$)
  \end{dataset}
  \begin{dataset}[type=int,range={$\geq 0$},length=1]{MAX\_NEWTON\_ITER\_SENS}
    Maximum number of Newton iterations in forward sensitivity time step (optional, defaults to $3$)
  \end{dataset}
\end{groupscope}

\begin{groupscope}{/input/solver/sections}{tab:FFSolverSections}
  \begin{dataset}[type=int,range={$\geq 1$},length=1]{NSEC}
    Number of sections
  \end{dataset}
  \begin{dataset}[type=double,unit={\si{\second}},range={$\geq 0$},length={$\texttt{NSEC}+1$}]{SECTION\_TIMES}
    Simulation times at which the model changes or behaves discontinously; including start and end times
  \end{dataset}
  \begin{dataset}[type=int,range={$\{0,1\}$},length={$\texttt{NSEC}-1$}]{SECTION\_CONTINUITY}
    Continuity indicator for each section transition: 0 (discontinuous) or 1 (continuous).
  \end{dataset}
\end{groupscope}

\section{Output group}\label{sec:FFOutput}

\begin{groupscope}{/output/solution}{tab:FFOutput}
  \begin{dataset}[type=double]{LAST\_STATE}
    Full state vector at the last time point of the time integrator if \texttt{WRITE\_SOLUTION\_LAST} in \texttt{/input/return} is enabled
  \end{dataset}
  \begin{dataset}[type=double]{LAST\_STATE\_YDOT}
    Full time derivative state vector at the last time point of the time integrator if \texttt{WRITE\_SOLUTION\_LAST} in \texttt{/input/return} is enabled
  \end{dataset}
  \begin{dataset}[type=double]{LAST\_STATE\_SENSY\_XXX}
    Full state vector of the \texttt{XXX}th sensitivity system at the last time point of the time integrator if \texttt{WRITE\_SENS\_LAST} in \texttt{/input/return} is enabled
  \end{dataset}
  \begin{dataset}[type=double]{LAST\_STATE\_SENSYDOT\_XXX}
    Full time derivative state vector of the \texttt{XXX}th sensitivity system at the last time point of the time integrator if \texttt{WRITE\_SENS\_LAST} in \texttt{/input/return} is enabled
  \end{dataset}
\end{groupscope}

\begin{groupscope}{/output/solution}{tab:FFOutputSolution}
  \begin{dataset}[type=double,unit={\si{\second}}]{SOLUTION\_TIMES}
    Time points at which the solution is written if \texttt{WRITE\_SOLUTION\_TIMES} in \texttt{/input/return} is enabled
  \end{dataset}
\end{groupscope}

\begin{groupscope}{/output/solution/unit\_XXX}{tab:FFOutputSensitivityUnit}
  \begin{dataset}[type=double,unit={\si{\mol\per\cubic\metre\of{IV}}}]{SOLUTION\_BULK}
    Interstitial solution as $n_{\text{Time}} \times \texttt{UNITOPORDERING}$ tensor in row-major storage
  \end{dataset}
  \begin{dataset}[type=double,unit={\si{\mol\per\cubic\metre\of{MP}}}]{SOLUTION\_PARTICLE}
    Mobile phase solution inside the particles as $n_{\text{Time}} \times \texttt{UNITOPORDERING}$ tensor in row-major storage.
    Only present if just one particle type is defined.
  \end{dataset}
  \begin{dataset}[type=double,unit={\si{\mol\per\cubic\metre\of{MP}}}]{SOLUTION\_PARTICLE\_PARTYPE\_XXX}
    Mobile phase solution inside the particles of type \texttt{XXX} as $n_{\text{Time}} \times \texttt{UNITOPORDERING}$ tensor in row-major storage.
    Only present if more than one particle type is defined.
  \end{dataset}
  \begin{dataset}[type=double,unit={\si{\mol\per\cubic\metre\of{MP}}}]{SOLUTION\_SOLID}
    Solid phase solution inside the particles as $n_{\text{Time}} \times \texttt{UNITOPORDERING}$ tensor in row-major storage.
    Only present if just one particle type is defined.
  \end{dataset}
  \begin{dataset}[type=double,unit={\si{\mol\per\cubic\metre\of{SP}}}]{SOLUTION\_SOLID\_PARTYPE\_XXX}
    Solid phase solution inside the particles of type \texttt{XXX} as $n_{\text{Time}} \times \texttt{UNITOPORDERING}$ tensor in row-major storage.
    Only present if more than one particle type is defined.
  \end{dataset}
  \begin{dataset}[type=double,unit={\si{\mol\per\square\metre\per\second}}]{SOLUTION\_FLUX}
    Flux solution as $n_{\text{Time}} \times \texttt{UNITOPORDERING}$ tensor in row-major storage
  \end{dataset}
  \begin{dataset}[type=double,unit={\si{\cubic\metre}}]{SOLUTION\_VOLUME}
    Volume solution
  \end{dataset}
  \begin{dataset}[type=double,unit={\si{\mol\per\cubic\metre\of{IV}}}]{SOLUTION\_OUTLET}
    Tensor of solutions at the unit operation outlet with components as columns in time-port-major storage.
    Only present if \texttt{SPLIT\_COMPONENTS\_DATA} and \texttt{SPLIT\_PORTS\_DATA} are both disabled.
  \end{dataset}
  \begin{dataset}[type=double,unit={\si{\mol\per\cubic\metre\of{IV}}}]{SOLUTION\_INLET}
    Tensor of solutions at the unit operation inlet with components as columns in time-port-major storage.
    Only present if \texttt{SPLIT\_COMPONENTS\_DATA} and \texttt{SPLIT\_PORTS\_DATA} are both disabled.
  \end{dataset}
  \begin{dataset}[type=double,unit={\si{\mol\per\cubic\metre\of{IV}}}]{SOLUTION\_OUTLET\_COMP\_XXX}
    Component \texttt{XXX} of the solution at all outlet ports of the unit operation as matrix in time-major storage.
    Only present if \texttt{SPLIT\_COMPONENTS\_DATA} is enabled and \texttt{SPLIT\_PORTS\_DATA} is disabled.
  \end{dataset}
  \begin{dataset}[type=double,unit={\si{\mol\per\cubic\metre\of{IV}}}]{SOLUTION\_INLET\_COMP\_XXX}
    Component \texttt{XXX} of the solution at all inlet ports of the unit operation inlet as matrix in time-major storage.
    Only present if \texttt{SPLIT\_COMPONENTS\_DATA} is enabled and \texttt{SPLIT\_PORTS\_DATA} is disabled.
  \end{dataset}
  \begin{dataset}[type=double,unit={\si{\mol\per\cubic\metre\of{IV}}}]{SOLUTION\_OUTLET\_PORT\_XXX}
    All components at outlet port \texttt{XXX} of the solution of the unit operation as matrix in time-major storage.
    Only present if \texttt{SPLIT\_COMPONENTS\_DATA} is disabled and \texttt{SPLIT\_PORTS\_DATA} is enabled.
  \end{dataset}
  \begin{dataset}[type=double,unit={\si{\mol\per\cubic\metre\of{IV}}}]{SOLUTION\_INLET\_PORT\_XXX}
    All components at inlet port \texttt{XXX} of the solution of the unit operation inlet as matrix in time-major storage.
    Only present if \texttt{SPLIT\_COMPONENTS\_DATA} is disabled and \texttt{SPLIT\_PORTS\_DATA} is enabled.
  \end{dataset}
  \begin{dataset}[type=double,unit={\si{\mol\per\cubic\metre\of{IV}}}]{SOLUTION\_OUTLET\_PORT\_XXX\_COMP\_YYY}
    Component \texttt{YYY} at outlet port \texttt{XXX} of the solution of the unit operation.
    Only present if \texttt{SPLIT\_COMPONENTS\_DATA} is enabled and \texttt{SPLIT\_PORTS\_DATA} is enabled.
  \end{dataset}
  \begin{dataset}[type=double,unit={\si{\mol\per\cubic\metre\of{IV}}}]{SOLUTION\_INLET\_PORT\_XXX\_COMP\_YYY}
    Component \texttt{YYY} at inlet port \texttt{XXX} of the solution of the unit operation.
    Only present if \texttt{SPLIT\_COMPONENTS\_DATA} is enabled and \texttt{SPLIT\_PORTS\_DATA} is enabled.
  \end{dataset}
  \begin{dataset}[type=double,unit={\si{\mol\per\cubic\metre\of{IV}\per\second}}]{SOLDOT\_BULK}
    Interstitial solution time derivative as $n_{\text{Time}} \times \texttt{UNITOPORDERING}$ tensor in row-major storage
  \end{dataset}
  \begin{dataset}[type=double,unit={\si{\mol\per\cubic\metre\of{MP}\per\second}}]{SOLDOT\_PARTICLE}
    Mobile phase solution time derivative inside the particles as $n_{\text{Time}} \times \texttt{UNITOPORDERING}$ tensor in row-major storage.
    Only present if just one particle type is defined.
  \end{dataset}
  \begin{dataset}[type=double,unit={\si{\mol\per\cubic\metre\of{MP}\per\second}}]{SOLDOT\_PARTICLE\_PARTYPE\_XXX}
    Mobile phase solution time derivative inside the particles of type \texttt{XXX} as $n_{\text{Time}} \times \texttt{UNITOPORDERING}$ tensor in row-major storage.
    Only present if more than one particle type is defined.
  \end{dataset}
  \begin{dataset}[type=double,unit={\si{\mol\per\cubic\metre\of{MP}\per\second}}]{SOLDOT\_SOLID}
    Solid phase solution time derivative inside the particles as $n_{\text{Time}} \times \texttt{UNITOPORDERING}$ tensor in row-major storage.
    Only present if just one particle type is defined.
  \end{dataset}
  \begin{dataset}[type=double,unit={\si{\mol\per\cubic\metre\of{SP}\per\second}}]{SOLDOT\_SOLID\_PARTYPE\_XXX}
    Solid phase solution time derivative inside the particles of type \texttt{XXX} as $n_{\text{Time}} \times \texttt{UNITOPORDERING}$ tensor in row-major storage.
    Only present if more than one particle type is defined.
  \end{dataset}
  \begin{dataset}[type=double,unit={\si{\mol\per\square\metre\per\square\second}}]{SOLDOT\_FLUX}
    Flux solution time derivative as $n_{\text{Time}} \times \texttt{UNITOPORDERING}$ tensor in row-major storage
  \end{dataset}
  \begin{dataset}[type=double,unit={\si{\cubic\metre\per\second}}]{SOLDOT\_VOLUME}
    Volume solution time derivative 
  \end{dataset}
  \begin{dataset}[type=double,unit={\si{\mol\per\cubic\metre\of{IV}\per\second}}]{SOLDOT\_OUTLET}
    Tensor of solution time derivatives at the unit operation outlet with components as columns in time-port-major storage.
    Only present if \texttt{SPLIT\_COMPONENTS\_DATA} and \texttt{SPLIT\_PORTS\_DATA} are both disabled.
  \end{dataset}
  \begin{dataset}[type=double,unit={\si{\mol\per\cubic\metre\of{IV}\per\second}}]{SOLDOT\_INLET}
    Tensor of solution time derivatives at the unit operation inlet with components as columns in time-port-major storage.
    Only present if \texttt{SPLIT\_COMPONENTS\_DATA} and \texttt{SPLIT\_PORTS\_DATA} are both disabled.
  \end{dataset}
  \begin{dataset}[type=double,unit={\si{\mol\per\cubic\metre\of{IV}\per\second}}]{SOLDOT\_OUTLET\_COMP\_XXX}
    Component \texttt{XXX} of the solution time derivative at all outlet ports of the unit operation as matrix in time-major storage.
    Only present if \texttt{SPLIT\_COMPONENTS\_DATA} is enabled and \texttt{SPLIT\_PORTS\_DATA} is disabled.
  \end{dataset}
  \begin{dataset}[type=double,unit={\si{\mol\per\cubic\metre\of{IV}\per\second}}]{SOLDOT\_INLET\_COMP\_XXX}
    Component \texttt{XXX} of the solution time derivative at all inlet ports of the unit operation inlet as matrix in time-major storage.
    Only present if \texttt{SPLIT\_COMPONENTS\_DATA} is enabled and \texttt{SPLIT\_PORTS\_DATA} is disabled.
  \end{dataset}
  \begin{dataset}[type=double,unit={\si{\mol\per\cubic\metre\of{IV}\per\second}}]{SOLDOT\_OUTLET\_PORT\_XXX}
    All components at outlet port \texttt{XXX} of the solution time derivative of the unit operation as matrix in time-major storage.
    Only present if \texttt{SPLIT\_COMPONENTS\_DATA} is disabled and \texttt{SPLIT\_PORTS\_DATA} is enabled.
  \end{dataset}
  \begin{dataset}[type=double,unit={\si{\mol\per\cubic\metre\of{IV}\per\second}}]{SOLDOT\_INLET\_PORT\_XXX}
    All components at inlet port \texttt{XXX} of the solution time derivative of the unit operation inlet as matrix in time-major storage.
    Only present if \texttt{SPLIT\_COMPONENTS\_DATA} is disabled and \texttt{SPLIT\_PORTS\_DATA} is enabled.
  \end{dataset}
  \begin{dataset}[type=double,unit={\si{\mol\per\cubic\metre\of{IV}\per\second}}]{SOLDOT\_OUTLET\_PORT\_XXX\_COMP\_YYY}
    Component \texttt{YYY} at outlet port \texttt{XXX} of the solution time derivative of the unit operation.
    Only present if \texttt{SPLIT\_COMPONENTS\_DATA} is enabled and \texttt{SPLIT\_PORTS\_DATA} is enabled.
  \end{dataset}
  \begin{dataset}[type=double,unit={\si{\mol\per\cubic\metre\of{IV}\per\second}}]{SOLDOT\_INLET\_PORT\_XXX\_COMP\_YYY}
    Component \texttt{YYY} at inlet port \texttt{XXX} of the solution time derivative of the unit operation.
    Only present if \texttt{SPLIT\_COMPONENTS\_DATA} is enabled and \texttt{SPLIT\_PORTS\_DATA} is enabled.
  \end{dataset}
\end{groupscope}

\begin{groupscope}{/output/sensitivity/param\_XXX/unit\_YYY}{tab:FFOutputSensitivityParamUnit}
  \begin{dataset}[type=double,unit={\si{\mol\per\cubic\metre\of{IV}\per\ParamUnit}}]{SENS\_BULK}
    Interstitial sensitivity as $n_{\text{Time}} \times \texttt{UNITOPORDERING}$ tensor in row-major storage
  \end{dataset}
  \begin{dataset}[type=double,unit={\si{\mol\per\cubic\metre\of{MP}\per\ParamUnit}}]{SENS\_PARTICLE}
    Mobile phase sensitivity inside the particles as $n_{\text{Time}} \times \texttt{UNITOPORDERING}$ tensor in row-major storage.
    Only present if just one particle type is defined.
  \end{dataset}
  \begin{dataset}[type=double,unit={\si{\mol\per\cubic\metre\of{MP}\per\ParamUnit}}]{SENS\_PARTICLE\_PARTYPE\_XXX}
    Mobile phase sensitivity inside the particles of type \texttt{XXX} as $n_{\text{Time}} \times \texttt{UNITOPORDERING}$ tensor in row-major storage.
    Only present if more than one particle type is defined.
  \end{dataset}
  \begin{dataset}[type=double,unit={\si{\mol\per\cubic\metre\of{MP}\per\ParamUnit}}]{SENS\_SOLID}
    Solid phase sensitivity inside the particles as $n_{\text{Time}} \times \texttt{UNITOPORDERING}$ tensor in row-major storage.
    Only present if just one particle type is defined.
  \end{dataset}
  \begin{dataset}[type=double,unit={\si{\mol\per\cubic\metre\of{SP}\per\ParamUnit}}]{SENS\_SOLID\_PARTYPE\_XXX}
    Solid phase sensitivity inside the particles of type \texttt{XXX} as $n_{\text{Time}} \times \texttt{UNITOPORDERING}$ tensor in row-major storage.
    Only present if more than one particle type is defined.
  \end{dataset}
  \begin{dataset}[type=double,unit={\si{\mol\per\square\metre\per\second\per\ParamUnit}}]{SENS\_FLUX}
    Flux sensitivity as $n_{\text{Time}} \times \texttt{UNITOPORDERING}$ tensor in row-major storage
  \end{dataset}
  \begin{dataset}[type=double,unit={\si{\cubic\metre\per\ParamUnit}}]{SENS\_VOLUME}
    Volume sensitivity
  \end{dataset}
  \begin{dataset}[type=double,unit={\si{\mol\per\cubic\metre\of{IV}\per\ParamUnit}}]{SENS\_OUTLET}
    Tensor of sensitivities at the unit operation outlet with components as columns in time-port-major storage.
    Only present if \texttt{SPLIT\_COMPONENTS\_DATA} and \texttt{SPLIT\_PORTS\_DATA} are both disabled.
  \end{dataset}
  \begin{dataset}[type=double,unit={\si{\mol\per\cubic\metre\of{IV}\per\ParamUnit}}]{SENS\_INLET}
    Tensor of sensitivities at the unit operation inlet with components as columns in time-port-major storage.
    Only present if \texttt{SPLIT\_COMPONENTS\_DATA} and \texttt{SPLIT\_PORTS\_DATA} are both disabled.
  \end{dataset}
  \begin{dataset}[type=double,unit={\si{\mol\per\cubic\metre\of{IV}\per\ParamUnit}}]{SENS\_OUTLET\_COMP\_XXX}
    Component \texttt{XXX} of the sensitivity at all outlet ports of the unit operation as matrix in time-major storage.
    Only present if \texttt{SPLIT\_COMPONENTS\_DATA} is enabled and \texttt{SPLIT\_PORTS\_DATA} is disabled.
  \end{dataset}
  \begin{dataset}[type=double,unit={\si{\mol\per\cubic\metre\of{IV}\per\ParamUnit}}]{SENS\_INLET\_COMP\_XXX}
    Component \texttt{XXX} of the sensitivity at all inlet ports of the unit operation inlet as matrix in time-major storage.
    Only present if \texttt{SPLIT\_COMPONENTS\_DATA} is enabled and \texttt{SPLIT\_PORTS\_DATA} is disabled.
  \end{dataset}
  \begin{dataset}[type=double,unit={\si{\mol\per\cubic\metre\of{IV}\per\ParamUnit}}]{SENS\_OUTLET\_PORT\_XXX}
    All components at outlet port \texttt{XXX} of the sensitivity of the unit operation as matrix in time-major storage.
    Only present if \texttt{SPLIT\_COMPONENTS\_DATA} is disabled and \texttt{SPLIT\_PORTS\_DATA} is enabled.
  \end{dataset}
  \begin{dataset}[type=double,unit={\si{\mol\per\cubic\metre\of{IV}\per\ParamUnit}}]{SENS\_INLET\_PORT\_XXX}
    All components at inlet port \texttt{XXX} of the sensitivity of the unit operation inlet as matrix in time-major storage.
    Only present if \texttt{SPLIT\_COMPONENTS\_DATA} is disabled and \texttt{SPLIT\_PORTS\_DATA} is enabled.
  \end{dataset}
  \begin{dataset}[type=double,unit={\si{\mol\per\cubic\metre\of{IV}\per\ParamUnit}}]{SENS\_OUTLET\_PORT\_XXX\_COMP\_YYY}
    Component \texttt{YYY} at outlet port \texttt{XXX} of the sensitivity of the unit operation.
    Only present if \texttt{SPLIT\_COMPONENTS\_DATA} is enabled and \texttt{SPLIT\_PORTS\_DATA} is enabled.
  \end{dataset}
  \begin{dataset}[type=double,unit={\si{\mol\per\cubic\metre\of{IV}\per\ParamUnit}}]{SENS\_INLET\_PORT\_XXX\_COMP\_YYY}
    Component \texttt{YYY} at inlet port \texttt{XXX} of the sensitivity of the unit operation.
    Only present if \texttt{SPLIT\_COMPONENTS\_DATA} is enabled and \texttt{SPLIT\_PORTS\_DATA} is enabled.
  \end{dataset}
  \begin{dataset}[type=double,unit={\si{\mol\per\cubic\metre\of{IV}\per\second\per\ParamUnit}}]{SENSDOT\_BULK}
    Interstitial sensitivity time derivative as $n_{\text{Time}} \times \texttt{UNITOPORDERING}$ tensor in row-major storage
  \end{dataset}
  \begin{dataset}[type=double,unit={\si{\mol\per\cubic\metre\of{MP}\per\second\per\ParamUnit}}]{SENSDOT\_PARTICLE}
    Mobile phase sensitivity time derivative inside the particles as $n_{\text{Time}} \times \texttt{UNITOPORDERING}$ tensor in row-major storage.
    Only present if just one particle type is defined.
  \end{dataset}
  \begin{dataset}[type=double,unit={\si{\mol\per\cubic\metre\of{MP}\per\second\per\ParamUnit}}]{SENSDOT\_PARTICLE\_PARTYPE\_XXX}
    Mobile phase sensitivity time derivative inside the particles of type \texttt{XXX} as $n_{\text{Time}} \times \texttt{UNITOPORDERING}$ tensor in row-major storage.
    Only present if more than one particle type is defined.
  \end{dataset}
  \begin{dataset}[type=double,unit={\si{\mol\per\cubic\metre\of{MP}\per\second\per\ParamUnit}}]{SENSDOT\_SOLID}
    Solid phase sensitivity time derivative inside the particles as $n_{\text{Time}} \times \texttt{UNITOPORDERING}$ tensor in row-major storage.
    Only present if just one particle type is defined.
  \end{dataset}
  \begin{dataset}[type=double,unit={\si{\mol\per\cubic\metre\of{SP}\per\second\per\ParamUnit}}]{SENSDOT\_SOLID\_PARTYPE\_XXX}
    Solid phase sensitivity time derivative inside the particles of type \texttt{XXX} as $n_{\text{Time}} \times \texttt{UNITOPORDERING}$ tensor in row-major storage.
    Only present if more than one particle type is defined.
  \end{dataset}
  \begin{dataset}[type=double,unit={\si{\mol\per\square\metre\per\square\second\per\ParamUnit}}]{SENSDOT\_FLUX}
    Flux sensitivity time derivative as $n_{\text{Time}} \times \texttt{UNITOPORDERING}$ tensor in row-major storage
  \end{dataset}
  \begin{dataset}[type=double,unit={\si{\cubic\per\second\metre\per\ParamUnit}}]{SENSDOT\_VOLUME}
    Volume sensitivity time derivative 
  \end{dataset}
  \begin{dataset}[type=double,unit={\si{\mol\per\cubic\metre\of{IV}\per\second\per\ParamUnit}}]{SENSDOT\_OUTLET}
    Tensor of sensitivity time derivatives at the unit operation outlet with components as columns in time-port-major storage.
    Only present if \texttt{SPLIT\_COMPONENTS\_DATA} and \texttt{SPLIT\_PORTS\_DATA} are both disabled.
  \end{dataset}
  \begin{dataset}[type=double,unit={\si{\mol\per\cubic\metre\of{IV}\per\second\per\ParamUnit}}]{SENSDOT\_INLET}
    Tensor of sensitivity time derivatives at the unit operation inlet with components as columns in time-port-major storage.
    Only present if \texttt{SPLIT\_COMPONENTS\_DATA} and \texttt{SPLIT\_PORTS\_DATA} are both disabled.
  \end{dataset}
  \begin{dataset}[type=double,unit={\si{\mol\per\cubic\metre\of{IV}\per\second\per\ParamUnit}}]{SENSDOT\_OUTLET\_COMP\_XXX}
    Component \texttt{XXX} of the sensitivity time derivative at all outlet ports of the unit operation as matrix in time-major storage.
    Only present if \texttt{SPLIT\_COMPONENTS\_DATA} is enabled and \texttt{SPLIT\_PORTS\_DATA} is disabled.
  \end{dataset}
  \begin{dataset}[type=double,unit={\si{\mol\per\cubic\metre\of{IV}\per\second\per\ParamUnit}}]{SENSDOT\_INLET\_COMP\_XXX}
    Component \texttt{XXX} of the sensitivity time derivative at all inlet ports of the unit operation inlet as matrix in time-major storage.
    Only present if \texttt{SPLIT\_COMPONENTS\_DATA} is enabled and \texttt{SPLIT\_PORTS\_DATA} is disabled.
  \end{dataset}
  \begin{dataset}[type=double,unit={\si{\mol\per\cubic\metre\of{IV}\per\second\per\ParamUnit}}]{SENSDOT\_OUTLET\_PORT\_XXX}
    All components at outlet port \texttt{XXX} of the sensitivity time derivative of the unit operation as matrix in time-major storage.
    Only present if \texttt{SPLIT\_COMPONENTS\_DATA} is disabled and \texttt{SPLIT\_PORTS\_DATA} is enabled.
  \end{dataset}
  \begin{dataset}[type=double,unit={\si{\mol\per\cubic\metre\of{IV}\per\second\per\ParamUnit}}]{SENSDOT\_INLET\_PORT\_XXX}
    All components at inlet port \texttt{XXX} of the sensitivity time derivative of the unit operation inlet as matrix in time-major storage.
    Only present if \texttt{SPLIT\_COMPONENTS\_DATA} is disabled and \texttt{SPLIT\_PORTS\_DATA} is enabled.
  \end{dataset}
  \begin{dataset}[type=double,unit={\si{\mol\per\cubic\metre\of{IV}\per\second\per\ParamUnit}}]{SENSDOT\_OUTLET\_PORT\_XXX\_COMP\_YYY}
    Component \texttt{YYY} at outlet port \texttt{XXX} of the sensitivity time derivative of the unit operation.
    Only present if \texttt{SPLIT\_COMPONENTS\_DATA} is enabled and \texttt{SPLIT\_PORTS\_DATA} is enabled.
  \end{dataset}
  \begin{dataset}[type=double,unit={\si{\mol\per\cubic\metre\of{IV}\per\second\per\ParamUnit}}]{SENSDOT\_INLET\_PORT\_XXX\_COMP\_YYY}
    Component \texttt{YYY} at inlet port \texttt{XXX} of the sensitivity time derivative of the unit operation.
    Only present if \texttt{SPLIT\_COMPONENTS\_DATA} is enabled and \texttt{SPLIT\_PORTS\_DATA} is enabled.
  \end{dataset}
\end{groupscope}

\section{Meta group}

\begin{groupscope}{/meta}{tab:FFMeta}
  \begin{dataset}[type=int,inout={In}]{FILE\_FORMAT}
    Version of the file format (defaults to $030102 = 3.1.2$ if omitted) with two digits per part (Major.Minor.Patch)
  \end{dataset}
  \begin{dataset}[type=string,inout={Out}]{CADET\_VERSION}
    Version of the executed CADET simulator
  \end{dataset}
  \begin{dataset}[type=string,inout={Out}]{CADET\_COMMIT}
    Git commit SHA1 from which the CADET simulator was built
  \end{dataset}
  \begin{dataset}[type=string,inout={Out}]{CADET\_BRANCH}
    Git branch from which the CADET simulator was built
  \end{dataset}
  \begin{dataset}[type=double,unit={\si{\second}},inout={Out}]{TIME\_SIM}
    Time that the time integration took (excluding any preparations and postprocessing)
  \end{dataset}
\end{groupscope}
