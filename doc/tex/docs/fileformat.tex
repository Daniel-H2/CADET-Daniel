%!TEX root = ../all.tex
% =============================================================================
%  CADET - The Chromatography Analysis and Design Toolkit
%  
%  Copyright © 2008-2017: The CADET Authors
%            Please see the AUTHORS and CONTRIBUTORS file.
%  
%  All rights reserved. This program and the accompanying materials
%  are made available under the terms of the GNU Public License v3.0 (or, at
%  your option, any later version) which accompanies this distribution, and
%  is available at http://www.gnu.org/licenses/gpl.html
% =============================================================================

\chapter{CADET File Format Specifications}
The CADET framework is designed to work on a file format structured into groups and datasets. This
concept may be implemented by different file formats.
At the moment, CADET natively supports HDF5 and XML as file formats.
The choice is not limited to those two formats but can be extended as needed.
In this section the general layout and structure of the file format is described.

\section{Global structure}

The global structure (see Fig.~\ref{fig:FFRoot}) is divided into three parts: \texttt{input}, \texttt{output}, and \texttt{meta}.
Every valid CADET file needs an \texttt{input} group (see Fig.~\ref{fig:FFInput}) which contains all relevant information for simulating a model.
It does not need an \texttt{output} (see Fig.~\ref{fig:FFOutput}) or \texttt{meta} (see Fig.~\ref{fig:FFRoot}) group, since those are created when results are written.
Whereas the \texttt{output} group is solely used as output and holds the results of the simulation, the \texttt{meta} group is used for input and output.
Details such as file format version and simulator version are read from and written to the \texttt{meta} group.

If not explicitly stated otherwise, all datasets are mandatory. 
By convention all group names are lowercase, whereas all dataset names are uppercase.
Note that this is just a description of the file format and not a detailed explanation of the meaning of the parameters.
For the latter, please refer to the corresponding sections in the previous chapter.

\begin{figure}[!ht]
\centering
\begin{tikzpicture}[%
  every node/.style={draw=black,semithick,font={\footnotesize\ttfamily}},
  level 2/.style={sibling distance=16mm},
  ]
  \node {/ (Root)} [edge from parent fork down]
    child[sibling distance=30mm] { node {\hyperref[fig:FFInput]{input} } }   
    child[sibling distance=30mm] { node {\hyperref[fig:FFOutput]{output} } }   
    child[sibling distance=30mm] { node {\hyperref[tab:FFMeta]{meta} } };
\end{tikzpicture}
\caption{\label{fig:FFRoot}Structure of the groups in the root group of the file format}
\end{figure}

\begin{figure}[!ht]
\centering
\begin{tikzpicture}[%
  every node/.style={draw=black,semithick,font={\footnotesize\ttfamily}},
  level 2/.style={sibling distance=16mm},
  ]
  \node {\hyperref[sec:FFInput]{input}} [edge from parent fork down]
    child[sibling distance=45mm] { node {\hyperref[tab:FFModelSystem]{model} } [edge from parent fork down]
              child[sibling distance=17mm] { node { \hyperref[tab:FFModelSystemConnections]{connections} } [edge from parent fork down]
                  child { node { \hyperref[tab:FFModelConnectionSwitch]{switch\_000} } }
              }
              child { node { external } [edge from parent fork down]
                  child { node { \hyperref[tab:FFModelExternalSourceLinInterp]{source\_000} } }
              }
              child { node { \hyperref[tab:FFModelSolver]{solver} } }
              child { node { \hyperref[sec:FFModelUnitOp]{unit\_000} } }
          }   
    child[sibling distance=40mm] { node { \hyperref[tab:FFSolver]{solver} } [edge from parent fork down]
              child[sibling distance=15mm] { node { \hyperref[tab:FFSolverSections]{sections} } }
              child[sibling distance=25mm] { node { \hyperref[tab:FFSolverTime]{time\_integrator} } }
          }
    child[sibling distance=28mm] { node { \hyperref[tab:FFReturn]{return} } [edge from parent fork down]
              child[sibling distance=25mm] { node { \hyperref[tab:FFReturnUnit]{unit\_000} } }
          }
    child[sibling distance=23mm] { node { \hyperref[tab:FFSensitivity]{sensitivity} } [edge from parent fork down]
              child[sibling distance=20mm] { node { \hyperref[tab:FFSensitivityParam]{param\_000} } }
          };
\end{tikzpicture}
\caption{\label{fig:FFInput}High-level structure of the groups in the input part of the file format}
\end{figure}

\begin{figure}[!ht]
\centering
\begin{tikzpicture}[%
  every node/.style={draw=black,semithick,font={\footnotesize\ttfamily}},
  ]
  \node { \hyperref[sec:FFModelUnitOp]{unit\_000} } [edge from parent fork down]
        child[sibling distance=23mm] { node { \hyperref[sec:FFAdsorption]{adsorption} } [edge from parent fork down]
                child { node { \hyperref[tab:FFModelUnitOpAdsorptionConsSolver]{consistency\_solver} } }
        }
        child[sibling distance=23mm] { node { \hyperref[tab:FFModelUnitOpDiscretization]{discretization} } [edge from parent fork down]
                child { node { \hyperref[tab:FFModelUnitOpDiscretizationWeno]{weno} } }
        };
\end{tikzpicture}
\caption[Structure of the groups in a column unit operation]{\label{fig:FFModelUnitOpColumn}Structure of the groups in a column unit operation (\texttt{/input/model} group)}
\end{figure}

\begin{figure}[!ht]
\centering
\begin{tikzpicture}[%
  every node/.style={draw=black,semithick,font={\footnotesize\ttfamily}},
  ]
  \node {\hyperref[sec:FFOutput]{output}} [edge from parent fork down]
    child[sibling distance=45mm] { node { \hyperref[tab:FFOutputSolution]{solution} } [edge from parent fork down]
              child[sibling distance=20mm] { node { \hyperref[tab:FFOutputSensitivityUnit]{unit\_000} } }
              child[sibling distance=20mm] { node { \hyperref[tab:FFOutputSensitivityUnit]{unit\_001} } }
    }   
    child[sibling distance=45mm] { node { sensitivity } [edge from parent fork down]
              child[sibling distance=20mm] { node { param\_000 } }
              child[sibling distance=20mm] { node { param\_001 } [edge from parent fork down]
                      child[sibling distance=20mm] { node { \hyperref[tab:FFOutputSensitivityParamUnit]{unit\_000} } }
                      child[sibling distance=20mm] { node { \hyperref[tab:FFOutputSensitivityParamUnit]{unit\_001} } }
              }
          };
\end{tikzpicture}
\caption{\label{fig:FFOutput}Structure of the groups in the output part of the file format}
\end{figure}

\FloatBarrier

\section{Notation and identifiers}

Reference volumes are denoted by subscripts:
\begin{itemize}
  \item[\si{\cubic\metre\of{IV}}] Interstitial volume
  \item[\si{\cubic\metre\of{MP}}] Bead mobile phase volume
  \item[\si{\cubic\metre\of{SP}}] Bead solid phase volume
\end{itemize}

Common notation and identifiers that are used in the subsequent description are listed in Table~\ref{tab:FFNotationIdentifiers}.
\begin{table}[!ht]
\centering
\footnotesize
\begin{tabu}to \linewidth[m]{ll} \toprule
\rowfont[c]\normalfont Identifier & Meaning \everyrow{\midrule}\\
\texttt{NCOMP} & Number of components of a unit operation\\
\texttt{NTOTALCOMP} & Total number of components in the system (sum of all unit operation components)\\
\texttt{NBND}\textsubscript{i} & Number of bound states of component $i$ of a unit operation\\
\texttt{NTOTALBND} & Total number of bound states of a unit operation (sum of all bound states of all components)\\
\texttt{NSTATES} & Maximum of the number of bound states for each component of a unit operation\\
\texttt{NDOF} & Total number of degrees of freedom of the current unit operation model or system of unit operations\\
\texttt{NSEC} & Number of time integration sections\\
\texttt{PARAM\_VALUE} & Value of a generic unspecified parameter \everyrow{}\\
\bottomrule
\end{tabu}
\caption{\label{tab:FFNotationIdentifiers}Common notation and identifiers used in the file format description}
\end{table}

\FloatBarrier
\section{Ordering of multi dimensional data}

Some model parameters, especially in certain binding models, require multi dimensional data.
Since CADET only reads one dimensional arrays, the layout of the data has to be specified (i.e., the way how the data is linearized in memory).
The term ``\emph{xyz}-major'' means that the index corresponding to \emph{xyz} changes the slowest.

For instance, suppose a model with $2$ components and $3$ bound states has a ``component-major'' dataset.
Then, the requested matrix is stored in memory such that all bound states are listed for each component (i.e., the component index changes the slowest and the bound state index the fastest):
\begin{verbatim}
  comp0bnd0, comp0bnd1, comp0bnd2, comp1bnd0, comp1bnd1, comp1bnd2.
\end{verbatim}
This linear array can be represented as a $2 \times 3$ matrix in ``row-major'' storage format, or a $3 \times 2$ matrix in ``column-major'' ordering.

\section{Section dependent model parameters}

Some model parameters (see Table~\ref{tab:FFSectionDependentParams}) can be assigned different values for each section. For example, the velocity a column is operated with could differ in the load, wash, and elution phases.
Section dependency is recognized by specifying the appropriate number of values for the parameters (see \emph{Length} column in the following tables).
If a parameter depends on both the component and the section, the ordering is section-major.

For instance, the \emph{Length} field of the parameter \texttt{VELOCITY} reads ``1 / \texttt{NSEC}'' which means that it is not recognized as section dependent if only $1$ value (scalar) is passed.
However, if \texttt{NSEC} many values (vector) are present, it will be treated as section dependent.

Note that all components of component dependent datasets have to be section dependent (e.g., you cannot have a section dependency on component $2$ only while the other components are not section dependent).

\begin{table}[!ht]
\centering
\footnotesize
\begin{tabu}to \linewidth[m]{lcc} \toprule
\rowfont[c]\normalfont Dataset & Component dependent & Section dependent \everyrow{\midrule}\\      
\texttt{COL\_DISPERSION} & & \checkmark \\
\texttt{FILM\_DIFFUSION} & \checkmark  & \checkmark \\
\texttt{PAR\_DIFFUSION} & \checkmark  & \checkmark \\
\texttt{PAR\_SURFDIFFUSION} & \checkmark  & \checkmark \\
\texttt{VELOCITY} & & \checkmark \everyrow{}\\
\bottomrule
\end{tabu}
\caption[Section dependent datasets in the unit operation models]{\label{tab:FFSectionDependentParams}Section dependent datasets in the unit operation models (\texttt{/input/model/unit\_XXX} group)}
\end{table}

\FloatBarrier
\section{Input group}\label{sec:FFInput}
\subsection{System of unit operations}\label{sec:FFModelSystem}

\newcommand{\GroupHeadline}[1]{\small Group \texttt{#1}}
\newcommand{\GroupHeadlineX}[2]{\small\texttt{#1} -- Group \texttt{#2}}
\newcommand{\GroupHeadlineY}[2]{\small#1 -- Group \texttt{#2}}

\begin{table}[!ht]
\footnotesize
\begin{tabu}to \linewidth[m]{lX[m]cccc} \toprule
\multicolumn{6}{c}{\GroupHeadline{/input/model}} \\
\rowfont[c]\normalfont Dataset & Description & Unit & Type & Range & Length \everyrow{\midrule}\\
\texttt{NUNITS}& Number of unit operations in the system & -- & int & $\geq 1$ & 1 \\
\texttt{INIT\_STATE\_Y}& Initial full state vector (optional, unit operation specific initial data is ignored) & -- & double & $\geq 0$ & \texttt{NDOF} \\
\texttt{INIT\_STATE\_YDOT}& Initial full time derivative state vector (optional, unit operation specific initial data is ignored) & -- & double & $\geq 0$ & \texttt{NDOF} \\
\texttt{INIT\_STATE\_SENSY\_XXX}& Initial full state vector of the \texttt{XXX}th sensitivity system (optional, unit operation specific initial data is ignored) & -- & double & $\geq 0$ & \texttt{NDOF} \\
\texttt{INIT\_STATE\_SENSYDOT\_XXX}& Initial full time derivative state vector of the \texttt{XXX}th sensitivity system (optional, unit operation specific initial data is ignored) & -- & double & $\geq 0$ & \texttt{NDOF} \everyrow{}\\
\bottomrule
\end{tabu}
\caption{\label{tab:FFModelSystem}Datasets in the \texttt{/input/model/} group}
\end{table}

\begin{table}[!ht]
\footnotesize
\begin{tabu}to \linewidth[m]{lX[m]cccc} \toprule
\multicolumn{6}{c}{\GroupHeadline{/input/model/connections}} \\
\rowfont[c]\normalfont Dataset & Description & Unit & Type & Range & Length \everyrow{\midrule}\\
\texttt{NSWITCHES}& Number of valve switches & -- & int & $\geq 1$ & 1 \everyrow{}\\
\bottomrule
\end{tabu}
\caption{\label{tab:FFModelSystemConnections}Datasets in the \texttt{/input/model/connections} group}
\end{table}

\begin{table}[!ht]
\footnotesize
\begin{tabu}to \linewidth[m]{lX[m]cccc} \toprule
\multicolumn{6}{c}{\GroupHeadline{/input/model/connections/switch\_XXX}} \\
\rowfont[c]\normalfont Dataset & Description & Unit & Type & Range & Length \everyrow{\midrule}\\
\texttt{SECTION}& Index of the section which activates this connection set & -- & int & $\geq 0$ & 1 \\
\texttt{CONNECTIONS}& Matrix with list of connections in row-major storage. Columns are \emph{UnitOpID from}, \emph{UnitOpID to}, \emph{Component from}, \emph{Component to}, \emph{volumetric flow rate}. If both component indices are $-1$, all components are connected. & -- & double & $\geq -1$ & $\texttt{NCONNECTIONS} \times 5$ \everyrow{}\\
\bottomrule
\end{tabu}
\caption{\label{tab:FFModelConnectionSwitch}Datasets in the \texttt{/input/model/connections/switch\_XXX} group}
\end{table}

\begin{table}[!ht]
\footnotesize
\begin{tabu}to \linewidth[m]{lX[m]cccc} \toprule
\multicolumn{6}{c}{\GroupHeadlineX{EXTFUN\_TYPE = LINEAR\_INTERP\_DATA}{/input/model/external/source\_XXX}} \\
\rowfont[c]\normalfont Dataset & Description & Unit & Type & Range & Length \everyrow{\midrule}\\
\texttt{VELOCITY} & Velocity of the external profile in positive column axial direction & \si{\per\second} & double & $\geq 0$ & 1\\
\texttt{DATA} & Function values $T$ at the data points & \si{\ExternalUnit} & double & $\mathds{R}$ & Arbitrary\\
\texttt{TIME} & Time of the data points & \si{\second} & double & $\geq 0.0$ & Same as \texttt{DATA}
\everyrow{}\\
\bottomrule
\end{tabu}
\caption{\label{tab:FFModelExternalSourceLinInterp}Datasets in the \texttt{/input/model/external/source\_XXX} group}
\end{table}

\begin{table}[!ht]
\footnotesize
\begin{tabu}to \linewidth[m]{lX[m]cccc} \toprule
\multicolumn{6}{c}{\GroupHeadlineX{EXTFUN\_TYPE = PIECEWISE\_CUBIC\_POLY}{/input/model/external/source\_XXX}} \\
\rowfont[c]\normalfont Dataset & Description & Unit & Type & Range & Length \everyrow{\midrule}\\
\texttt{VELOCITY} & Velocity of the external profile in positive column axial direction & \si{\per\second} & double & $\geq 0$ & 1\\
\texttt{CONST\_COEFF} & Constant coefficients of piecewise cubic polynomial & \si{\ExternalUnit} & double & $\mathds{R}$ & Arbitrary \\
\texttt{LIN\_COEFF} & Linear coefficients of piecewise cubic polynomial & \si{\ExternalUnit\per\second} & double & $\mathds{R}$ & Same as \texttt{CONST\_COEFF} \\
\texttt{QUAD\_COEFF} & Quadratic coefficients of piecewise cubic polynomial & \si{\ExternalUnit\per\square\second} & double & $\mathds{R}$ & Same as \texttt{CONST\_COEFF} \\
\texttt{CUBE\_COEFF} & Cubic coefficients of piecewise cubic polynomial & \si{\ExternalUnit\per\cubic\second} & double & $\mathds{R}$ & Same as \texttt{CONST\_COEFF} \\
\texttt{SECTION\_TIMES} & Simulation times at which a new piece begins (breaks of the piecewise polynomial) & \si{\second} & double & $\geq 0.0$ & \texttt{CONST\_COEFF}+1
\everyrow{}\\
\bottomrule
\end{tabu}
\caption{\label{tab:FFModelExternalSourcePieceCubicPoly}Datasets in the \texttt{/input/model/external/source\_XXX} group}
\end{table}

\begin{table}[!ht]
\footnotesize
\begin{tabu}to \linewidth[m]{lX[m]ccc} \toprule
\multicolumn{5}{c}{\GroupHeadline{/input/model/solver}} \\
\rowfont[c]\normalfont Dataset & Description & Type & Range & Length \everyrow{\midrule}\\
\texttt{GS\_TYPE} & Type of Gram-Schmidt orthogonalization, see IDAS guide
4.5.7.3, 41f. & int &
\begin{tabular}{c}
  0 (\texttt{CLASSICAL\_GS}) \\
  1 (\texttt{MODIFIED\_GS})
\end{tabular} & 1 \\
\texttt{MAX\_KRYLOV} & Defines the size of the Krylov subspace in the iterative linear GMRES solver (0: \texttt{MAX\_KRYLOV} = \texttt{NCOL}) & int & $0-\texttt{NCOL}$ & 1\\
\texttt{MAX\_RESTARTS} & Maximum number of restarts in the GMRES algorithm. If lack of memory isn't an issue, better use a larger Krylov space than restarts & int & $\geq 0$ & 1 \\
\texttt{SCHUR\_SAFETY} & Schur safety factor; Influences the tradeof between linear iterations and nonlinear error control; see IDAS guide 2.1, 5 & double & $\geq 0.0$ & 1\everyrow{}\\
\bottomrule
\end{tabu}
\caption{\label{tab:FFModelSolver}Datasets in the \texttt{/input/model/solver} group}
\end{table}

\FloatBarrier
\subsection{Unit operation models}\label{sec:FFModelUnitOp}

\subsubsection{Inlet}

\begin{table}[!ht]
\footnotesize
\begin{tabu}to \linewidth[m]{lX[m]cccc} \toprule
\multicolumn{6}{c}{\GroupHeadlineX{UNIT\_TYPE = INLET}{/input/model/unit\_XXX}} \\
\rowfont[c]\normalfont Dataset & Description & Unit & Type & Range & Length \everyrow{\midrule}\\
\texttt{UNIT\_TYPE} & Specifies the type of unit operation model & -- & string & \texttt{INLET} & 1 \\
\texttt{NCOMP}& Number of chemical components in the chromatographic media & -- & int  & $\geq 1$ & 1 \\
\texttt{INLET\_TYPE} & Specifies the type of inlet profile & -- & string & \texttt{PIECEWISE\_CUBIC\_POLY} & 1
\everyrow{}\\
\bottomrule
\end{tabu}
\caption[Datasets for the inlet unit operation]{\label{tab:FFModelUnitOpInlet}Datasets for the inlet unit operation (\texttt{/input/model/unit\_XXX} group)}
\end{table}

\begin{table}[!ht]
\footnotesize
\begin{tabu}to \linewidth[m]{lX[m]cccc} \toprule
\multicolumn{6}{c}{\GroupHeadline{/input/model/unit\_XXX/sec\_XXX}} \\
\rowfont[c]\normalfont Dataset & Description & Unit & Type & Range & Length \everyrow{\midrule}\\
\texttt{CONST\_COEFF} & Constant coefficients for inlet concentrations & \si{\mol\per\cubic\metre\of{IV}} & double & $\mathds{R}$ & \texttt{NCOMP} \\
\texttt{LIN\_COEFF} & Linear coefficients for inlet concentrations & \si{\mol\per\cubic\metre\of{IV}\per\second} & double & $\mathds{R}$ & \texttt{NCOMP} \\
\texttt{QUAD\_COEFF} & Quadratic coefficients for inlet concentrations & \si{\mol\per\cubic\metre\of{IV}\per\square\second} & double & $\mathds{R}$ & \texttt{NCOMP} \\
\texttt{CUBE\_COEFF} & Cubic coefficients for inlet concentrations & \si{\mol\per\cubic\metre\of{IV}\per\cubic\second} & double & $\mathds{R}$ & \texttt{NCOMP} 
\everyrow{}\\
\bottomrule
\end{tabu}
\caption{\label{tab:FFModelInletPiecewiseCubicPoly}Datasets in the \texttt{/input/model/unit\_XXX/sec\_XXX} groups}
\end{table}

\FloatBarrier
\subsubsection{Outlet}

\begin{table}[!ht]
\footnotesize
\begin{tabu}to \linewidth[m]{lX[m]cccc} \toprule
\multicolumn{6}{c}{\GroupHeadlineX{UNIT\_TYPE = OUTLET}{/input/model/unit\_XXX}} \\
\rowfont[c]\normalfont Dataset & Description & Unit & Type & Range & Length \everyrow{\midrule}\\
\texttt{UNIT\_TYPE} & Specifies the type of unit operation model & -- & string & \texttt{OUTLET} & 1 \\
\texttt{NCOMP}& Number of chemical components in the chromatographic media & -- & int  & $\geq 1$ & 1
\everyrow{}\\
\bottomrule
\end{tabu}
\caption[Datasets for the outlet unit operation]{\label{tab:FFModelUnitOpOutlet}Datasets for the outlet unit operation (\texttt{/input/model/unit\_XXX} group)}
\end{table}

\FloatBarrier
\subsubsection{General rate model}

\begin{table}[!ht]
\footnotesize
\begin{tabu}to \linewidth[m]{lX[m]cccc} \toprule
\multicolumn{6}{c}{\GroupHeadlineX{UNIT\_TYPE = GENERAL\_RATE\_MODEL}{/input/model/unit\_XXX}} \\
\rowfont[c]\normalfont Dataset & Description & Unit & Type & Range & Length \everyrow{\midrule}\\
\texttt{UNIT\_TYPE} & Specifies the type of unit operation model & -- & string & \texttt{GENERAL\_RATE\_MODEL} & 1 \\
\texttt{NCOMP}& Number of chemical components in the chromatographic media & -- & int  & $\geq 1$ & 1 \\
\texttt{ADSORPTION\_MODEL} & Specifies the type of adsorption model & -- & string & See Section~\ref{sec:FFAdsorption} & 1 \\
\texttt{INIT\_C} & Initial concentrations for each comp.\ in the bulk mobile phase & \si{\mol\per\cubic\metre\of{IV}} & double & $\geq 0.0$ & \texttt{NCOMP}\\
\texttt{INIT\_CP} & Initial concentrations for each comp.\ in the bead liquid phase (optional, \texttt{INIT\_C} is used if left out) & \si{\mol\per\cubic\metre\of{MP}} & double & $\geq 0.0$ & \texttt{NCOMP}\\
\texttt{INIT\_Q} & Same as \texttt{INIT\_C} but for the bound phase & \si{\mol\per\cubic\metre\of{SP}} & double & $\geq 0.0$ & \texttt{NTOTALBND}\\
\texttt{INIT\_STATE} & Full state vector for initialization (optional, \texttt{INIT\_C}, \texttt{INIT\_CP}, and \texttt{INIT\_Q} will be ignored; if length is $2 * \texttt{NDOF}$, then the second half is used for time derivatives) & various & double & -- & \texttt{NDOF} \\
\texttt{COL\_DISPERSION} & Axial dispersion coefficient & \si{\square\metre\of{IV}\per\second} & double & $\geq 0.0$ & 1 / \texttt{NSEC}\\
\texttt{COL\_LENGTH} & Column length & \si{\metre} & double & $> 0.0$ & 1\\
\texttt{COL\_POROSITY} & Column porosity & -- & double & $\geq 0.0$ & 1\\
\texttt{FILM\_DIFFUSION} & Film diffusion coefficients & \si{\metre\per\second} & double & $\geq 0.0$ & \texttt{NCOMP} / {$\texttt{NCOMP} \times \texttt{NSEC}$}\\
\texttt{PAR\_DIFFUSION} & Effective particle diffusion coefficients & \si{\square\metre\of{MP}\per\second} & double & $\geq 0.0$ & \texttt{NCOMP} / {$\texttt{NCOMP} \times \texttt{NSEC}$}\\
\texttt{PAR\_POROSITY} & Particle porosity & -- & double & $> 0.0$ & 1\\
\texttt{PAR\_RADIUS} & Particle radius & \si{\metre} & double & $> 0.0$ & 1\\
\texttt{PAR\_SURFDIFFUSION} & Particle surface diffusion coefficients & \si{\square\metre\of{SP}\per\second} & double & $\geq 0.0$ & \texttt{NTOTALBND} / {$\texttt{NTOTALBND} \times \texttt{NSEC}$}\\
\texttt{VELOCITY} & Interstitial velocity of the mobile phase (optional if \texttt{CROSS\_SECTION\_AREA} is present, see Section~\ref{par:MUOPGRMflow}) & \si{\metre\per\second} & double & & 1 / \texttt{NSEC} \\
\texttt{CROSS\_SECTION\_AREA} & Cross section area of the column (optional if \texttt{VELOCITY} is present, see Section~\ref{par:MUOPGRMflow}) & \si{\square\metre} & double & $> 0.0$ & 1
\everyrow{}\\
\bottomrule
\end{tabu}
\caption[Datasets for the general rate model unit operation]{\label{tab:FFModelUnitOpGRM}Datasets for the general rate model unit operation (\texttt{/input/model/unit\_XXX} group)}
\end{table}

\begin{table}[!ht]
\footnotesize
\begin{tabu}to \linewidth[m]{lX[m]cccc} \toprule
\multicolumn{6}{c}{\GroupHeadlineX{UNIT\_TYPE = GENERAL\_RATE\_MODEL}{/input/model/unit\_XXX/discretization}} \\
\rowfont[c]\normalfont Dataset & Description & Unit & Type & Range & Length \everyrow{\midrule}\\
\texttt{NCOL} & Number of column (axial) discretization cells & -- & int & $\geq 1$ & 1\\
\texttt{NPAR} & Number of particle (radial) discretization cells & -- & int & $\geq 1$ & 1\\
\texttt{NBOUND} & Number of bound states for each component & -- & int & $\geq 0$ & \texttt{NCOMP}\\
\texttt{PAR\_DISC\_TYPE} & Specifies the discretization scheme inside the particles & -- & string
& \begin{tabular}{c}
  \texttt{EQUIDISTANT\_PAR} \\
  \texttt{EQUIVOLUME\_PAR} \\
  \texttt{USER\_DEFINED\_PAR} \\
  \end{tabular} & 1\\
\texttt{PAR\_DISC\_VECTOR} & Node coordinates for the cell boundaries (ignored if $\texttt{PAR\_DISC\_TYPE} \neq \texttt{USER\_DEFINED\_PAR}$) & \si{\metre} & double
  & $[0, 1]$ & \texttt{NPAR}+1 \\
\texttt{USE\_ANALYTIC\_JACOBIAN} & Use analytically computed jacobian matrix (faster) instead of jacobian generated by algorithmic differentiation (slower) & -- & int & 0/1 & 1\\
\texttt{RECONSTRUCTION} & Type of reconstruction method for fluxes & -- & string
& \begin{tabular}{c}
  \texttt{WENO}
  \end{tabular} & 1 \\
\texttt{GS\_TYPE} & Type of Gram-Schmidt orthogonalization, see IDAS guide
4.5.7.3, 41f. & -- & int &
\begin{tabular}{c}
  0 (\texttt{CLASSICAL\_GS}) \\
  1 (\texttt{MODIFIED\_GS})
\end{tabular} & 1 \\
\texttt{MAX\_KRYLOV} & Defines the size of the Krylov subspace in the iterative linear GMRES solver (0: \texttt{MAX\_KRYLOV} = \texttt{NCOL}) & -- & int & $0-\texttt{NCOL}$ & 1\\
\texttt{MAX\_RESTARTS} & Maximum number of restarts in the GMRES algorithm. If lack of memory isn't an issue, better use a larger Krylov space than restarts & -- & int & $\geq 0$ & 1 \\
\texttt{SCHUR\_SAFETY} & Schur safety factor; Influences the tradeof between linear iterations and nonlinear error control; see IDAS guide 2.1, 5 & -- & double & $\geq 0.0$ & 1\everyrow{}\\
\bottomrule
\end{tabu}
\caption[Datasets for the discretization of the general rate model unit operation]{\label{tab:FFModelUnitOpDiscretization}Datasets for the discretization of the general rate model unit operation (\texttt{/input/model/unit\_XXX/discretization} group)}
\end{table}

\FloatBarrier
\subsubsection{Continuously stirred tank reactor model}

\begin{table}[!ht]
\footnotesize
\begin{tabu}to \linewidth[m]{lX[m]cccc} \toprule
\multicolumn{6}{c}{\GroupHeadlineX{UNIT\_TYPE = CSTR}{/input/model/unit\_XXX}} \\
\rowfont[c]\normalfont Dataset & Description & Unit & Type & Range & Length \everyrow{\midrule}\\
\texttt{UNIT\_TYPE} & Specifies the type of unit operation model & -- & string & \texttt{CSTR} & 1 \\
\texttt{NCOMP}& Number of chemical components in the chromatographic media & -- & int  & $\geq 1$ & 1 \\
\texttt{NBOUND} & Number of bound states for each component (optional, defaults to all $0$) & -- & int & $\geq 0$ & \texttt{NCOMP}\\
\texttt{USE\_ANALYTIC\_JACOBIAN} & Use analytically computed Jacobian matrix (faster) instead of Jacobian generated by algorithmic differentiation (optional, defaults to $1$) & -- & int & 0/1 & 1\\
\texttt{ADSORPTION\_MODEL} & Specifies the type of adsorption model (optional, defaults to \texttt{NONE}) & -- & string & See Section~\ref{sec:FFAdsorption} & 1 \\
\texttt{INIT\_C} & Initial concentrations for each comp.\ in the mobile phase & \si{\mol\per\cubic\metre\of{IV}} & double & $\geq 0.0$ & \texttt{NCOMP}\\
\texttt{INIT\_VOLUME} & Initial tank volume & \si{\cubic\metre} & double & $\geq 0.0$ & 1\\
\texttt{INIT\_Q} & Same as \texttt{INIT\_C} but for the bound phase (optional, defaults to all $0$) & \si{\mol\per\cubic\metre\of{SP}} & double & $\geq 0.0$ & \texttt{NTOTALBND}\\
\texttt{INIT\_STATE} & Full state vector for initialization (optional, \texttt{INIT\_C}, \texttt{INIT\_Q}, and \texttt{INIT\_VOLUME} will be ignored; if length is $2 * \texttt{NDOF}$, then the second half is used for time derivatives) & various & double & -- & \texttt{NDOF} \\
\texttt{POROSITY} & Porosity $\varepsilon$ (defaults to $1$) & -- & double & $[0,1]$ & 1\\
\texttt{FLOWRATE\_FILTER} & Flow rate of pure liquid without components (optional, defaults to \SI{0}{\cubic\metre\per\second}) & \si{\cubic\metre\per\second} & double & $\geq 0.0$ & 1 / \texttt{NSEC}
\everyrow{}\\
\bottomrule
\end{tabu}
\caption[Datasets for the continuously stirred tank reactor unit operation]{\label{tab:FFModelUnitOpCSTR}Datasets for the continuously stirred tank reactor unit operation (\texttt{/input/model/unit\_XXX} group)}
\end{table}

\FloatBarrier
\subsection{Flux reconstruction methods}

\begin{table}[!ht]
\footnotesize
\begin{tabu}to \linewidth[m]{lX[m]cc} \toprule
\multicolumn{4}{c}{\GroupHeadlineY{WENO parameters}{/input/model/unit\_XXX/discretization/weno}} \\
\rowfont[c]\normalfont Dataset & Description & Type & Range \everyrow{\midrule}\\
\texttt{BOUNDARY\_MODEL} & Boundary model type: 0 = Lower WENO order (stable), 1 = Zero weights (unstable for small $D_{ax}$), 2 = Zero weights for p $\neq$ 0 (stable?), 3 = Large ghost points & int & $0-3$\\
\texttt{WENO\_EPS} & WENO $\varepsilon$ & double & $\geq 0.0$\\
\texttt{WENO\_ORDER} & WENO Order: 1 = standard upwind scheme, 2, 3; also called WENO K & int & $1-3$ \everyrow{}\\
\bottomrule
\end{tabu}
\caption[Datasets for the WENO reconstruction]{\label{tab:FFModelUnitOpDiscretizationWeno}Datasets for the WENO reconstruction (\texttt{/input/model/unit\_XXX/discretization/weno} group)}
\end{table}

\FloatBarrier
\subsection{Adsorption models}\label{sec:FFAdsorption}

\paragraph{Externally dependent binding models}

Some binding models have a variant that can use external sources as specified in Section~\ref{sec:FFModelSystem} (also see Section~\ref{par:MBFeatureMatrix} and Table~\ref{tab:MBFeatureMatrix} on which binding models support this feature).
For the sake of brevity, only the standard variant of those binding models is specified below.
In order to obtain the format for the externally dependent variant, first replace the binding model name \texttt{XXX} by \texttt{EXT\_XXX}.
Each parameter $p$ (except for reference concentrations \texttt{XXX\_REFC0} and \texttt{XXX\_REFQ}) depends on a (possibly distinct) external source in a polynomial way:
\begin{align*}
  p(T) &= p_{\texttt{TTT}} T^3 + p_{\texttt{TT}} T^2 + p_{\texttt{T}} T + p.
\end{align*}
Thus, a parameter \texttt{XXX\_YYY} of the standard binding model variant is replaced by the four parameters \texttt{EXT\_XXX\_YYY}, \texttt{EXT\_XXX\_YYY\_T}, \texttt{EXT\_XXX\_YYY\_TT}, and \texttt{EXT\_XXX\_YYY\_TTT}.
Since each parameter can depend on a different external source, the dataset \texttt{EXTFUN} (not listed in the standard variants below) should contain a vector of 0-based integer indices of the external source of each parameter.
The ordering of the parameters in \texttt{EXTFUN} is given by the ordering in the standard variant.
However, if only one index is passed in \texttt{EXTFUN}, this external source is used for all parameters.

\begin{table}[!ht]
\footnotesize
\begin{tabu}to \linewidth[m]{lX[m]cccc} \toprule
\multicolumn{5}{c}{\GroupHeadlineY{Nonlinear consistency solver parameters}{/input/model/unit\_XXX/adsorption/consistency\_solver}} \\
\rowfont[c]\normalfont Dataset & Description & Type & Range & Length \everyrow{\midrule}\\
\texttt{SOLVER\_NAME} & Name of the solver & string & \begin{tabular}{@{}c@{}}
  \texttt{LEVMAR} \\
  \texttt{ATRN\_RES} \\
  \texttt{ATRN\_ERR} \\
  \texttt{COMPOSITE} \\
  \end{tabular} & 1\\
\texttt{INIT\_DAMPING} & Initial damping factor (default is $0.01$) & double & $\geq 0.0$ & 1\\
\texttt{MIN\_DAMPING} & Minimal damping factor (default is $0.0001$; ignored by \texttt{LEVMAR}) & double & $\geq 0.0$ & 1\\
\texttt{MAX\_ITERATIONS} & Maximum number of iterations (default is $50$) & int & $> 0$ & 1 \\
\texttt{SUBSOLVERS} & Vector with names of solvers for the composite solver (only required for composite solver) & string & see \texttt{SOLVER\_NAME} & $> 1$ \everyrow{}\\
\bottomrule
\end{tabu}
\caption[Datasets for the nonlinear consistency solver]{\label{tab:FFModelUnitOpAdsorptionConsSolver}Datasets for the nonlinear consistency solver (\texttt{/input/model/unit\_XXX/adsorption/consistency\_solver} group)}
\end{table}

\begin{table}[!ht]
\footnotesize
\begin{tabu}to \linewidth[m]{lX[m]cccc} \toprule
\multicolumn{6}{c}{\GroupHeadlineX{ADSORPTION\_MODEL = LINEAR}{/input/model/unit\_XXX/adsorption}} \\
\rowfont[c]\normalfont Dataset & Description & Unit & Type & Range & Length \everyrow{\midrule}\\
\texttt{IS\_KINETIC} & Selects kinetic or quasi-stationary adsorption mode: 1 = kinetic, 0 = quasi-stationary & -- & int & 0/1 & 1\\
\texttt{LIN\_KA} & Adsorption rate constants & \si{\cubic\metre\of{MP}\per\cubic\metre\of{SP}\per\second} & double & $\geq 0.0$ & \texttt{NCOMP}\\
\texttt{LIN\_KD} & Desorption rate constants & \si{\per\second} & double & $\geq 0.0$ & \texttt{NCOMP} \everyrow{}\\
\bottomrule
\end{tabu}
\caption[Datasets for the linear adsorption model]{\label{tab:FFAdsorptionLinear}Datasets for the linear adsorption model (\texttt{/input/model/unit\_XXX/adsorption} group)}
\end{table}

\begin{table}[!ht]
\footnotesize
\begin{tabu}to \linewidth[m]{lX[m]cccc} \toprule
\multicolumn{6}{c}{\GroupHeadlineX{ADSORPTION\_MODEL = MULTI\_COMPONENT\_LANGMUIR}{/input/model/unit\_XXX/adsorption}} \\
\rowfont[c]\normalfont Dataset & Description & Unit & Type & Range & Length \everyrow{\midrule}\\
\texttt{IS\_KINETIC} & Selects kinetic or quasi-stationary adsorption mode: 1 = kinetic, 0 = quasi-stationary & -- & int & 0/1 & 1\\
\texttt{MCL\_KA} & Adsorption rate constants & \si{\cubic\metre\of{MP}\per\mol\per\second} & double & $\geq 0.0$ & \texttt{NCOMP}\\
\texttt{MCL\_KD} & Desorption rate constants & \si{\per\second} & double & $\geq 0.0$ & \texttt{NCOMP}\\
\texttt{MCL\_QMAX} & Maximum adsorption capacities & \si{\mol\per\cubic\metre\of{SP}} & double & $> 0.0$ & \texttt{NCOMP}\everyrow{}\\
\bottomrule
\end{tabu}
\caption[Datasets for the multi component Langmuir adsorption model]{\label{tab:FFAdsorptionMultiCompLangmuir}Datasets for the multi component Langmuir adsorption model (\texttt{/input/model/unit\_XXX/adsorption} group)}
\end{table}

\begin{table}[!ht]
\footnotesize
\begin{tabu}to \linewidth[m]{lX[m]cccc} \toprule
\multicolumn{6}{c}{\GroupHeadlineX{ADSORPTION\_MODEL = MULTI\_COMPONENT\_ANTILANGMUIR}{/input/model/unit\_XXX/adsorption}} \\
\rowfont[c]\normalfont Dataset & Description & Unit & Type & Range & Length \everyrow{\midrule}\\
\texttt{IS\_KINETIC} & Selects kinetic or quasi-stationary adsorption mode: 1 = kinetic, 0 = quasi-stationary & -- & int & 0/1 & 1\\
\texttt{MCAL\_KA} & Adsorption rate constants & \si{\cubic\metre\of{MP}\per\mol\per\second} & double & $\geq 0.0$ & \texttt{NCOMP}\\
\texttt{MCAL\_KD} & Desorption rate constants & \si{\per\second} & double & $\geq 0.0$ & \texttt{NCOMP}\\
\texttt{MCAL\_QMAX} & Maximum adsorption capacities & \si{\mol\per\cubic\metre\of{SP}} & double & $> 0.0$ & \texttt{NCOMP}\\
\texttt{MCAL\_ANTILANGMUIR} & Anti-Langmuir coefficients (optional) & -- & double & $\{ -1, 1\}$ & \texttt{NCOMP}\everyrow{}\\
\bottomrule
\end{tabu}
\caption[Datasets for the multi component Anti-Langmuir adsorption model]{\label{tab:FFAdsorptionMultiCompAntiLangmuir}Datasets for the multi component Anti-Langmuir adsorption model (\texttt{/input/model/unit\_XXX/adsorption} group)}
\end{table}

\begin{table}[!ht]
\footnotesize
\begin{tabu}to \linewidth[m]{lX[m]cccc} \toprule
\multicolumn{6}{c}{\GroupHeadlineX{ADSORPTION\_MODEL = MOBILE\_PHASE\_MODULATOR}{/input/model/unit\_XXX/adsorption}} \\
\rowfont[c]\normalfont Dataset & Description & Unit & Type & Range & Length \everyrow{\midrule}\\
\texttt{IS\_KINETIC} & Selects kinetic or quasi-stationary adsorption mode: 1 = kinetic, 0 = quasi-stationary & -- & int & 0/1 & 1\\
\texttt{MPM\_KA} & Adsorption rate constants & \si{\cubic\metre\of{MP}\per\mol\per\second} & double & $\geq 0.0$ & \texttt{NCOMP}\\
\texttt{MPM\_KD} & Desorption rate constants & \si{\raiseto{3\beta}\metre\of{MP}\per\raiseto{\beta}\mol\per\second} & double & $\geq 0.0$ & \texttt{NCOMP}\\
\texttt{MPM\_QMAX} & Maximum adsorption capacities & \si{\mol\per\cubic\metre\of{SP}} & double & $\geq 0.0$ & \texttt{NCOMP}\\
\texttt{MPM\_BETA} & Parameters describing the ion-exchange characteristics (IEX) & -- & double & $\geq 0.0$ & \texttt{NCOMP}\\
\texttt{MPM\_GAMMA} & Parameters describing the hydrophobicity (HIC) & \si{\cubic\metre\of{MP}\per\mol} & double & $\geq 0.0$ & \texttt{NCOMP}\everyrow{}\\
\bottomrule
\end{tabu}
\caption[Datasets for the mobile phase modulator adsorption model]{\label{tab:FFAdsorptionMobilePhaseModulator}Datasets for the mobile phase modulator adsorption model (\texttt{/input/model/unit\_XXX/adsorption} group)}
\end{table}

\begin{table}[!ht]
\footnotesize
\begin{tabu}to \linewidth[m]{lX[m]cccc} \toprule
\multicolumn{6}{c}{\GroupHeadlineX{ADSORPTION\_MODEL = STERIC\_MASS\_ACTION}{/input/model/unit\_XXX/adsorption}} \\
\rowfont[c]\normalfont Dataset & Description & Unit & Type & Range & Length \everyrow{\midrule}\\
\texttt{IS\_KINETIC} & Selects kinetic or quasi-stationary adsorption mode: 1 = kinetic, 0 = quasi-stationary & -- & int & 0/1 & 1\\
\texttt{SMA\_KA} & Adsorption rate constants & \si{\raiseto{3}\metre\of{MP}\per\raiseto{3}\metre\of{SP}\per\second} & double & $\geq 0.0$ & \texttt{NCOMP}\\
\texttt{SMA\_KD} & Desorption rate constants & \si{\per\second} & double & $\geq 0.0$ & \texttt{NCOMP}\\
\texttt{SMA\_NU} & Characteristic charges of the protein; The number of sites $\nu$ that the protein interacts with on the resin surface & -- & double & $\geq 0.0$ & \texttt{NCOMP}\\
\texttt{SMA\_SIGMA} & Steric factors of the protein; The number of sites $\sigma$ on the surface that are shielded by the protein and prevented from exchange with the salt counterions in solution & -- & double & $\geq 0.0$ & \texttt{NCOMP}\\
\texttt{SMA\_LAMBDA} & Stationary phase capacity (monovalent salt counterions); The total number of binding sites available on the resin surface & \si{\mol\per\cubic\metre\of{SP}} & double & $\geq 0.0$ & 1\\
\texttt{SMA\_REFC0} & Reference liquid phase concentration (optional, defaults to $1.0$) & \si{\mol\per\raiseto{3}\metre\of{MP}} & double & $> 0.0$ & 1\\
\texttt{SMA\_REFQ} & Reference solid phase concentration (optional, defaults to $1.0$) & \si{\mol\per\raiseto{3}\metre\of{SP}} & double & $> 0.0$ & 1\everyrow{}\\
\bottomrule
\end{tabu}
\caption[Datasets for the steric mass action adsorption model]{\label{tab:FFAdsorptionStericMassAction}Datasets for the steric mass action adsorption model (\texttt{/input/model/unit\_XXX/adsorption} group)}
\end{table}

\begin{table}[!ht]
\footnotesize
\begin{tabu}to \linewidth[m]{lX[m]cccc} \toprule
\multicolumn{6}{c}{\GroupHeadlineX{ADSORPTION\_MODEL = SELF\_ASSOCIATION}{/input/model/unit\_XXX/adsorption}} \\
\rowfont[c]\normalfont Dataset & Description & Unit & Type & Range & Length \everyrow{\midrule}\\
\texttt{IS\_KINETIC} & Selects kinetic or quasi-stationary adsorption mode: 1 = kinetic, 0 = quasi-stationary & -- & int & 0/1 & 1\\
\texttt{SAI\_KA1} & Adsorption rate constants & \si{\raiseto{3}\metre\of{MP}\per\raiseto{3}\metre\of{SP}\per\second} & double & $\geq 0.0$ & \texttt{NCOMP}\\
\texttt{SAI\_KA2} & Adsorption rate constants of dimerization & \si{\raiseto{6}\metre\of{MP}\per\raiseto{6}\metre\of{SP}\per\second} & double & $\geq 0.0$ & \texttt{NCOMP}\\
\texttt{SAI\_KD} & desorption rate constants & \si{\per\second} & double & $\geq 0.0$ & \texttt{NCOMP}\\
\texttt{SAI\_NU} & Characteristic charges $\nu$ of the protein & -- & double & $\geq 0.0$ & \texttt{NCOMP}\\
\texttt{SAI\_SIGMA} & Steric factors $\sigma$ of the protein & -- & double & $\geq 0.0$ & \texttt{NCOMP}\\
\texttt{SAI\_LAMBDA} & Stationary phase capacity (monovalent salt counterions); The total number of binding sites available on the resin surface & \si{\mol\per\cubic\metre\of{SP}} & double & $\geq 0.0$ & 1\\
\texttt{SAI\_REFC0} & Reference liquid phase concentration (optional, defaults to $1.0$) & \si{\mol\per\raiseto{3}\metre\of{MP}} & double & $> 0.0$ & 1\\
\texttt{SAI\_REFQ} & Reference solid phase concentration (optional, defaults to $1.0$) & \si{\mol\per\raiseto{3}\metre\of{SP}} & double & $> 0.0$ & 1\everyrow{}\\
\bottomrule
\end{tabu}
\caption[Datasets for the self association adsorption model]{\label{tab:FFAdsorptionSelfAssociation}Datasets for the self association adsorption model (\texttt{/input/model/unit\_XXX/adsorption} group)}
\end{table}

\begin{table}[!ht]
\footnotesize
\begin{tabu}to \linewidth[m]{lX[m]cccc} \toprule
\multicolumn{6}{c}{\GroupHeadlineX{ADSORPTION\_MODEL = SASKA}{/input/model/unit\_XXX/adsorption}} \\
\rowfont[c]\normalfont Dataset & Description & Unit & Type & Range & Length \everyrow{\midrule}\\
\texttt{IS\_KINETIC} & Selects kinetic or quasi-stationary adsorption mode: 1 = kinetic, 0 = quasi-stationary & -- & int & 0/1 & 1\\
\texttt{SASKA\_H} & Henry coefficient & \si{\cubic\metre\of{MP}\per\cubic\metre\of{SP}\per\second} & double & $\mathds{R}$ & \texttt{NCOMP}\\
\texttt{SASKA\_K} & Quadratic factors & \si{\raiseto{6}\metre\of{MP}\per\cubic\metre\of{SP}\per\mol\per\second} & double & $\mathds{R}$ & $\texttt{NCOMP}^2$ \everyrow{}\\
\bottomrule
\end{tabu}
\caption[Datasets for the Saska adsorption model]{\label{tab:FFAdsorptionSaska}Datasets for the Saska adsorption model (\texttt{/input/model/unit\_XXX/adsorption} group)}
\end{table}

\begin{table}[!ht]
\footnotesize
\begin{tabu}to \linewidth[m]{lX[m]cccc} \toprule
\multicolumn{6}{c}{\GroupHeadlineX{ADSORPTION\_MODEL = MULTI\_COMPONENT\_BILANGMUIR}{/input/model/unit\_XXX/adsorption}} \\
\rowfont[c]\normalfont Dataset & Description & Unit & Type & Range & Length \everyrow{\midrule}\\
\texttt{IS\_KINETIC} & Selects kinetic or quasi-stationary adsorption mode: 1 = kinetic, 0 = quasi-stationary & -- & int & 0/1 & 1\\
\texttt{MCBL\_KA} & Adsorption rate constants in state-major ordering & \si{\cubic\metre\of{MP}\per\mol\per\second} & double & $\geq 0.0$ & $\texttt{NSTATES} \cdot \texttt{NCOMP}$ \\
\texttt{MCBL\_KD} & Desorption rate constants in state-major ordering & \si{\per\second} & double & $\geq 0.0$ & $\texttt{NSTATES} \cdot \texttt{NCOMP}$\\
\texttt{MCBL\_QMAX} & Maximum adsorption capacities in state-major ordering & \si{\mol\per\cubic\metre\of{SP}} & double & $> 0.0$ & $\texttt{NSTATES} \cdot \texttt{NCOMP}$ \everyrow{}\\
\bottomrule
\end{tabu}
\caption[Datasets for the Bi-Langmuir adsorption model]{\label{tab:FFAdsorptionBiLangmuir}Datasets for the Bi-Langmuir adsorption model (\texttt{/input/model/unit\_XXX/adsorption} group)}
\end{table}

\begin{table}[!ht]
\footnotesize
\begin{tabu}to \linewidth[m]{lX[m]cccc} \toprule
\multicolumn{6}{c}{\GroupHeadlineX{ADSORPTION\_MODEL = KUMAR\_MULTI\_COMPONENT\_LANGMUIR}{/input/model/unit\_XXX/adsorption}} \\
\rowfont[c]\normalfont Dataset & Description & Unit & Type & Range & Length \everyrow{\midrule}\\
\texttt{IS\_KINETIC} & Selects kinetic or quasi-stationary adsorption mode: 1 = kinetic, 0 = quasi-stationary & -- & int & 0/1 & 1\\
\texttt{KMCL\_KA} & Adsorption pre-exponential factors & \si{\raiseto{3}\metre\of{MP}\per\mol\per\second} & double & $\geq 0.0$ & \texttt{NCOMP}\\
\texttt{KMCL\_KD} & Desorption rate & \si{\raiseto{3\nu_i}\metre\of{MP}\per\raiseto{\nu_i}\mol\per\second} & double & $\geq 0.0$ & \texttt{NCOMP}\\
\texttt{KMCL\_KACT} & Activation temperatures & \si{\kelvin} & double & $\geq 0.0$ & \texttt{NCOMP}\\
\texttt{KMCL\_QMAX} & Maximum adsorption capacities & \si{\mol\per\cubic\metre\of{SP}} & double & $> 0.0$ & \texttt{NCOMP}\\
\texttt{KMCL\_NU} & Salt exponents / characteristic charges & -- & double & $> 0.0$ & \texttt{NCOMP}\\
\texttt{KMCL\_TEMP} & Temperature & \si{\kelvin} & double & $\geq 0$ & 1 \everyrow{}\\
\bottomrule
\end{tabu}
\caption[Datasets for the Kumar-Langmuir adsorption model]{\label{tab:FFAdsorptionKumarLangmuir}Datasets for the Kumar-Langmuir adsorption model (\texttt{/input/model/unit\_XXX/adsorption} group)}
\end{table}

\begin{table}[!ht]
\footnotesize
\begin{tabu}to \linewidth[m]{lX[m]cccc} \toprule
\multicolumn{6}{c}{\GroupHeadlineX{ADSORPTION\_MODEL = MULTI\_COMPONENT\_SPREADING}{/input/model/unit\_XXX/adsorption}} \\
\rowfont[c]\normalfont Dataset & Description & Unit & Type & Range & Length \everyrow{\midrule}\\
\texttt{IS\_KINETIC} & Selects kinetic or quasi-stationary adsorption mode: 1 = kinetic, 0 = quasi-stationary & -- & int & 0/1 & 1\\
\texttt{MCSPR\_KA} & Adsorption rate constants in state-major ordering & \si{\cubic\metre\of{MP}\per\mol\per\second} & double & $\geq 0.0$ & \texttt{NTOTALBND} \\
\texttt{MCSPR\_KD} & Desorption rate constants in state-major ordering & \si{\per\second} & double & $\geq 0.0$ & \texttt{NTOTALBND} \\
\texttt{MCSPR\_QMAX} & Maximum adsorption capacities in state-major ordering & \si{\mol\per\cubic\metre\of{SP}} & double & $> 0.0$ & \texttt{NTOTALBND} \\
\texttt{MCSPR\_K12} & Exchange rates from the first to the second bound state & \si{\per\second} & double & $\geq 0.0$ & \texttt{NCOMP} \\
\texttt{MCSPR\_K21} & Exchange rates from the second to the first bound state & \si{\per\second} & double & $\geq 0.0$ & \texttt{NCOMP} \everyrow{}\\
\bottomrule
\end{tabu}
\caption[Datasets for the multi component spreading adsorption model]{\label{tab:FFAdsorptionMultiCompSpreading}Datasets for the multi component spreading adsorption model (\texttt{/input/model/unit\_XXX/adsorption} group)}
\end{table}

\begin{table}[!ht]
\footnotesize
\begin{tabu}to \linewidth[m]{lX[m]cccc} \toprule
\multicolumn{6}{c}{\GroupHeadlineX{ADSORPTION\_MODEL = MULTISTATE\_STERIC\_MASS\_ACTION}{/input/model/unit\_XXX/adsorption}} \\
\rowfont[c]\normalfont Dataset & Description & Unit & Type & Range & Length \everyrow{\midrule}\\
\texttt{IS\_KINETIC} & Selects kinetic or quasi-stationary adsorption mode: 1 = kinetic, 0 = quasi-stationary & -- & int & 0/1 & 1\\
\texttt{MSSMA\_KA} & Adsorption rate constants of the components to the different bound states in component-major ordering & \si{\raiseto{3}\metre\of{MP}\per\raiseto{3}\metre\of{SP}\per\second} & double & $\geq 0.0$ & \texttt{NTOTALBND}\\
\texttt{MSSMA\_KD} & Desorption rate constants of the components in the different bound states in component-major ordering & \si{\per\second} & double & $\geq 0.0$ & \texttt{NTOTALBND}\\
\texttt{MSSMA\_NU} & Characteristic charges of the components in the different bound states in component-major ordering & -- & double & $\geq 0.0$ & \texttt{NTOTALBND}\\
\texttt{MSSMA\_SIGMA} & Steric factors of the components in the different bound states in component-major ordering & -- & double & $\geq 0.0$ & \texttt{NTOTALBND}\\
\texttt{MSSMA\_RATES} & Conversion rates between different bound states in component-row-major ordering & \si{\per\second} & double & $\geq 0.0$ & $\sum_{i=0}^{\texttt{NCOMP} - 1} \texttt{NBND}_i^2$\\
\texttt{MSSMA\_LAMBDA} & Stationary phase capacity (monovalent salt counterions); The total number of binding sites available on the resin surface & \si{\mol\per\cubic\metre\of{SP}} & double & $\geq 0.0$ & 1\\
\texttt{MSSMA\_REFC0} & Reference liquid phase concentration (optional, defaults to $1.0$) & \si{\mol\per\raiseto{3}\metre\of{MP}} & double & $> 0.0$ & 1\\
\texttt{MSSMA\_REFQ} & Reference solid phase concentration (optional, defaults to $1.0$) & \si{\mol\per\raiseto{3}\metre\of{SP}} & double & $> 0.0$ & 1\everyrow{}\\
\bottomrule
\end{tabu}
\caption[Datasets for the multi state steric mass action adsorption model]{\label{tab:FFAdsorptionMultiStateStericMassAction}Datasets for the multi state steric mass action adsorption model (\texttt{/input/model/unit\_XXX/adsorption} group)}
\end{table}

\begin{table}[!ht]
\footnotesize
\begin{tabu}to \linewidth[m]{lX[m]cccc} \toprule
\multicolumn{6}{c}{\GroupHeadlineX{ADSORPTION\_MODEL = SIMPLE\_MULTISTATE\_STERIC\_MASS\_ACTION}{/input/model/unit\_XXX/adsorption}} \\
\rowfont[c]\normalfont Dataset & Description & Unit & Type & Range & Length \everyrow{\midrule}\\
\texttt{IS\_KINETIC} & Selects kinetic or quasi-stationary adsorption mode: 1 = kinetic, 0 = quasi-stationary & -- & int & 0/1 & 1\\
\texttt{SMSSMA\_LAMBDA} & Stationary phase capacity (monovalent salt counterions); The total number of binding sites available on the resin surface & \si{\mol\per\cubic\metre\of{SP}} & double & $\geq 0.0$ & 1\\
\texttt{SMSSMA\_KA} & Adsorption rate constants of the components to the different bound states in component-major ordering & \si{\raiseto{3}\metre\of{MP}\per\raiseto{3}\metre\of{SP}\per\second} & double & $\geq 0.0$ & \texttt{NTOTALBND}\\
\texttt{SMSSMA\_KD} & Desorption rate constants of the components to the different bound states in component-major ordering & \si{\per\second} & double & $\geq 0.0$ & \texttt{NTOTALBND}\\
\texttt{SMSSMA\_NU\_MIN} & Characteristic charges of the components in the first (weakest) bound state & -- & double & $\geq 0.0$ & \texttt{NCOMP}\\
\texttt{SMSSMA\_NU\_MAX} & Characteristic charges of the components in the last (strongest) bound state & -- & double & $\geq 0.0$ & \texttt{NCOMP}\\
\texttt{SMSSMA\_NU\_QUAD} & Quadratic modifiers of the characteristic charges of the different components depending on the index of the bound state & -- & double & $\mathds{R}$ & \texttt{NCOMP}\\
\texttt{SMSSMA\_SIGMA\_MIN} & Steric factors of the components in the first (weakest) bound state & -- & double & $\geq 0.0$ & \texttt{NCOMP} \\
\texttt{SMSSMA\_SIGMA\_MAX} & Steric factors of the components in the last (strongest) bound state & -- & double & $\geq 0.0$ & \texttt{NCOMP} \\
\texttt{SMSSMA\_SIGMA\_QUAD} & Quadratic modifiers of steric factors of the different components depending on the index of the bound state & -- & double & $\mathds{R}$ & \texttt{NCOMP} \\
\texttt{SMSSMA\_KWS} & Exchange rates from a weakly bound state to the next stronger bound state & \si{\per\second} & double & $\geq 0.0$ & \texttt{NCOMP} \\
\texttt{SMSSMA\_KWS\_LIN} & Linear exchange rate coefficients from a weakly bound state to the next stronger bound state & \si{\per\second} & double & $\mathds{R}$ & \texttt{NCOMP} \\
\texttt{SMSSMA\_KWS\_QUAD} & Quadratic exchange rate coefficients from a weakly bound state to the next stronger bound state & \si{\per\second} & double & $\mathds{R}$ & \texttt{NCOMP} \\
\texttt{SMSSMA\_KSW} & Exchange rates from a strongly bound state to the next weaker bound state & \si{\per\second} & double & $\geq 0.0$ & \texttt{NCOMP} \\
\texttt{SMSSMA\_KSW\_LIN} & Linear exchange rate coefficients from a strongly bound state to the next weaker bound state & \si{\per\second} & double & $\mathds{R}$ & \texttt{NCOMP} \\
\texttt{SMSSMA\_KSW\_QUAD} & Quadratic exchange rate coefficients from a strongly bound state to the next weaker bound state & \si{\per\second} & double & $\mathds{R}$ & \texttt{NCOMP} \\
\texttt{SMSSMA\_REFC0} & Reference liquid phase concentration (optional, defaults to $1.0$) & \si{\mol\per\raiseto{3}\metre\of{MP}} & double & $> 0.0$ & 1\\
\texttt{SMSSMA\_REFQ} & Reference solid phase concentration (optional, defaults to $1.0$) & \si{\mol\per\raiseto{3}\metre\of{SP}} & double & $> 0.0$ & 1\everyrow{}\\
\bottomrule
\end{tabu}
\caption[Datasets for the simplified multi state steric mass action adsorption model]{\label{tab:FFAdsorptionSimpleMultiStateStericMassAction}Datasets for the simplified multi state steric mass action adsorption model (\texttt{/input/model/unit\_XXX/adsorption} group)}
\end{table}

\begin{table}[!ht]
\footnotesize
\begin{tabu}to \linewidth[m]{lX[m]cccc} \toprule
\multicolumn{6}{c}{\GroupHeadlineX{ADSORPTION\_MODEL = BI\_STERIC\_MASS\_ACTION}{/input/model/unit\_XXX/adsorption}} \\
\rowfont[c]\normalfont Dataset & Description & Unit & Type & Range & Length \everyrow{\midrule}\\
\texttt{IS\_KINETIC} & Selects kinetic or quasi-stationary adsorption mode: 1 = kinetic, 0 = quasi-stationary & -- & int & 0/1 & 1\\
\texttt{BISMA\_KA} & Adsorption rate constants in state-major ordering & \si{\raiseto{3}\metre\of{MP}\per\raiseto{3}\metre\of{SP}\per\second} & double & $\geq 0.0$ & $\texttt{NSTATES} \cdot \texttt{NCOMP}$ \\
\texttt{BISMA\_KD} & Desorption rate constants in state-major ordering & \si{\per\second} & double & $\geq 0.0$ & $\texttt{NSTATES} \cdot \texttt{NCOMP}$\\
\texttt{BISMA\_NU} & Characteristic charges $\nu_{i,j}$ of the $i$th protein with respect to the $j$th binding site type in state-major ordering & -- & double & $\geq 0.0$ & $\texttt{NSTATES} \cdot \texttt{NCOMP}$\\
\texttt{BISMA\_SIGMA} & Steric factors $\sigma_{i,j}$ of the $i$th protein with respect to the $j$th binding site type in state-major ordering & -- & double & $\geq 0.0$ & $\texttt{NSTATES} \cdot \texttt{NCOMP}$ \\
\texttt{BISMA\_LAMBDA} & Stationary phase capacity (monovalent salt counterions) of the different binding site types $\lambda_j$ & \si{\mol\per\cubic\metre\of{SP}} & double & $\geq 0.0$ & \texttt{NSTATES}\\
\texttt{BISMA\_REFC0} & Reference liquid phase concentration for each binding site type or one value for all types (optional, defaults to $1.0$) & \si{\mol\per\raiseto{3}\metre\of{MP}} & double & $> 0.0$ & 1/\texttt{NSTATES} \\
\texttt{BISMA\_REFQ} & Reference solid phase concentration for each binding site type or one value for all types (optional, defaults to $1.0$) & \si{\mol\per\raiseto{3}\metre\of{SP}} & double & $> 0.0$ & 1/\texttt{NSTATES} \everyrow{}\\
\bottomrule
\end{tabu}
\caption[Datasets for the bi steric mass action adsorption model]{\label{tab:FFAdsorptionBiStericMassAction}Datasets for the bi steric mass action adsorption model (\texttt{/input/model/unit\_XXX/adsorption} group)}
\end{table}

\FloatBarrier
\subsection{Return data}

\begin{table}[!ht]
\footnotesize
\begin{tabu}to \linewidth[m]{lX[m]cc} \toprule
\multicolumn{4}{c}{\GroupHeadline{/input/return}} \\
\rowfont[c]\normalfont Dataset & Description & Type & Range \everyrow{\midrule}\\
\texttt{WRITE\_SOLUTION\_TIMES} & Write times at which a solution was produced (optional, defaults to 1) & int & 0/1 \\
\texttt{WRITE\_SOLUTION\_LAST} & Write full solution state vector at last time point (optional, defaults to 0) & int & 0/1 \\
\texttt{WRITE\_SENS\_LAST} & Write full sensitivity state vectors at last time point (optional, defaults to 0) & int & 0/1 \\
\texttt{SPLIT\_COMPONENTS\_DATA} & Determines whether a joint dataset (matrix) for all components is created or if each component is put in a separate dataset (\texttt{XXX\_COMP\_000}, \texttt{XXX\_COMP\_001}, etc.) (optional, defaults to 1) & int & 0/1 \everyrow{}\\
\bottomrule
\end{tabu}
\caption{\label{tab:FFReturn}Datasets in the \texttt{/input/model/return} group}
\end{table}

\begin{table}[!ht]
\footnotesize
\begin{tabu}to \linewidth[m]{lX[m]cc} \toprule
\multicolumn{4}{c}{\GroupHeadline{/input/return/unit\_XXX}} \\
\rowfont[c]\normalfont Dataset & Description & Type & Range \everyrow{\midrule}\\
\texttt{WRITE\_SOLUTION\_COLUMN\_INLET} & Write solutions at column inlet $c_i(t,0)$ & int & 0/1 \\
\texttt{WRITE\_SOLUTION\_COLUMN\_OUTLET} & Write solutions at column outlet (chromatograms) $c_i(t,L)$ & int & 0/1 \\
\texttt{WRITE\_SOLUTION\_COLUMN} & Write solutions of the column bulk volume $c_i$ & int & 0/1 \\
\texttt{WRITE\_SOLUTION\_PARTICLE} & Write solutions of the particle mobile phase $c_{p,i}$ & int & 0/1 \\
\texttt{WRITE\_SOLUTION\_SOLID} & Write solutions of the solid phase $q_{i,j}$ & int & 0/1 \\
\texttt{WRITE\_SOLUTION\_FLUX} & Write solutions of the bead fluxes $j_{f,i}$ & int & 0/1 \\
\texttt{WRITE\_SOLDOT\_COLUMN\_INLET} & Write solution time derivatives at column inlet $\partial c_i(t,0) / \partial t$ & int & 0/1 \\
\texttt{WRITE\_SOLDOT\_COLUMN\_OUTLET} & Write solution time derivatives at column outlet (chromatograms) $\partial c_i(t,L) / \partial t$ & int & 0/1 \\
\texttt{WRITE\_SOLUTION\_VOLUME} & Write solutions of the volume $V$ & int & 0/1 \\
\texttt{WRITE\_SOLDOT\_COLUMN} & Write solution time derivatives of the column bulk volume $\partial c_i / \partial t$ & int & 0/1 \\
\texttt{WRITE\_SOLDOT\_PARTICLE} & Write solution time derivatives of the particle mobile phase $\partial c_{p,i} / \partial t$ & int & 0/1 \\
\texttt{WRITE\_SOLDOT\_SOLID} & Write solution time derivatives of the solid phase $\partial q_{i,j} / \partial t$ & int & 0/1 \\
\texttt{WRITE\_SOLDOT\_FLUX} & Write solution time derivatives of the bead fluxes $\partial j_{f,i} / \partial t$ & int & 0/1 \\
\texttt{WRITE\_SENS\_COLUMN\_INLET} & Write sensitivities at column inlet $\partial c_i(t,0) / \partial p$ & int & 0/1 \\
\texttt{WRITE\_SENS\_COLUMN\_OUTLET} & Write sensitivities at column outlet (chromatograms) $\partial c_i(t,L) / \partial p$ & int & 0/1 \\
\texttt{WRITE\_SOLDOT\_VOLUME} & Write solution time derivatives of the volume $\partial V / \partial t$ & int & 0/1 \\
\texttt{WRITE\_SENS\_COLUMN} & Write sensitivities of the column bulk volume $\partial c_i / \partial p$ & int & 0/1 \\
\texttt{WRITE\_SENS\_PARTICLE} & Write sensitivities of the particle mobile phase $\partial c_{p,i} / \partial p$ & int & 0/1 \\
\texttt{WRITE\_SENS\_SOLID} & Write sensitivities of the solid phase $\partial q_{i,j} / \partial p$ & int & 0/1 \\
\texttt{WRITE\_SENS\_FLUX} & Write sensitivities of the bead fluxes $\partial j_{f,i} / \partial p$ & int & 0/1 \\
\texttt{WRITE\_SENSDOT\_COLUMN\_INLET} & Write sensitivity time derivatives at column inlet $\partial^2 c_i(t,0) / (\partial p, \partial t)$ & int & 0/1 \\
\texttt{WRITE\_SENSDOT\_COLUMN\_OUTLET} & Write sensitivity time derivatives at column outlet (chromatograms) $\partial^2 c_i(t,L) / (\partial p, \partial t)$ & int & 0/1 \\
\texttt{WRITE\_SENS\_VOLUME} & Write sensitivities of the volume $\partial V / \partial p$ & int & 0/1 \\
\texttt{WRITE\_SENSDOT\_COLUMN} & Write sensitivity time derivatives of the column bulk volume $\partial^2 c_i / (\partial p, \partial t)$ & int & 0/1 \\
\texttt{WRITE\_SENSDOT\_FLUX} & Write sensitivity time derivatives of the bead fluxes $\partial^2 j_{f,i} / (\partial p, \partial t)$ & int & 0/1 \everyrow{}\\
\texttt{WRITE\_SENSDOT\_PARTICLE} & Write sensitivity time derivatives of the particle mobile phase $\partial^2 c_{p,i} / (\partial p, \partial t)$ & int & 0/1 \\
\texttt{WRITE\_SENSDOT\_SOLID} & Write sensitivity time derivatives of the solid phase $\partial^2 q_{i,j} / (\partial p, \partial t)$ & int & 0/1 \\
\texttt{WRITE\_SENSDOT\_VOLUME} & Write sensitivity time derivatives of the volume $\partial^2 V / (\partial p, \partial t)$ & int & 0/1 \everyrow{}\\
\bottomrule
\end{tabu}
\caption{\label{tab:FFReturnUnit}Datasets in the \texttt{/input/model/return/unit\_XXX} group}
\end{table}

\FloatBarrier
\subsection{Parameter sensitivities}

\begin{table}[!ht]
\footnotesize
\begin{tabu}to \linewidth[m]{lX[m]cc} \toprule
\multicolumn{4}{c}{\GroupHeadline{/input/sensitivity}} \\
\rowfont[c]\normalfont Dataset & Description & Type & Range \everyrow{\midrule}\\      
\texttt{NSENS} & Number of sensitivities to be computed & int & $\geq 0$\\
\texttt{SENS\_METHOD} & Method used for computation of sensitivities (algorithmic differentiation) & string
& \begin{tabular}{@{}c@{}}
  \texttt{ad1}
  \end{tabular} \everyrow{}\\
\bottomrule
\end{tabu}
\caption{\label{tab:FFSensitivity}Datasets in the \texttt{/input/sensitivity} group}
\end{table}

\begin{table}[!ht]
\footnotesize
\begin{tabu}to \linewidth[m]{lX[m]ccc} \toprule
\multicolumn{5}{c}{\GroupHeadline{/input/sensitivity/param\_XXX}} \\
\rowfont[c]\normalfont Dataset & Description & Type & Range & Length \everyrow{\midrule}\\      
\texttt{SENS\_UNIT} & Unit operation index & int & $\geq 0$ & $\geq 1$\\
\texttt{SENS\_NAME} & Name of the parameter & string & *\footnote{See} & $\geq 1$ \\
\texttt{SENS\_COMP} & Component index ($-1$ if parameter is independent of components) & int & $\geq -1$ & $\geq 1$\\
\texttt{SENS\_REACTION} & Reaction index ($-1$ if parameter is independent of reactions) & int & $\geq -1$ & $\geq 1$\\
\texttt{SENS\_BOUNDPHASE} & Bound phase index ($-1$ if parameter is independent of bound phases) & int & $\geq -1$ & $\geq 1$\\
\texttt{SENS\_SECTION} & Section index ($-1$ if parameter is independent of sections) & int & $\geq -1$ & $\geq 1$\\
\texttt{SENS\_ABSTOL} & Absolute tolerance used in the computation of the sensitivities (optional). Rule of thumb: \texttt{ABSTOL} / \texttt{PARAM\_VALUE} & double & $\geq 0.0$ & $\geq 1$\\
\texttt{SENS\_FACTOR} & Linear factor of the combined sensitivity (optional, taken as $1.0$ if left out) & double & $\mathds{R}$ & $\geq 1$\everyrow{}\\
\bottomrule
\end{tabu}
\caption{\label{tab:FFSensitivityParam}Datasets in the \texttt{/input/sensitivity/param\_XXX} groups}
\end{table}

\FloatBarrier
\subsection{Solver configuration}

\begin{table}[!ht]
\footnotesize
\begin{tabu}to \linewidth[m]{lX[m]cccc} \toprule
\multicolumn{6}{c}{\GroupHeadline{/input/solver}} \\
\rowfont[c]\normalfont Dataset & Description & Unit & Type & Range & Length \everyrow{\midrule}\\      
\texttt{NTHREADS} & Number of used threads & -- & int & $\geq 1$ & 1\\
\texttt{USER\_SOLUTION\_TIMES} & Vector with timepoints at which a solution is desired & \si{\second} & double & $\geq 0.0$ & Arbitrary \\
\texttt{CONSISTENT\_INIT\_MODE} & Consistent initialization mode (optional, defaults to $1$) & -- & int & \begin{tabular}{c}
    0 (none) \\
    1 (full) \\
    2 (once, full) \\
    3 (lean) \\
    4 (once, lean) \\
    5 (full once, then lean) \\
    6 (none once, then full) \\
    7 (none once, then lean)
  \end{tabular} & 1 \\
\texttt{CONSISTENT\_INIT\_MODE\_SENS} & Consistent initialization mode (optional, defaults to $1$) & -- & int & \begin{tabular}{c}
    0 (none) \\
    1 (full) \\
    2 (once, full) \\
    3 (lean) \\
    4 (once, lean) \\
    5 (full once, then lean) \\
    6 (none once, then full) \\
    7 (none once, then lean)
  \end{tabular} & 1 \everyrow{}\\
\bottomrule
\end{tabu}
\caption{\label{tab:FFSolver}Datasets in the \texttt{/input/solver} group}
\end{table}

\begin{table}[!ht]
\footnotesize
\begin{tabu}to \linewidth[m]{lX[m]ccc} \toprule
\multicolumn{5}{c}{\GroupHeadline{/input/solver/time\_integrator}} \\
\rowfont[c]\normalfont Dataset & Description & Type & Range & Length \everyrow{\midrule}\\      
\texttt{ABSTOL} & Absolute tolerance in the solution of the original system & double & $>0.0$ & 1\\
\texttt{ALGTOL} & Tolerance in the solution of the nonlinear consistency equations & double & $>0.0$ & 1\\
\texttt{RELTOL} & Relative tolerance in the solution of the original system & double & $\geq 0.0$ & 1\\
\texttt{RELTOL\_SENS} & Relative tolerance in the solution of the sensitivity systems & double & $\geq 0.0$ & 1 / \texttt{NSENS}\\
\texttt{INIT\_STEP\_SIZE} & Initial time integrator step size for each section or one value for all sections (0.0: IDAS default value), see IDAS guide 4.5, 36f. & double & $\geq 0.0$ & 1 / \texttt{NSEC}\\
\texttt{MAX\_STEPS} & Maximum number of timesteps taken by IDAS (0: IDAS default = 500), see IDAS guide Sec.~4.5 & int & $\geq 0$ & 1 \\
\texttt{MAX\_STEP\_SIZE} & Maximum size of timesteps taken by IDAS (0.0: unlimited), see IDAS guide Sec.~4.5 & double & $\geq 0$ & 1 \everyrow{}\\
\bottomrule
\end{tabu}
\caption{\label{tab:FFSolverTime}Datasets in the \texttt{/input/solver/time\_integrator} group}
\end{table}

\begin{table}[!ht]
\footnotesize
\begin{tabu}to \linewidth[m]{lX[m]cccc} \toprule
\multicolumn{6}{c}{\GroupHeadline{/input/solver/sections}} \\
\rowfont[c]\normalfont Dataset & Description & Unit & Type & Range & Length \everyrow{\midrule}\\
\texttt{NSEC} & Number of sections & -- & int & $\geq 1$ & 1\\
\texttt{SECTION\_TIMES} & Simulation times at which the model changes or behaves discontinously; including start and end times & \si{\second} & double & $\geq 0.0$ & \texttt{NSEC}+1\\
\texttt{SECTION\_CONTINUITY} & Continuity indicator for each section transition & -- & int &  
  \begin{tabular}{c}
    0 (discontinuous) \\
    1 (continuous)
  \end{tabular} & \texttt{NSEC}-1
\everyrow{}\\
\bottomrule
\end{tabu}
\caption{\label{tab:FFSolverSections}Datasets in the \texttt{/input/solver/sections} group}
\end{table}

\FloatBarrier
\section{Output group}\label{sec:FFOutput}

\begin{table}[!ht]
\footnotesize
\begin{tabu}to \linewidth[m]{lX[m]cc} \toprule
\multicolumn{4}{c}{\GroupHeadline{/output/solution}} \\
\rowfont[c]\normalfont Dataset & Description & Unit & Type \everyrow{\midrule}\\      
\texttt{LAST\_STATE\_Y}& Full state vector at the last time point of the time integrator & -- & double \\
\texttt{LAST\_STATE\_YDOT}& Full time derivative state vector at the last time point of the time integrator & -- & double \\
\texttt{LAST\_STATE\_SENSY\_XXX}& Full state vector of the \texttt{XXX}th sensitivity system at the last time point of the time integrator & -- & double \\
\texttt{LAST\_STATE\_SENSYDOT\_XXX}& Full time derivative state vector of the \texttt{XXX}th sensitivity system at the last time point of the time integrator & -- & double \everyrow{}\\
\bottomrule
\end{tabu}
\caption{\label{tab:FFOutput}Datasets in the \texttt{/output} group}
\end{table}

\begin{table}[!ht]
\footnotesize
\begin{tabu}to \linewidth[m]{lX[m]cc} \toprule
\multicolumn{4}{c}{\GroupHeadline{/output/solution}} \\
\rowfont[c]\normalfont Dataset & Description & Unit & Type \everyrow{\midrule}\\      
\texttt{SOLUTION\_TIMES} & Time points at which the solution is written & \si{\second} & double \everyrow{}\\
\bottomrule
\end{tabu}
\caption{\label{tab:FFOutputSolution}Datasets in the \texttt{/output/solution} group}
\end{table}

\begin{table}[!ht]
\footnotesize
\begin{tabu}to \linewidth[m]{lX[m]cc} \toprule
\multicolumn{4}{c}{\GroupHeadline{/output/solution/unit\_XXX}} \\
\rowfont[c]\normalfont Dataset & Description & Unit & Type \everyrow{\midrule}\\      
\texttt{SOLUTION\_COLUMN} & Interstitial solution as $n_{\text{Time}} \times \texttt{UNITOPORDERING}$ tensor in row-major storage & \si{\mol\per\cubic\metre\of{IV}} & double \\
\texttt{SOLUTION\_PARTICLE} & Mobile phase solution inside the beads as $n_{\text{Time}} \times \texttt{UNITOPORDERING}$ tensor in row-major storage & \si{\mol\per\cubic\metre\of{MP}} & double \\
\texttt{SOLUTION\_SOLID} & Solid phase solution as $n_{\text{Time}} \times \texttt{UNITOPORDERING}$ tensor in row-major storage & \si{\mol\per\cubic\metre\of{SP}} & double \\
\texttt{SOLUTION\_FLUX} & Flux solution as $n_{\text{Time}} \times \texttt{UNITOPORDERING}$ tensor in row-major storage & \si{\mol\per\square\metre\per\second} & double \\
\texttt{SOLUTION\_VOLUME} & Volume solution & \si{\cubic\metre} & double \\
\texttt{SOLUTION\_COLUMN\_OUTLET} & Matrix of solutions at the unit operation outlet with components as columns (only present if \texttt{SPLIT\_COMPONENTS\_DATA} is disabled) & \si{\mol\per\cubic\metre\of{IV}} & double \\
\texttt{SOLUTION\_COLUMN\_INLET} & Matrix of solutions at the unit operation inlet with components as columns (only present if \texttt{SPLIT\_COMPONENTS\_DATA} is disabled) & \si{\mol\per\cubic\metre\of{IV}} & double \\
\texttt{SOLUTION\_COLUMN\_OUTLET\_COMP\_XXX} & Component \texttt{XXX} of the solution at the unit operation outlet (only present if \texttt{SPLIT\_COMPONENTS\_DATA} is enabled) & \si{\mol\per\cubic\metre\of{IV}} & double \\
\texttt{SOLUTION\_COLUMN\_INLET\_COMP\_XXX} & Component \texttt{XXX} of the solution at the unit operation inlet (only present if \texttt{SPLIT\_COMPONENTS\_DATA} is enabled) & \si{\mol\per\cubic\metre\of{IV}} & double \\
\texttt{SOLDOT\_COLUMN} & Interstitial solution time derivative as $n_{\text{Time}} \times \texttt{UNITOPORDERING}$ tensor in row-major storage & \si{\mol\per\cubic\metre\of{IV}} & double \\
\texttt{SOLDOT\_FLUX} & Flux solution time derivative as $n_{\text{Time}} \times \texttt{UNITOPORDERING}$ tensor in row-major storage & \si{\mol\per\square\metre\per\second} & double \\
\texttt{SOLDOT\_PARTICLE} & Mobile phase solution time derivative inside the beads as $n_{\text{Time}} \times \texttt{UNITOPORDERING}$ tensor in row-major storage & \si{\mol\per\cubic\metre\of{MP}\per\second} & double \\
\texttt{SOLDOT\_SOLID} & Solid phase solution time derivative as $n_{\text{Time}} \times \texttt{UNITOPORDERING}$ tensor in row-major storage & \si{\mol\per\cubic\metre\of{SP}\per\second} & double \\
\texttt{SOLDOT\_VOLUME} & Volume time derivatives & \si{\cubic\metre\per\second} & double \\
\texttt{SOLDOT\_COLUMN\_OUTLET} & Matrix of solution time derivatives at the unit operation outlet with components as columns (only present if \texttt{SPLIT\_COMPONENTS\_DATA} is disabled) & \si{\mol\per\cubic\metre\of{IV}} & double \\
\texttt{SOLDOT\_COLUMN\_INLET} & Matrix of solution time derivatives at the unit operation inlet with components as columns (only present if \texttt{SPLIT\_COMPONENTS\_DATA} is disabled) & \si{\mol\per\cubic\metre\of{IV}} & double \\
\texttt{SOLDOT\_COLUMN\_OUTLET\_COMP\_XXX} & Component \texttt{XXX} of the solution time derivative at the unit operation outlet (only present if \texttt{SPLIT\_COMPONENTS\_DATA} is enabled) & \si{\mol\per\cubic\metre\of{IV}} & double \\
\texttt{SOLDOT\_COLUMN\_INLET\_COMP\_XXX} & Component \texttt{XXX} of the solution time derivative at the unit operation inlet (only present if \texttt{SPLIT\_COMPONENTS\_DATA} is enabled) & \si{\mol\per\cubic\metre\of{IV}} & double \everyrow{}\\
\bottomrule
\end{tabu}
\caption{\label{tab:FFOutputSensitivityUnit}Datasets in the \texttt{/output/solution/unit\_XXX} group}
\end{table}

\begin{table}[!ht]
\footnotesize
\begin{tabu}to \linewidth[m]{lX[m]cc} \toprule
\multicolumn{4}{c}{\GroupHeadline{/output/sensitivity/param\_XXX/unit\_YYY}} \\
\rowfont[c]\normalfont Dataset & Description & Unit & Type \everyrow{\midrule}\\      
\texttt{SENS\_COLUMN} & Interstitial sensitivity as $n_{\text{Time}} \times \texttt{UNITOPORDERING}$ tensor in row-major storage & \si{\mol\per\cubic\metre\of{IV}} & double \\
\texttt{SENS\_FLUX} & Flux sensitivity as $n_{\text{Time}} \times \texttt{UNITOPORDERING}$ tensor in row-major storage & \si{\mol\per\square\metre\per\second} & double \\
\texttt{SENS\_PARTICLE} & Mobile phase sensitivity inside the beads as $n_{\text{Time}} \times \texttt{UNITOPORDERING}$ tensor in row-major storage & \si{\mol\per\cubic\metre\of{MP}\per\ParamUnit} & double \\
\texttt{SENS\_SOLID} & Solid phase sensitivity as $n_{\text{Time}} \times \texttt{UNITOPORDERING}$ tensor in row-major storage & \si{\mol\per\cubic\metre\of{SP}\per\ParamUnit} & double \\
\texttt{SENS\_VOLUME} & Volume sensitivity & \si{\cubic\metre\per\ParamUnit} & double \\
\texttt{SENS\_COLUMN\_OUTLET} & Matrix of sensitivities at the unit operation outlet with components as columns (only present if \texttt{SPLIT\_COMPONENTS\_DATA} is disabled) & \si{\mol\per\cubic\metre\of{IV}} & double \\
\texttt{SENS\_COLUMN\_INLET} & Matrix of sensitivities at the unit operation inlet with components as columns (only present if \texttt{SPLIT\_COMPONENTS\_DATA} is disabled) & \si{\mol\per\cubic\metre\of{IV}} & double \\
\texttt{SENS\_COLUMN\_OUTLET\_COMP\_XXX} & Component \texttt{XXX} of the sensitivity at the unit operation outlet (only present if \texttt{SPLIT\_COMPONENTS\_DATA} is enabled) & \si{\mol\per\cubic\metre\of{IV}} & double \\
\texttt{SENS\_COLUMN\_INLET\_COMP\_XXX} & Component \texttt{XXX} of the sensitivity at the unit operation inlet (only present if \texttt{SPLIT\_COMPONENTS\_DATA} is enabled) & \si{\mol\per\cubic\metre\of{IV}} & double \\
\texttt{SENSDOT\_COLUMN} & Interstitial sensitivity time derivative as $n_{\text{Time}} \texttt{UNITOPORDERING}$ tensor in row-major storage & \si{\mol\per\cubic\metre\of{IV}} & double \\
\texttt{SENSDOT\_FLUX} & Flux sensitivity time derivative as $n_{\text{Time}} \times \texttt{UNITOPORDERING}$ tensor in row-major storage & \si{\mol\per\square\metre\per\second} & double \\
\texttt{SENSDOT\_PARTICLE} & Mobile phase sensitivity time derivative inside the beads as $n_{\text{Time}} \texttt{UNITOPORDERING}$ tensor in row-major storage & \si{\mol\per\cubic\metre\of{MP}\per\second\per\ParamUnit} & double \\
\texttt{SENSDOT\_SOLID} & Solid phase sensitivity time derivative as $n_{\text{Time}} \texttt{UNITOPORDERING}$ tensor in row-major storage & \si{\mol\per\cubic\metre\of{SP}\per\second\per\ParamUnit} & double \\
\texttt{SENSDOT\_VOLUME} & Volume sensitivity time derivatives & \si{\cubic\metre\per\second\per\ParamUnit} & double \\
\texttt{SENSDOT\_COLUMN\_OUTLET} & Matrix of sensitivity time derivatives at the unit operation outlet with components as columns (only present if \texttt{SPLIT\_COMPONENTS\_DATA} is disabled) & \si{\mol\per\cubic\metre\of{IV}} & double \\
\texttt{SENSDOT\_COLUMN\_INLET} & Matrix of sensitivity time derivatives at the unit operation inlet with components as columns (only present if \texttt{SPLIT\_COMPONENTS\_DATA} is disabled) & \si{\mol\per\cubic\metre\of{IV}} & double \\
\texttt{SENSDOT\_COLUMN\_OUTLET\_COMP\_XXX} & Component \texttt{XXX} of the sensitivity time derivative at the unit operation outlet (only present if \texttt{SPLIT\_COMPONENTS\_DATA} is enabled) & \si{\mol\per\cubic\metre\of{IV}} & double \\
\texttt{SENSDOT\_COLUMN\_INLET\_COMP\_XXX} & Component \texttt{XXX} of the sensitivity time derivative at the unit operation inlet (only present if \texttt{SPLIT\_COMPONENTS\_DATA} is enabled) & \si{\mol\per\cubic\metre\of{IV}} & double \everyrow{}\\
\bottomrule
\end{tabu}
\caption{\label{tab:FFOutputSensitivityParamUnit}Datasets in the \texttt{/output/sensitivity/param\_XXX/unit\_YYY} groups}
\end{table}

\FloatBarrier
\section{Meta group}

\begin{table}[!ht]
\footnotesize
\begin{tabu}to \linewidth[m]{lX[m]cc} \toprule
\multicolumn{4}{c}{\GroupHeadline{/meta}} \\
\rowfont[c]\normalfont Dataset & Description & In / out & Type \everyrow{\midrule}\\      
\texttt{FILE\_FORMAT} & Version of the file format (defaults to 3.0 if omitted) & In & string \\
\texttt{CADET\_VERSION} & Version of the executed CADET simulator & Out & string \\
\texttt{CADET\_COMMIT} & Git commit SHA1 from which the CADET simulator was built & Out & string \\
\texttt{CADET\_BRANCH} & Git branch from which the CADET simulator was built & Out & string \\
\texttt{TIME\_SIM} & Time that the time integration took in seconds (excluding any preparations and postprocessing) & Out & double 
\everyrow{}\\
\bottomrule
\end{tabu}
\caption{\label{tab:FFMeta}Datasets in the \texttt{/meta} group}
\end{table}

\FloatBarrier
